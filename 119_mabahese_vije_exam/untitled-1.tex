\documentclass[12pt]{article}


\usepackage{tabularx}
\usepackage[table]{xcolor}
\usepackage{multirow}



\usepackage{amsmath, amssymb}
\usepackage{mathtools}

\usepackage[margin=1.1in,footskip=.25in]{geometry}

\usepackage{listings}
\usepackage{xcolor}
\usepackage{color}

\definecolor{dkgreen}{rgb}{0,0.6,0}
\definecolor{gray}{rgb}{0.5,0.5,0.5}
\definecolor{mauve}{rgb}{0.58,0,0.82}
 
\definecolor{codegreen}{rgb}{0,0.6,0}
\definecolor{codegray}{rgb}{0.5,0.5,0.5}
\definecolor{codepurple}{rgb}{0.58,0,0.82}
\definecolor{backcolour}{rgb}{0.95,0.95,0.92}

\lstdefinestyle{mystyle}{
    backgroundcolor=\color{backcolour},   
    commentstyle=\color{codegreen},
    keywordstyle=\color{magenta},
    numberstyle=\tiny\color{codegray},
    stringstyle=\color{codepurple},
    basicstyle=\ttfamily\footnotesize,
    breakatwhitespace=false,         
    breaklines=true,                 
    captionpos=b,                    
    keepspaces=true,                 
    numbers=left,                    
    numbersep=5pt,                  
    showspaces=false,                
    showstringspaces=false,
    showtabs=false,                  
    tabsize=3
}
\lstset{style=mystyle}



\usepackage[most]{tcolorbox}

\tcbset{
    frame code={}
    center title,
    left=10pt,
    right=10pt,
    top=10pt,
    bottom=10pt,
    colback=gray!5,
    colframe=gray,
    width=\dimexpr\textwidth\relax,
    enlarge left by=0mm,
    boxsep=5pt,
    arc=0pt,outer arc=0pt,
}


\usepackage{xepersian}
\settextfont[Scale=1]{Vazir}

\renewcommand{\baselinestretch}{1.3} 


\begin{document}

\noindent
1. برنامه ای بنویسید که نام خودتان را در صفحه ی خروجی چاپ کند .




\noindent
2 . برنامه ای بنویسید که 2 عدد صحیح را از ورودی دریافت کند و حاصل جمع آنها را محاسبه کرده و نمایش دهد .




\noindent
3 . برنامه ای بنویسید که 2 عدد اعشاری را از ورودی دریافت کرده و حاصل جمع آنها را محاسبه کرده و نمایش دهد .





\noindent
4 . برنامه ای بنویسید که 3 عدد دلخواه را از ورودی دریافت کرده و میانگین آنها را محاسبه کرده و نمایش دهد .








\noindent
5 . برنامه ای بنویسید که نام و نام خانوادگی شما را دریافت کرده و در یک خط به صورت نام-نام خانوادگی چاپ نماید .







\noindent
6 . برنامه ای بنویسید که طول و عرض یک مستطیل را از ورودی دریافت کرده ، مساحت و محیط مستطیل را محاسبه کند .






\noindent
7 . برنامه ای بنویسید که شعاع یک دایره را دریافت کرده و محیط و مساحت دایره را محاسبه کرده و نمایش دهد .






\noindent
8 . برنامه ای بنویسید که قاعده و ارتفاع یک مثلث را دریافت کرده و مساحت مثلث را محاسبه کرده و چاپ کند .






\noindent
9 . برنامه ای بنویسید که حقوق ناخالص کارمندی را دریافت کرده و با استفاده از قوانین زیر حقوق خالص کارمند را محاسبه نماید و نمایش دهد .

\begin{itemize}
	\item بیمه $=$ حقوق خالص 
	$\times$ 7 درصد
	\item مالیات $=$ حقوق خالص 
	$\times$ 10 درصد
	\item حقوق خالص $=$ حقوق ناخالص $-$ بیمه $-$ مالیات
\end{itemize}











\noindent
10 . برنامه ای بنویسید که سن شما را به روز دریافت کرده و با سال ، ماه ، هفته و روز نمایش دهد .  









\noindent
13 . برنامه ای بنویسید که
\lr{ATM}
 یک عدد صحیح دلخواه را به عنوان پول درخواستی از کاربر دریافت کند و سپس آن مبلغ را به پول های 1 و 5 و 10 و 50 هزار تومانی خرد کند .
 
 
 
 
 
 
 
 
 
 \noindent
14 . برنامه ای بنویسید که بدون استفاده از دستور شرطی 
\lr{if}
،
یک عدد از ورودی دریافت کرده و مقدار قدر مطلق آن را نمایش دهد .







\noindent
15 . برنامه ای بنویسید که دو عدد دلخواه را از ورودی دریافت کرده و ماکزیمم و مینیمم این دو عدد را بدون استفاده از دستور شرطی
\lr{if}
محاسبه کرده و نمایش دهد .


\begin{align*}
Max(x,y) = \cfrac{|x+y| + |x-y|}{2} \qquad \qquad
Min(x,y) = \cfrac{|x+y| - |x-y|}{2}
\end{align*}






\noindent
16 . برنامه ای بنویسید که 2 عدد را از ورودی دریافت کرده و در دو متغیر قرار دهد ، سپس بدون استفاده از متغیر سوم ، مقدار این دو متغیر را با یکدیگر عوض کرده و نمایش دهد .











\noindent
17 . برنامه ای بنویسید که عددی را از ورودی دریافت کرده و مشخص کند عدد وارد شده زوج است یا فرد .
















\noindent
18 . برنامه ای بنویسید که 2 عدد را از ورودی دریافت کرده و عدد بزرگتر را نمایش دهد .










\noindent
19 . برنامه ای بنویسید که نمرات 5 درس یک دانش آموز را دریافت کرده و معدل انش آموز را محاسبه نماید، اگر معدل دانش آموز کمتر از 12 بود ، دانش آموز مشروط ، اگر معدل بیشتر از 17 بود به عنوان دانش آموز ممتاز و در غیر این صورت به عنوان دانش آموز متوسط معرفی نماید .
















\noindent
20 . برنامه ای بنویسید که 4 عدد را از ورودی دریافت کند و بزرگترین عدد را نمایش دهد .












\noindent
21 . برنامه ای بنویسید که با استفاده از دستور 
\lr{switch}
یک عدد از 0 تا 7 را دریافت کرده و نام روز متناسب با آن را نمایش دهد . ( به طور مثال اگر عدد وارد شده 0 بود ، روز شنبه را نمایش دهد و اگر عدد وارد شده 6 بود ، روز جمعه را نمایش دهد  )













\noindent
22 . برنامه ای بنویسید که 4 عدد را از ورودی دریافت کرده و دومین بزرگترین عدد را نمایش دهد .


















\noindent
23 . برنامه ای بنویسید که حقوق ناخالص کارمندی را دریافت کرده و میزان مالیت را بر اساس قوانین زیر محاسبه کند .

\begin{itemize}
	\item اگر حقوق ناخالص کمتر از 1000 بود معاف از مالیات
	\item اگر حقوق ناخالص کمتر از 2000 بود نرخ مالیات
	 5\%
	\item اگر حقوق ناخالص کمتر از 3000 بود نرخ مالیات
	10\%
	\item اگر حقوق ناخالص بیشتر از 3000 بود نرخ مالیات
	15\%
\end{itemize}












\noindent
24 . برنامه ای بنویسید که 3 عدد دلخواه را از ورودی دریافت کرده و مشخص کند که آیا این 3 عدد تشکیل یک مثلث خواهند داد یا خیر .



\begin{align*}
\text{شرط تشکیل مثلث : } \\ 
a+b>c \qquad \&\& \qquad b+c>a \qquad \&\& \qquad a+c>b
\end{align*}










\noindent
25 . برنامه ای بنویسید که 3 عدد دلخواه را به عنوان اضلاع یک مثلث دریافت کند و بررسی کند که این مثلث متساوی الساقین است یا خیر . ( مثلث متساوی الساقین دارای 2 ضلع برابر است )








\noindent
26 . برنامه ای بنویسید که 3 عدد دلخواه را به عنوان اضلاع یک مثلث دریافت کند و بررسی کند که این مثلث متساوی الضلاع است یا خیر . ( مثلث متساوی الاضلاع دارای 3 ضلع برابر است )














\noindent
27 . رنامه ای بنویسید که 3 عدد دلخواه را به عنوان اضلاع یک مثلث دریافت کند و بررسی کند که این مثلث قائم الزاویه است یا خیر . 

در مثلث قائم الزاویه یکی از روابط زیر برقرار است :


\begin{align*}
\colorbox{gray!10}{\parbox{90pt}{
$$a^{2} = b^{2} + c^{2}$$
}}
\qquad
\colorbox{gray!10}{\parbox{90pt}{
$$b^{2} = a^{2} + c^{2}$$
}}
\qquad
\colorbox{gray!10}{\parbox{90pt}{
$$c^{2} = a^{2} + b^{2} $$
}}
\end{align*}













\noindent
29 . برنامه ای بنویسید که اعداد 1 تا 100 را چاپ کند .








\noindent
30 . برنامه ای بنویسید که حاصل جمع اعداد 1 تا 100 را چاپ کند .












\noindent
31 . برنامه ای بنویسید که اعداد زوج بین 1 تا 100 را چاپ کند .














\noindent
32 . برنامه ای بنویسید که حاصل جمع اعداد فرد بین 1 تا 100 را چاپ کند .













\noindent
33 . برنامه ای بنویسید که یک عدد صحیح را از ورودی دریافت کرده و اعداد کوچکتر از آن را چاپ نماید .















\noindent
34 . برنامه ای بنویسید که یک عدد را از ورودی دریافت کرده و مقسوم علیه های آن را چاپ کند .













\noindent
35 . برنامه ای بنویسید که یک عدد را از ورودی دریافت کرده و مجموعه مقسوم علیه های آن را چاپ کند .










\noindent
36 . برنامه ای بنویسید که یک عدد را از ورودی دریافت کرده و تعداد مقسوم علیه های آن را چاپ کند .


















\noindent
37 . برنامه ای بنویسید که یک عدد را از ورودی دریافت کرده و مجموع مقسوم علیه های فرد آن را چاپ کند .















\noindent
38 . برنامه ای بنویسید که یک عدد را از ورودی دریافت کرده و تعداد مقسوم علیه های زوج آن را چاپ کند .
















\noindent
39 . برنامه ای بنویسید که یک عدد را از ورودی دریافت کرده و مشخص کند که عدد وارد شده عدد اول است یا خیر . ( عددی اول است که به غیر از 1 و خودش مقسوم علیه دیگری نداشته باشد ) 



















\noindent
40 . برنامه ای بنویسید که یک عدد را از ورودی دریافت کرده و مشخص کند که عدد وارد شده عدد کامل است یا خیر . ( عددی کامل است که مجموع مقسوم علیه های به غیر از خودش با خود عدد برابر باشد )













\noindent
41 . برنامه ای بنویسید که با استفاده از حلقه ی تکرار 10 عدد را از ورودی دریافت کند و میانگین آنها را نمایش دهد .













\noindent
42 . برنامه ای بنویسید که با استفاده از حلقه ی تکرار 10 عدد را از ورودی دریافت کند و بزرگترین آنها را نمایش دهد .














\noindent
43 . برنامه ای بنویسید که با استفاده از حلقه ی تکرار 10 عدد را از ورودی دریافت کند و دومین بزرگترین آنها را نمایش دهد .
















\noindent
44 . برنامه ای بنویسید که عددی را از ورودی دریافت کرده و تعداد ارقام آن را نمایش دهد . ( برای مثال عدد 123 ، 3 رقم دارد )

















\noindent
45 . برنامه ای بنویسید که یک عدد صحیح را از ورودی دریافت کرده و عدد وارد شده را مقلوب نماید . ( برای مثال مقلوب عدد 123 ، عدد 321 است )















\noindent
46 . برنامه ای بنویسید که یک عدد صحیح را دریافت کرده و مجموع اعداد زوج آن را نمایش دهد . ( برای مثال مجموع اعداد زوج 249 ، عدد 6 می باشد )








\noindent
47 . برنامه ای بنویسید که یم عدد صحیح را دریافت کرده و تعداد اعداد فرد آن را نمایش دهد . ( برای مثال تعداد اعداد فرد عدد 163 ، عدد 2 می باشد )














\noindent
49 . برنامه ای بنویسید که 50 جمله ی اول سری فیبوناتچی را چاپ کند .

$$
0 , 1 , 1 , 2 , 3 , 5 , 8 , 13 , 21 , 34 , 55 , \dots
$$


\begin{align*}
F(n) = 
\begin{cases}
0 & n = 0 \\
1 & n = 1 \\
F(n-1) + F(n-2) & n > 1 
\end{cases}
\end{align*}








\noindent
51 . برنامه ای بنویسید که یک جدول ضرب 10 در 10 را چاپ نماید .
















\noindent
54 . برنامه ای بنویسید که یک عدد دلخواه را از ورودی دریافت کرده و معادل مبنای 2 آن را نمایش دهد .














\noindent
55 . برنامه ای بنویسید که یک عدد دلخواه را از ورودی دریافت کرده و معادل مبنای 8 آن را نمایش دهد .















\noindent
59 . برنامه ای بنویسید که یک عدد را از ورودی دریافت کرده و فاکتوریل آن را محاسبه نماید .

\begin{tcolorbox}
$$
n ! = 1 \times 2 \times 3 \times \dots \times (n-1) \times (n-2) \times n
$$
\end{tcolorbox}






\end{document}
\documentclass[12pt]{article}

\renewcommand{\familydefault}{\sfdefault}

\begin{document}

\section{Difference between DA and DBA}


\subsection{DA}

\begin{itemize}
	\item DA is a person in the organization who controls the data of the database.
	\item DA decides what data needs to be stored in database based on requirements of the organization.
	\item Involvement of DA is more in requirements gathering and analysis phase of software development.
	\item DA is a top level person of an organization or may be a manager having good knowledge of organizational requirements of data.
	\item It is ok if DA does not have any technical knowledge, but any kind of knowledge about database can be more fruitful.
	\item Main focus of DA is in  business, but he should understand more about database.
\end{itemize}


\subsection{DBA}

\begin{itemize}
	\item DBA is a person in the organization who design and access the database.
	\item DBA provides a technical support to implement a database.
	\item Involvement of DBA is more in database design, development, testing and operational phase of software development.
	\item DBA is a technical person having good knowledge of database technology.
	\item It is ok if DBA does not have any business knowledge but any kind of knowledge about a functionality of an organization can be more fruitful.
	\item Main focus of DBA is to provide technical support, but he should understand more about the business to design the database efficiently.
\end{itemize}



\newpage



\section{DA vs DBA}
DA (Data Administrator) and DBA (Database Administrator) both are responsible for managing database for an organization.
They differ from each other in their required skills and responsibilities.


\subsection{Data Administrator (DA)}

\begin{itemize}
	\item "Person in the organization who controls the data of the database refers data administrator."
	\item DA determines what data to be stored in database based on requirement of the organization.
	\item DA works on such as requirements gathering, analysis, and design phases.
	\item DA does not to be a technical person, any kind of knowledge about database technology can be more beneficiary
	\item DA is some senior level person in the organization. in short, DA is a business focused person but should understand about the database technology.
\end{itemize}


\subsection{Database Administrator (DBA)}

\begin{itemize}
	\item "Person in the organization who controls the design and the use of the database refers database administrator."
	\item DBA provides necessary technical support for implementing a database.
	\item DBA works on such as design, development , testing, and operational phases.
	\item DBA is a technical person having knowledge of database technology.
	\item DBA does not need to be a business person. in short, DBA is a technically focused person but should understand about the business to administrator the database effectively.
\end{itemize}



\end{document}
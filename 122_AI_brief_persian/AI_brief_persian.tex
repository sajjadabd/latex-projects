\documentclass[12pt]{article}

\usepackage{tabularx}
\usepackage[table]{xcolor}
\usepackage{multirow}

\usepackage[margin=1.1in,footskip=.25in]{geometry}

\usepackage[most]{tcolorbox}

\tcbset{
    frame code={}
    center title,
    left=10pt,
    right=10pt,
    top=10pt,
    bottom=10pt,
    colback=gray!5,
    colframe=gray,
    width=\dimexpr\textwidth\relax,
    enlarge left by=0mm,
    boxsep=5pt,
    arc=0pt,outer arc=0pt,
}

\usepackage{amsmath, amssymb}
\usepackage{mathtools}

\usepackage{graphics}
\usepackage{graphicx}

\usepackage{tikz}
\usetikzlibrary{positioning}
\usetikzlibrary{patterns}
\usetikzlibrary{matrix,backgrounds}
\usetikzlibrary{arrows,shapes,trees}
\usetikzlibrary{chains,shapes}
\usetikzlibrary{arrows}

\usepackage{xepersian}
\settextfont[Scale=1]{Vazir}

\renewcommand{\baselinestretch}{1.3} 

\begin{document}


\noindent
هوش مصنوعی از 4 دیدگاه تعریف می شود .

\begin{itemize}
	\item عملکرد انسانگونه 
	\item تفکر انسانگونه
	\item تفکر منطقی
	\item عملکرد منطقی
\end{itemize}


\vspace{30pt}

\noindent
قابلیت هایی که دستگاه هوش مصنوعی باید داشته باشد تا بتواند در آزمون تورینگ شرکت کند :

\begin{itemize}
	\item قابلیت نمایش دانش
	\item استدلال خودکار
	\item پردازش زبان
	\item یادگیری ماشین
\end{itemize}

\vspace{30pt}

\noindent
عملکرد منطقی یا رفتار منطقی 
$\leftarrow$
انجام کار درست را رفتار منطقی گویند .

\noindent
کار درست 
$\leftarrow$
 کاری است که قاعدتاً در راستای رسیدن به هدف بیشترین موفقیت را دارد .


\vspace{10pt}


\noindent
عامل
\lr{(Agent) : }
چیزی است که بتواند قابلیت درک مطلب و اجرای کاری را داشته باشد .


\vspace{10pt}

\noindent
عامل ها و محیط ها : 

\noindent
هر عاملی با محیط اطراف خود در ارتباط است و این عامل از طریق حسگرها اطلاعات محیط را درک می کند ، سپس از طریق عملگرها روی محیط تاثیر می گذارد .


\vspace{10pt}


\noindent
دنیای جاروبرقی : 
2 اتاق وجود دارد و یک عامل جاروبرقی ، یکی از اتاق ها تمیز و دیگری کثیف است ، بر روی جاروبرقی 2 سنسور وجود دارد ؛ اولی مکان آن و دیگری تمیز بودن یا کثیف بودن اتاق فعلی را سنس می کند .

\noindent
اکشن ها : تمیز کردن ، حرکت کردن ( راست ، چپ ) ، 
\lr{No OP}




\newpage


\section{عامل های منطقی}

\noindent
عامل که بر اساس اطلاعاتی که از سنسور دریافت می کند و اعمالی که می تواند انجام دهد و دانش داخلی اش همواره بتواند کاری را انجام دهد که قاعدتاً در راه رسیدن به هدف بیشترین موفقیت را کسب کند .


\vspace{10pt}


\noindent
معیار کارایی : 

\noindent
معیاری است که رفتار عامل را در راه رسیدن به هدف می سنجد .

\vspace{10pt}


\noindent
عامل های عقل کل : 

\noindent
عاملی عقل کل است که نتیجه ی واقعی اعمالش را بداند . 



\vspace{10pt}



\noindent
عامل های اکتشافی :

\noindent
این عامل ها اکشن هایی که انجام می دهند ممکن است در راستای هدف نباشد و برای جمع آوری اطلاعات باشد .


\vspace{10pt}


\noindent
عامل های خود مختار : 

\noindent
عاملی خودمختار است که اکشن های آن بر اساس ادراکاتش اجرا شود . 

\noindent
خود مختاری وقتی که اکشن ها توسط ما به حسگر ها داده شوند از درجه ی پایین تری برخوردار است .

\noindent
شی ای که بر اساس یادگیری خود عمل می کند درجه ی خود 
مختاری بالاتری دارد .


\vspace{30pt}


\noindent
\lr{PEAS} :

\noindent
برای طراحی یک عامل به چهار چیز احتیاج است که به آنها 
\lr{PEAS}
می گویند .

\begin{itemize}
	\item معیار کارایی
	\lr{(Performace Measure)}
	\item محیط
	\lr{(Environment)}
	\item عملگر ها 
	\lr{(Actuators)}
	\item حسگر ها 
	\lr{(Sensors)}
\end{itemize}




\newpage

\section{انواع محیط}

\noindent
دسته بندی اول :
\begin{itemize}
	\item محیط های کاملاً مشاهده پذیر 
	\lr{(Full Observable)} :
	
	اگر کلیه ی اطلاعات از طریق حسگرها در دسترس باشد ، به این محیط ، محیط کاملاً مشاهده پذیر گویند .
	\item محیط های نسبتاً مشاهده پذیر
	\lr{(Partially Observable)} :
	
	اگر فقط برخی اطلاعات محیط از طریق حسگرها در دسترس باشند ، به آن محیط نسبتاً مشاهده پذیر می گویند .
\end{itemize}



\vspace{10pt}


\noindent
دسته بندی دوم :

\begin{itemize}
	\item  محیط های قطعی یا معین
	\lr{()}
	\item محیط های غیر قطعی یا نامعین
	\lr{(Stochastic)}
\end{itemize}



\noindent
اگر در یک محیط حالت فعلی مشخص باشد ، با مشخص شدن عملی که می خواهیم انجام دهیم ، حالت بعدی محیط به صورت قطع معین خواهد شد ، به این محیط ، محیط معین گویند .



\vspace{10pt}



\noindent
دسته بندی سوم :

\begin{itemize}
	\item محیط های اپیزودیک
	\lr{(Episodic)}
	\item محیط های غیر اپیزودیک
	\lr{()}
\end{itemize}


\noindent
اگر محیط را به چند اپیزود تقسیم کنیم و اکشن های هر اپیزود به اپیزود جاری مربوط باشد و به اپیزود های قبلی و بعدی ارتباطی نداشته باشد ، محیط اپیزودیک است ، در غیر این صورت محیط غیر اپیزودیک است .


\newpage

\noindent
دسته بندی چهارم : 

\begin{itemize}
	\item محیط های ایستا
	\lr{(Static)}
	\item محیط های پویا
	\lr{(Dynamic)}
\end{itemize}

\noindent
اگر در مدت سنجش محیط توسط عامل و اتخاذ تصمیم لازم ، محیط هیچ تغییری نکند ، آنگاه محیط ایستا است . در غیر این صورت محیط پویا است .

\noindent
محیط های نیمه پویا :

\noindent
محیط ذاتاً ایستا است اما با گذشت زمان معیار کارایی عامل تغییر پیدا کند . مثال : بازی شطرنج با ساعت .



\vspace{10pt}


\noindent
دسته بندی پنجم :

\begin{itemize}
	\item محیط های گسسته
	\lr{(Discrete)}
	
	اگر در یک محیط تعداد اعمال محدود باشد و بتوان آنها را به طور مشخص تعریف کرد ، محیط گسسته خواهد بود 
	\item محیط های پیوسته
	\lr{(Continous)}
	اگر در یک محیط تعداد اعمال نا محدود باشد و نتوان آنها را به طور مشخص تعریف کرد ، محیط پیوسته خواهد بود 
\end{itemize}




\vspace{10pt}


\noindent
دسته بندی ششم :

\begin{itemize}
	\item محیط های تک عامله 
	\lr{(Single)}
	
	اگر در یک محیط فقط یک عامل باشد ، محیط تک عامله  است .
	\item محیط های چند عامله
	\lr{(Multi)} 
	
	اگر در یک محیط چندین عامل وجود داشته باشد ، محیط چند عامله است
\end{itemize}





\vspace{30pt}


\begin{tcolorbox}
عامل های مستقل : 
کار و اهداف عامل ها به همدیگر ارتباطی ندارد

\vspace{10pt}

عامل های همکار : 
معمولاً هدف مشترک دارند و برای رسیدن به این هدف مشترک ، به یکدیگر کمک می کنند .

\vspace{10pt}

عامل های رقابتی :
معمولاً هدف آنها با هم در تضاد هستند .
\end{tcolorbox}




\newpage

\section{انواع عامل}

\noindent
انواع عامل ها بر اساس برنامه ی عامل :


\begin{itemize}
	\item عامل های واکنشی ساده
	\item عامل های واکنشی مبتنی بر مدل
	\item عامل های هدفمند ( هدف گرا )
	\item عامل مبتنی بر سودمندی ( مطلوبیت )
\end{itemize}


\noindent
عامل های واکنشی ساده 
\lr{(Simple Reflex)} :

\noindent
دارای جدول جستجوی ساده هستند . در آنها تعدادی از وضعیت ها می توانند توسط قانون های شرط - عملکرد خلاصه شوند . پیاده سازی این نوع عامل ها آسان می باشد ولی دارای کاربرد کمی می باشند 


\vspace{10pt}

\noindent
عامل های واکنشی مبتنی بر مدل 
\lr{(Reflex Based Model)} :

\noindent
اطلاعات عامل به تنهایی در مورد محیط های نسبتاً مشاهده پذیر کافی نیستند ، لازم است که جریان تغییرات جهان را نیز نگهداری نماییم ( حافظه یا مدل )


\vspace{10pt}



\noindent
عامل های هدفمند ( مبتنی بر هدف ) 
\lr{(Goal Based Agents)} :

\noindent
در اینگونه عامل ها وضعیت و عملکرد ها نمی گویند که کجا برویم . از قوانین یکسان برای اهداف مختلف استفاده می نماید .


\vspace{10pt}



\noindent
عوامل مبتنی بر سودمندی 
\lr{(Utility Based Agents)} :

\noindent
مانند عوامل مبتنی بر هدف است و تفاوت آن در بررسی وضعیت به جای هدف ، رضایت کار را بررسی می کند و کار را بر اساس رضایت بیشتر انجام می دهد . 



\newpage

\noindent
عامل های یادگیری 
\lr{(Learning Agents)} :

\noindent
المان کارایی 
\lr{(Performance Measure)}
: بر اساس ورودی های سنس شده عمل مناسب را انجام  می دهد .

\vspace{10pt}

\noindent
وظیفه ی نقاد : 

\noindent
بررسی اطلاعات دریافتی و مقایسه ی آنها با اطلاعات استاندارد کارایی می باشد ، نتیجه ی این بررسی بازخورد است .


\vspace{10pt}

\noindent
وظیفه ی المان یادگیری : 

\noindent
دریافت بازخورد است که بر اساس آن در المان کارایی تغییر ایجاد می کند که این کار دانش را نتیجه می دهد .






\vspace{30pt}

\noindent
حل مسئله :

\noindent
جستجو ابتدایی ترین روش برای حل یک مسئله است .


\noindent
برای حل یک مسئله از طریق جستجو سه کار :


\begin{itemize}
	\item فرموله سازی هدف
	\item فرموله سازی مسئله
	\item جستجو
\end{itemize}

\noindent
باید انجام شود .



\vspace{30pt}

\noindent
انواع مسائل :

\noindent
مسائل تک حالته 
\lr{(Single State)} :

\noindent
مسائلی هستند که در آنها محیط قطعی و کاملاً مشاهده پذیر است . در این مسائل راه حل دنباله ای از اکشن ها است.


\vspace{10pt}

\noindent
مسائل بدون حسگر 
\lr{(Sensor Less)} :

\noindent
محیطی غیر قابل مشاهده دارند . در این مسائل راه حل دنباله ای از اکشن ها است .


\vspace{10pt}

\noindent
مسائل احتمالی 
\lr{(Contingency)} :

\noindent
مسائلی هستند که محیط آنها غیر قطعی و یا مشاهده پذیر نسبی است که مهم غیر قطعی بودن محیط است . در این حالت پس از انجام کار محیط دوباره سنس شده تا تغییرات بررسی گردد .


\newpage


\noindent
مسائل اکتشافی 
\lr{(Exploration)} :

\noindent
مسائلی هستند که فضای حالت آنها نامشخص است یعنی امکان ایجاد وضعیت وجود ندارد ، در اینگونه مسائل برای رسیدن به هدف کار انجام نمی شود ، بلکه برای شناسایی محیط ، کار انجام می شود .





\vspace{30pt}


\noindent
چهار مورد زیر برای فرموله سازی مسئله استفاده می شوند .



\begin{itemize}
	\item حالت ها
	\item اعمال
	\item آزمون هدف
	\item هزینه ی مسیری
\end{itemize}



\noindent
حالت : 
به هر چیدمان یا صورت وضعیت محیط یک حالت می گوییم .


\vspace{10pt}



\noindent
حالت اولیه : حالتی است که جستجو از آن آغاز می شود .


\vspace{10pt}


\noindent
اعمال : تعداد اکشن های ممکن برای عامل


\vspace{10pt}


\noindent
تابع پسین 
\lr{(Successor Function)} :
تابعی است که مشخص می کند در هر حالتی در صورتی که اکشن خاصی انجام شود ، محیط به چه حالتی می رود .


\vspace{10pt}


\noindent
آزمون هدف : یعنی وقتی به حالتی رسیده ایم توانایی بررسی آن را داشته باشیم و مشخص کنیم به هدف رسیده ایم یا خیر .



\vspace{10pt}


\noindent
هزینه ی مسیری : بیانگر مجموعه هزینه های پرداخت شده از حالت شروع تا حالت فعلی است .


\vspace{10pt}


\noindent
برای مثال : دنیای جارو برقی 

\noindent
تعداد حالت ها : 8 حالت 

\noindent
اکشن ها :
\lr{Left, Right , Clear}

\noindent
آزمون هدف : هر دو اتاق تمیز است 


\noindent
هزینه : تعداد اکشن ها 






\newpage


\noindent
برای بررسی الگوریتم های جستجو 4 معیار تعریف می شود :


\vspace{10pt}


\noindent
کامل بودن 
\lr{(Completeness)} :

\noindent
اگر مسئله ای دارای جواب باشد و استراتژی جستجوی مورد نظر همیشه بتواند آنرا پیدا کند به آن استراتژی کامل می گوییم .


\vspace{10pt}


\noindent
بهینه بودن 
\lr{(Optimality)} :

\noindent
اگر در مسئله ای بیش از یک مسیر به جواب وجود داشته باشد و یا دو جواب متفاوت وجود داشته باشد . الگوریتمی بهتر است که هزینه ی مسیری کمتری داشته باشد .



\vspace{10pt}


\noindent
پیچیدگی زمانی 
\lr{(Time Complexity)} :

\noindent
تعداد گره های تولید شده تا رسیدن به جواب ، پیچیدگی زمانی در نظر گرفته می شود .



\vspace{10pt}


\noindent
پیچیدگی مکانی 
\lr{(Space Complexity)} :

\noindent
میزان حافظه ی مورد نیاز برای رسیدن به جواب .


\vspace{30pt}


\noindent
به صورت کلی 2 نوع استراتژی جستجو وجود دارد :

\begin{itemize}
	\item جستجوی ناآگاهانه
	\item جستجوی آگاهانه
\end{itemize}






\end{document}
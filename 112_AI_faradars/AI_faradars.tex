\documentclass[12pt]{article}

\usepackage{tabularx}
\usepackage[table]{xcolor}
\usepackage{multirow}

\usepackage{amsmath, amssymb}
\usepackage{mathtools}

\usepackage{listings}

\usepackage[margin=1.1in,footskip=.25in]{geometry}

\usepackage[most]{tcolorbox}

\tcbset{
    frame code={}
    center title,
    left=10pt,
    right=10pt,
    top=10pt,
    bottom=10pt,
    colback=gray!5,
    colframe=gray,
    width=\dimexpr\textwidth\relax,
    enlarge left by=0mm,
    boxsep=5pt,
    arc=0pt,outer arc=0pt,
}


\usepackage{xepersian}
\settextfont[Scale=1]{Vazir}
\renewcommand{\baselinestretch}{1.3} 

\begin{document}


\tableofcontents






\section{عامل}

\noindent
عامل می تواند شامل ربات انسان نما ، چراغ راهرو ، ترموستات و . . . باشد .
\newline

\noindent
تعریف عامل : عامل موجودی است که به طور متناوب بر اساس رشته دریافت هایی که از حسگر ها (سنسور ها) می گیرد ، دنباله ای از اعمال را در محیط انجام می دهد .




\section{اجزای عامل}


\noindent
سنسور 
\lr{(Sensor)}
 : وظیفه دریافت مشخصه هایی از محیط 
 \lr{(Percept)}



\noindent
عملگر 
\lr{(Actuator)}
: وظیفه ی انجام اعمال بر روی محیط
\lr{(Action)}






\section{وظیفه ی عامل}

\noindent
عامل وظیفه دارد رشته دریافت های ورودی را به دنباله ای از اعمال نگاشت نماید .

\noindent
بنابراین می توان گفت عامل مانند یک تابع است .

\noindent
عامل می تواند اعمال خود در محیط را درک کند ، اما تاثیر آنها بر روی محیط همیشه قابل پیش بینی نیست .






\section{عامل و محیط}


\begin{enumerate}
	\item بنابراین هر محیط دارای مجموعه ای از حالت ها می باشد
	\item محیط در هر لحظه در یکی از این حالت ها می باشد
	\item عمل عامل در محیط باعث تغییر حالت محیط می باشد 
\end{enumerate}





\section{مفهوم عقلانیت}


\noindent
برای دستیابی به عقلانیت چهار فاکتور زیر باید به درستی تعریف شود :

\begin{enumerate}
	\item معیار کارایی
	\item دانش اولیه محیطی
	\item اعمال
	\item رشته دریافت ها
\end{enumerate}


\noindent
تعریف عامل هوشمند : عاملی است که بر اساس رشته دریافت ها و دانش اولیه محیطی ، عملی را انتخاب کند ، که به واسطه ی اجرای آن عمل ، معیار کارایی مورد انتظار حداکثر گردد .


\vspace{30pt}

\noindent
عقلانیت با دانای کل
\lr{(omniscience)}
متفاوت است .


\noindent
دانای کل نتیجه هر عمل خود را از قبل می داند .




\vspace{30pt}

\noindent
عقلانیت با کمال گرایی 
\lr{(perfection)}
متفاوت است .

\noindent
کمال گرا هر عمل را به بهترین شکل ممکن انجام می دهد .




\section{انواع محیط}

برای طراحی یک عامل هوشمند باید مشخصات دقیق مسئله تعیین شود .

\noindent
مشخصات مسئله اصطلاحاً با 
\lr{PEAS}
بیان می شود .


\begin{enumerate}
	\item \lr{Performance}
	\item \lr{Environment}
	\item \lr{Actuators}
	\item \lr{Sensors}
\end{enumerate}




\section{انواع محیط}


\begin{enumerate}
	\item کاملاً قابل مشاهده
	\lr{(Fully Observable)} :
	تمام جنبه های محیط که بر روی انتخاب عمل تاثیر گذار است ، توسط سنسورها قابل دریافت باشد .
	\item قطعی
	\lr{(Deterministic)} :
	حالت بعدی مساله از روی وضعیت فعلی قابل شناسایی باشد .
	\item اپیزودیک
	\lr{(Episodic)} :
	مساله را بتوان به بخش های کوچکتر اتمیک ( غیر قابل تجزیه ) تقسیم نمود . سنسور هر بخش را جداگانه دریافت نموده و عمل مورد نظر را بر روی آن انجام دهد . عمل مورد نظر به اعمال قبلی و بعدی ارتباط ندارد .
\end{enumerate}


$$
----------
$$

\vspace{10pt}

\begin{enumerate}
	\item ایستا
	\lr{(Static)} :
	محیطی که در حین تصمیم گیری عامل ، امکان تغییر نداشته باشد .
	\item  گسسته
	\lr{(Discrete)} :
	محیطی که تعداد اعمال قابل انجام بر روی آن شمارا باشد ( غیر بی نهایت )
	\item چند عامله
	\lr{(Multi Agent)} :
	محیطی که شامل عامل های دیگر باشد که درصدد حداکثر نمودن معیار کارایی خودشان هستند و بر روی کارایی عامل ممکن است تاثیر گذار باشند .
\end{enumerate}


$$
----------
$$

\vspace{10pt}


\begin{itemize}
	\item ساده ترین محیط ، محیطی است که کاملاً قابل مشاهده ، قطعی ، اپیزودیک ، ایستا ، گسسته و تک عامله باشد
	\item اغلب محیط های مسائل واقعی محیط های بخشی قابل مشاهده ، غیر قطعی ، ترتیبی ، پویا ، پیوسته و چند عامله هستند .
\end{itemize}






\section{انواع عامل}

\noindent
یک عامل چگونه کار می کند ؟

\begin{center}
عامل = برنامه + سخت افزار
\end{center}




\noindent
عامل ها همگی دارای یک ساختار مشترک هستند .

\begin{itemize}
	\item ورودی : دریافت فعلی
	\item خروجی : عمل مناسب
	\item برنامه : پردازش ورودی برای تعیین خروجی
\end{itemize}




\noindent
تفاوت عامل ها در نحوه پردازش است .



\vspace{20pt}


چهار نوع عامل عبارتند از :

\begin{itemize}
	\item عامل واکنشی ساده
	\lr{(Simple Reflex)}
	\item عامل واکنشی مبتنی بر مدل
	\lr{(Model Based Reflex)}
	\item عامل مبتنی بر هدف
	\lr{(Goal Based)}
	\item عامل مبتنی بر سودمندی
	\lr{(Utility Based)}
\end{itemize}


\noindent
همه ی عامل ها می توانند به قاعده یادگیری مجهز شوند .






\section{انواع مساله}


\begin{itemize}
	\item قطعی و کاملاً قابل مشاهده : مسایل تک حالته
	\lr{(Single State)}
	\item قطعی و بخشی قابل مشاهده : مسایل غیر قابل دریافت
	\lr{(Sensorless/Conformant)}
	\item غیر قطعی و بخشی قابل مشاهده : مسایل احتمالی
	\lr{(Contingency)}
	\item فضای حالت ناشناخته : مسایل اکتشافی یا بر خط
	\lr{(Exploration/Online)}
\end{itemize}








\section{فرموله سازی مساله}


یک مساله با موارد زیر تعریف می شود .


\begin{itemize}
	\item حالت شروع 
	\lr{(initial state)}
	\item تابع جانشینی
	\lr{(successor function)}
	\item آزمون هدف
	\lr{(goal test)}
	\item هزینه ی مسیر
	\lr{(path cost)}
	\item راه حل
	\lr{(solution)}
	\item راه حل بهینه
	\lr{(optimal solution)}
\end{itemize}







\section{دنیای جاروبرقی}



\begin{itemize}
	\item حالات : 8 حالت مختلف وجود دارد 
	\item حالت شروع : هر یک از حالات
	\item اعمال : 
	\{ چپ ، راست ، مکش ، هیچ کار \}
	\item آزمون هدف : حالت 7 یا 8
	\item هزینه ی مسیر : تعداد اعمال انجام شده تا رسیدن به مسیر
\end{itemize}




\section{پازل اعداد}


\begin{itemize}
	\item حالات : جایگشت های مختلف
	\item حالت شروع : هر یک از حالات
	\item اعمال :
	\{ چپ ، راست ، بالا ، پایین \}
	\item آزمون هدف : حالت هدف
	\item هزینه ی مسیر : تعداد اعمال انجام شده
\end{itemize}







\section{8 وزیر}


\begin{itemize}
	\item حالات : جایگشت های مختلف چینش
	\item حالت شروع : صفحه ی خالی
	\item اعمال :
	\{ اضافه نمودن وزیر در جای مناسب \}
	\item آزمون هدف : 8 وزیر بر روی صفحه ی شطرنج
	\item هزینه ی مسیر : زمان اجرا
\end{itemize}




\section{بازوی مکانیکی قطعه ساز}


\begin{itemize}
	\item حالات : جایگشت های مختلف مفاصل
	\item حالت شروع : هر مقدار قرارگیری مفاصل
	\item اعمال : جابجایی مفاصل
	\item آزمون هدف : ساخت کامل قطعه
	\item هزینه ی مسیر : زمان اجرا
\end{itemize}









\end{document}
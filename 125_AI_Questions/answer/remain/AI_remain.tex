\documentclass[12pt]{article}

\usepackage{tabularx}
\usepackage[table]{xcolor}
\usepackage{multirow}


\usepackage[margin=1.1in,footskip=.25in]{geometry}



\usepackage[most]{tcolorbox}

\tcbset{
    frame code={}
    center title,
    left=10pt,
    right=10pt,
    top=10pt,
    bottom=10pt,
    colback=gray!5,
    colframe=gray,
    width=\dimexpr\textwidth\relax,
    enlarge left by=0mm,
    boxsep=5pt,
    arc=0pt,outer arc=0pt,
}


\usepackage{amsmath, amssymb}
\usepackage{mathtools}



\usepackage{xepersian}
\settextfont[Scale=1]{Vazir}

\renewcommand{\baselinestretch}{1.3} 


\begin{document}


\vspace{20pt}
\noindent
ادراکات چیست؟


\begin{tcolorbox}
به ورودی های دریافت شده توسط عامل ادراکات گفته می شود .
\end{tcolorbox}



\vspace{20pt}
\noindent
تابع عامل چیست؟

\begin{tcolorbox}
تابع عامل ادراکات را از محیط دریافت کرده و به خروجی تبدیل می کند که با استفاده از عملگرها بر محیط اعمال می شود .
\end{tcolorbox}


\vspace{20pt}
\noindent
تفاوت الگوریتم
 \lr{BFS}
 و 
 \lr{UCS}
 چیست؟


\begin{tcolorbox}
الگوریتم 
 \lr{BFS} : 
 
 در این الگوریتم ابتدا نود ریشه بسط داده می شود ، سپس همه ی نود های بسط داده شده توسط ریشه بسط داده می شوند و سپس فرزندان آنها سطر به سطر بسط داده می شود .
 
 در این الگوریتم ابتدا تمامی نود ها در عمق 
 \lr{d}
 و سپس همه ی نودها  در عمق
 \lr{d+1}
 بسط داده می شوند .
 
 
 \vspace{20pt}
 
 الگوریتم
  \lr{UCS} : 
  
  در این الگوریتم نود با کمترین هزینه بسط داده می شود .

\end{tcolorbox}



\vspace{20pt}
\noindent
خصوصیات الگوریتمهای جستجوی محلی چیست؟



\begin{tcolorbox}
الگوریتم های جستجوی محلی با استفاده از حالت فعلی عمل می کنند .

\vspace{20pt}

مسیر هایی که در جستجو برای رسیدن به هدف در پیش گرفته می شوند . در حافظه نگهداری نمی شوند .

\vspace{20pt}

در این الگوریتم ها حافظه ی زیادی نیاز ندارد .
\end{tcolorbox}


\vspace{20pt}
\noindent
در چه شرایطی استفاده از الگوریتمهای جستجوی محلی بهتر است.


\begin{tcolorbox}
در مسائلی که مسیر رسیدن به هدف مهم نباشد و فقط حالت نهایی مهم باشد ، استفاده از الگوریتم های جستجوی محلی بهتر است .
\end{tcolorbox}



\vspace{20pt}
\noindent
در مورد الگوریتمهای جستجوی 
\lr{(BFS- DFS- DLS- IDS)}
کدام یک کامل هستند و کدام یک کامل نیستند. دلیل خود را به طور واضح بیان نمایید.




\vspace{20pt}
\noindent
سه روش از روشهای رسمی تولید 
\lr{heuristic}
 را بنویسید. برای هر مورد یک مثال کوچک بزنید.





\newpage
\vspace{20pt}
\noindent
منظور از الگوریتم جستجوی ناآگاهانه چیست؟


\begin{tcolorbox}
الگوریتم جستجوی ناآگاهانه به غیر از تعریف مسئله ، هیچ گونه اطلاعات دیگری در رابطه با مسئله در اختیار ندارد .
\end{tcolorbox}


\vspace{20pt}
\noindent
نکته کلیدی الگوریتمهای جستجوی آگاهانه چیست؟


\begin{tcolorbox}
الگوریتم های جستجوی آگاهانه علاوه بر تعریف مسئله ، اطلاعات بیشتری در رابطه با مسئله دارد .


نگرش کلی جستجو های آگاهانه انتخاب اولین-بهترین 
\lr{(Best-First)}
بر اساس تابع ارزیابی است .
\end{tcolorbox}


\begin{tcolorbox}
نکته ی کلیدی الگوریتم های جستجو ی آگاهانه استفاده از تابع هیوریستیک می باشد .
\end{tcolorbox}


\end{document}
\documentclass[12pt]{article}

\usepackage[siunitx, RPvoltages]{circuitikz}
\usepackage{tikz}
\usetikzlibrary{arrows}
\usetikzlibrary{shapes}
\usepackage[margin=1in,footskip=.25in]{geometry}
\usepackage{amsmath, amssymb}

\begin{document}

\section{Learning KVL \& KCL}

\begin{center}
\begin{circuitikz}[american]
	\draw (0,0) -- (0,3);
	\draw (0,3) to[R=$R_1$] (7,3);
	\draw (7,3) to[short, f>=$i_1$] (7,0);
	\draw (7,0) to[R=$R_2$,  f>=$i_2$] (2,0);
	\draw (2,0) to[battery1,v=$\varepsilon_{1}$] (0,0);
	\draw (7,0) to[short, f>=$i_3$] (7,-3);
	\draw (7,-3) to[R=$R_3$] (2,-3);
	\draw (0,-3) to[battery1,v=$\varepsilon_{2}$] (2,-3);
	\draw (0,-3) -- (0,0);
	\draw[thin, <-, >=triangle 45] (3,1.5) node{$s_1$}  ++(-40:1) arc (-40:170:1);
	\draw[thin, <-, >=triangle 45] (3,-2) node{$s_2$}  ++(-40:1) arc (-40:170:1);
\end{circuitikz}
\end{center}

\noindent
Assume an electric network consisting of two voltage sources and three resistors.
\noindent
According to the first law we have

{
\Large
$$
i_{1} = i_{2} + i_{3} 
$$
}

\noindent
The second law applied to the closed circuit $s_{1}$ gives

{
\Large
$$
- R_{2} i_{2}  + \varepsilon_{1} + ( - R_{1} i _{1} ) = 0   
$$
}

\noindent
The second law applied to the closed circuit $s_{2}$ gives

{
\Large
$$
- R_{3} i_{3} - \varepsilon_{2} - \varepsilon_{1} + R_{2} i_{2} = 0 
$$
}

\noindent
Thus we get a system of linear equations in $i_{1}, i_{2}, i_{3}$ :

{
\Large
\begin{align*}
\begin{cases}
i_{2} = i_{1} + i_{3}  \\
- R_{2} i_{2}  + \varepsilon_{1} + ( - R_{1} i _{1} ) = 0  \\
- R_{3} i_{3} - \varepsilon_{2} - \varepsilon_{1} + R_{2} i_{2} = 0 
\end{cases}
\end{align*}
}
\end{document}
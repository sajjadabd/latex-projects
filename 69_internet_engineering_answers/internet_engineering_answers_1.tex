\documentclass{article}

\usepackage{tabularx}
\usepackage[table]{xcolor}
\usepackage{multirow}

\usepackage{xcolor}
\usepackage{color}

\usepackage{amsmath, amssymb}
\usepackage{mathtools}
\usepackage[margin=1.1in,footskip=.25in]{geometry}
\usepackage{listings}

\usepackage{xepersian}
\settextfont[Scale=1]{Vazir}

\renewcommand{\baselinestretch}{1.3} 
\definecolor{codepurple}{rgb}{0.58,0.8,0.82}
\definecolor{backcolour}{rgb}{0.95,0.95,0.92}

\begin{document}

\section{انواع کلاس های IP}

\begin{latin}
\begin{itemize}
	\item Class A :
\end{itemize}
\end{latin}
\begin{align*}
 0.  0.  0.  0 = &00000000.00000000.00000000.00000000 \\
127.255.255.255 = &01111111.11111111.11111111.11111111 \\
                  &0nnnnnnn.HHHHHHHH.HHHHHHHH.HHHHHHHH
\end{align*}


\begin{latin}
\begin{itemize}
	\item Class B :
\end{itemize}
\end{latin}
\begin{align*}
128.  0.  0.  0 = &10000000.00000000.00000000.00000000 \\
191.255.255.255 = &10111111.11111111.11111111.11111111 \\
                  &10nnnnnn.nnnnnnnn.HHHHHHHH.HHHHHHHH
\end{align*}


\begin{latin}
\begin{itemize}
	\item Class C :
\end{itemize}
\end{latin}
\begin{align*}
192.  0.  0.  0 = &11000000.00000000.00000000.00000000 \\
223.255.255.255 = &11011111.11111111.11111111.11111111 \\
                  &110nnnnn.nnnnnnnn.nnnnnnnn.HHHHHHHH
\end{align*}



\begin{latin}
\begin{itemize}
	\item Class D :
\end{itemize}
\end{latin}
\begin{align*}
224.  0.  0.  0 = &11100000.00000000.00000000.00000000 \\
239.255.255.255 = &11101111.11111111.11111111.11111111 \\
                  &1110XXXX.XXXXXXXX.XXXXXXXX.XXXXXXXX
\end{align*}





\begin{latin}
\begin{itemize}
	\item Class E :
\end{itemize}
\end{latin}
\begin{align*}
240.  0.  0.  0 = &11110000.00000000.00000000.00000000 \\
255.255.255.255 = &11111111.11111111.11111111.11111111 \\
                  &1111XXXX.XXXXXXXX.XXXXXXXX.XXXXXXXX
\end{align*}


\section{سوالات محاسباتی مهندسی اینترنت}

\subsection{نوع کلاس IP آدرس های زیر را به دست آورید ؟}

\begin{align*}
23.1.3.5 &\to 00010111.00000001.00000011.00000101 \Rightarrow class \:\: A \\
198.34.54.23 &\to 11000110.00100010.00110110.00010111 \Rightarrow class \:\: C \\
233.12.3.4 &\to 11101001.00001100.00000011.00000100  \Rightarrow class \:\: D \\
45.2.3.67 &\to 00101101.00000010.00000011.01000011  \Rightarrow class \:\: A \\
178.11.23.5 &\to 10110010.00001011.00010111.00000101 \Rightarrow class \:\: B \\
254.12.34.5 &\to 11111110.00001100.00100010.00000101 \Rightarrow class \:\: E \\
\end{align*}





\section{یک شبکه کلاس C با آدرس 
$
194.34.56.0
$
داده شده است، چند میزبان برای  این شبکه وجود دارد ؟ 
}

\begin{align*}
194.34.56.0 \to \underbrace{11000010.00100010.00111000}_{Network}.\underbrace{00000000}_{Host}
\end{align*}

{
\LARGE
$$
2^{8} - 2
$$
}



\section{یک شبکه کلاس B با آدرس 
$
166.23.0.0
$
داده شده است، چند میزبان برای  این شبکه وجود دارد ؟ 
}


\begin{align*}
166.23.0.0 \to \underbrace{10100110.00010111}_{Network}.\underbrace{00000000.00000000}_{Host}
\end{align*}

{
\LARGE
$$
2^{16} - 2
$$
}


\section{آدرس کلاس
A
با چه عدد دودویی شروع می شود، و محدوده ی آدرس این کلاس را مشخص کنید ؟}

با عدد $0$ شروع می شود .


\begin{latin}
\begin{itemize}
	\item Class A :
\end{itemize}
\end{latin}
\begin{align*}
 0.  0.  0.  0 = &00000000.00000000.00000000.00000000 \\
127.255.255.255 = &01111111.11111111.11111111.11111111 \\
                  &0nnnnnnn.HHHHHHHH.HHHHHHHH.HHHHHHHH
\end{align*}




\section{آدرس کلاس
B
با چه عدد دودویی شروع می شود، و محدوده ی آدرس این کلاس را مشخص کنید ؟}


با عدد $10$ شروع می شود .

\begin{latin}
\begin{itemize}
	\item Class B :
\end{itemize}
\end{latin}
\begin{align*}
128.  0.  0.  0 = &10000000.00000000.00000000.00000000 \\
191.255.255.255 = &10111111.11111111.11111111.11111111 \\
                  &10nnnnnn.nnnnnnnn.HHHHHHHH.HHHHHHHH
\end{align*}


\newpage


\section{محدوده ی شبکه و میزبان را در کلاس های آدرس
A و B و C مشخص کنید ؟}


\begin{latin}
\begin{itemize}
	\item Class A :
\end{itemize}
\end{latin}
\begin{align*}
 0.  0.  0.  0 = &\underbrace{00000000}_{Network}.\underbrace{00000000.00000000.00000000}_{Host} \\
127.255.255.255 = &\underbrace{01111111}_{Network}.\underbrace{11111111.11111111.11111111}_{Host} \\
                  &0nnnnnnn.HHHHHHHH.HHHHHHHH.HHHHHHHH
\end{align*}


\begin{latin}
\begin{itemize}
	\item Class B :
\end{itemize}
\end{latin}
\begin{align*}
128.  0.  0.  0 = &\underbrace{10000000.00000000}_{Network}.\underbrace{00000000.00000000}_{Host} \\
191.255.255.255 = &\underbrace{10111111.11111111}_{Network}.\underbrace{11111111.11111111}_{Host} \\
                  &10nnnnnn.nnnnnnnn.HHHHHHHH.HHHHHHHH
\end{align*}


\begin{latin}
\begin{itemize}
	\item Class C :
\end{itemize}
\end{latin}
\begin{align*}
192.  0.  0.  0 = &\underbrace{11000000.00000000.00000000}_{Network}.\underbrace{00000000}_{Host} \\
223.255.255.255 = &\underbrace{11011111.11111111.11111111}_{Network}.\underbrace{11111111}_{Host} \\
                  &110nnnnn.nnnnnnnn.nnnnnnnn.HHHHHHHH
\end{align*}




\newpage

\section{مشخص کنید که آدرس 
$
192.168.1.18/24
$
جزء کدام دسته کلاس آدرس می باشد و آدرس خود شبکه ، اولین میزبان ، آخرین میزبان و آدرس Broadcast را در این شبکه مشخص کنید ؟
}


$$
192.168.1.18 \to 11000000.10101000.00000001.00010010 \Rightarrow class \:\: C
$$

$$
\underbrace{192.168.1}_{Network}.\underbrace{18}_{Host}
$$


\begin{align*}
Subnet &= 192.168.1.00000000 \\
1st \:\: Host &= 192.168.1.00000001 \\
Last \:\: Host &= 192.168.1.11111110 \\
Broadcast &= 192.168.1.11111111 \\
\end{align*}





\section{مشخص کنید که آدرس 
$
172.16.35.123/20
$
جزء کدام دسته کلاس آدرس می باشد و آدرس خود شبکه ، اولین میزبان ، آخرین میزبان و آدرس Broadcast را در این شبکه مشخص کنید ؟
}


$$
172.16.35.123 \to 10101100.00010000.00100011.01111011 \Rightarrow class \:\: B
$$


$$
\underbrace{172.16}_{Network}.\underbrace{35.123}_{Host}
$$


\begin{latin}
\begin{center}
  \begin{tabular}{ r  r | l  }
    Subnet $\to$ & 172.16.0010 & 0000.00000000 \\
    1st Host $\to$ & 172.16.0010 & 0000.00000001 \\
    Last Host $\to$ & 172.16.0010 & 1111.11111110 \\
    Broadcast $\to$ & 172.16.0010 & 1111.11111111 \\
  \end{tabular}
\end{center}
\end{latin}


\begin{latin}
\begin{center}
  \begin{tabular}{ r  l   }
    Subnet $\to$ & 172.16.32.0  \\
    1st Host $\to$ & 172.16.32.1  \\
    Last Host $\to$ & 172.16.47.254 \\
    Broadcast $\to$ & 172.16.47.255 \\
  \end{tabular}
\end{center}
\end{latin}




\section{مشخص کنید که آدرس 
$
172.16.129.1/17
$
جزء کدام دسته کلاس آدرس می باشد و آدرس خود شبکه ، اولین میزبان ، آخرین میزبان و آدرس Broadcast را در این شبکه مشخص کنید ؟
}


$$
172.16.129.1 \to 10101100.00010000.10000001.00000001 \Rightarrow class \:\: B
$$


$$
\underbrace{172.16}_{Network}.\underbrace{129.1}_{Host}
$$



\begin{latin}
\begin{center}
  \begin{tabular}{ r  r | l  }
    Subnet $\to$ & 172.16.1 & 0000000.00000000 \\
    1st Host $\to$ & 172.16.1 & 0000000.00000001 \\
    Last Host $\to$ & 172.16.1 & 1111111.11111110 \\
    Broadcast $\to$ & 172.16.1 & 1111111.11111111 \\
  \end{tabular}
\end{center}
\end{latin}



\begin{latin}
\begin{center}
  \begin{tabular}{ r  l   }
    Subnet $\to$ & 172.16.128.0  \\
    1st Host $\to$ & 172.16.128.1  \\
    Last Host $\to$ & 172.16.255.254 \\
    Broadcast $\to$ & 172.16.255.255 \\
  \end{tabular}
\end{center}
\end{latin}


\section{$8-bit \:\: Binary \:\: Trick$}

\begin{latin}
\begin{center}
  \bgroup
  \def\arraystretch{1.5}%
  \begin{tabular}{ c c c c c c c c  }
    128 & 64 & 32 & 16 & 8 & 4 & 2 & 1 \\
    \hline
    128 & 192 & 224 & 240 & 248 & 252 & 254 & 255  \\
  \end{tabular}
  \egroup
\end{center}
\end{latin}


\newpage

\section{خلاصه ی کلاس های IP به صورت جدول}

\begin{latin}
\begin{center}
  \rowcolors{1}{codepurple}{backcolour}
  \bgroup
  \def\arraystretch{1.5}%
  \begin{tabular}{ c | c | c | c | c | c   }
    Class & Starting Bits & Size of network ( bit) & Size of Host (bit) & Number of networks & Hosts per Network \\
    \hline
    Class A & 0 & 8 & 24 & 128 $(2^{7})$ & 16,777,216 $(2^{24})$  \\ \hline
   & \multicolumn{2}{c |}{Total addresses in class}  & Start address & End address  & \\ \hline
   & \multicolumn{2}{c |}{2,147,483,648 $(2^{31})$} & 0.0.0.0 & 	127.255.255.255 &  \\ \hline
  \end{tabular}
  \egroup
\end{center}
\end{latin}




\begin{latin}
\begin{center}
  \rowcolors{1}{codepurple}{backcolour}
  \bgroup
  \def\arraystretch{1.5}%
  \begin{tabular}{ c | c | c | c | c | c   }
    Class & Starting Bits & Size of network ( bit) & Size of Host (bit) & Number of networks & Hosts per Network \\
    \hline
    Class B & 10 & 16 & 16 & 16,384 $(2^{14})$ & 65,536 $(2^{16})$  \\ \hline
   & \multicolumn{2}{c |}{Total addresses in class}  & Start address & End address  & \\ \hline
   & \multicolumn{2}{c |}{1,073,741,824 $(2^{30})$} & 128.0.0.0 & 	191.255.255.255 &  \\ \hline
  \end{tabular}
  \egroup
\end{center}
\end{latin}



\begin{latin}
\begin{center}
  \rowcolors{1}{codepurple}{backcolour}
  \bgroup
  \def\arraystretch{1.5}%
  \begin{tabular}{ c | c | c | c | c | c   }
    Class & Starting Bits & Size of network ( bit) & Size of Host (bit) & Number of networks & Hosts per Network \\
    \hline
    Class C & 110 & 24 & 8 & 2,097,152 $(2^{21})$ & 256 $(2^{8})$  \\ \hline
   & \multicolumn{2}{c |}{Total addresses in class}  & Start address & End address  & \\ \hline
   & \multicolumn{2}{c |}{536,870,912 $(2^{29})$} & 192.0.0.0 & 	223.255.255.255 &  \\ \hline
  \end{tabular}
  \egroup
\end{center}
\end{latin}


\end{document}
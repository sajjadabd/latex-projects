\documentclass[12pt]{article}

\renewcommand{\baselinestretch}{1.3} 

\usepackage{amsmath, amssymb}
\usepackage{mathtools}

\usepackage{listings}
\usepackage{xcolor}
\usepackage{color}

\definecolor{dkgreen}{rgb}{0,0.6,0}
\definecolor{gray}{rgb}{0.5,0.5,0.5}
\definecolor{mauve}{rgb}{0.58,0,0.82}
 
\definecolor{codegreen}{rgb}{0,0.6,0}
\definecolor{codegray}{rgb}{0.5,0.5,0.5}
\definecolor{codepurple}{rgb}{0.58,0,0.82}
\definecolor{backcolour}{rgb}{0.95,0.95,0.92}
 
\lstdefinestyle{mystyle}{
    backgroundcolor=\color{backcolour},   
    commentstyle=\color{codegreen},
    keywordstyle=\color{magenta},
    numberstyle=\tiny\color{codegray},
    stringstyle=\color{codepurple},
    basicstyle=\ttfamily\small,
    breakatwhitespace=false,         
    breaklines=true,                 
    captionpos=b,                    
    keepspaces=true,                 
    numbers=left,                    
    numbersep=5pt,                  
    showspaces=false,                
    showstringspaces=false,
    showtabs=false,                  
    tabsize=3
}


\lstdefinestyle{javaStyle}{
  frame=tb,
  language=Java,
  aboveskip=3mm,
  belowskip=3mm,
  showstringspaces=false,
  columns=flexible,
  basicstyle={\small\ttfamily},
  numbers=none,
  numberstyle=\tiny\color{gray},
  keywordstyle=\color{blue},
  commentstyle=\color{dkgreen},
  stringstyle=\color{mauve},
  breaklines=true,
  breakatwhitespace=true,
  tabsize=3
}

\lstset{style=mystyle}

\usepackage[margin=1.1in,footskip=.25in]{geometry}




\begin{document}


\section{Types Of Parameter Passing}

\subsection{call-by-value}

the parameter refer to the new copied values in this new memory . at the function conclusion this memory may simply be deallocated .



\subsection{call-by-reference}

in the call-by-reference evaluation strategy , when a function is called it recieves the memory locations of the arguments passed to it .

in this case , no extra memory need to be allocated .


\section{Parameter Passing in Languages}


\subsection{C}

predefined data type like : int , char , $\dots$ are passed with call-by-value in C .

call-by-reference simulated in C by employing the concept of a pointer , 

This is done with the operator * .

if $a$ is a pointer , then $*a$ is the actual memory location pointed to .


\subsection{C++}

C++ allows one to declare reference types, for example 
\lstinline{int& b = a;}
defines an integer reference b, which simply acts as an alias for the integer a .


Additionally, C++ offers a $const$ keyword if one would like the performance benefits of call-by-reference but still wants assurance that the arguments are never modified as in call-by-value .


\subsection{Java}

Java is almost like C++ . 

but does not support the pointer manipulation and reference types offered in C++ .

in Java , variables are passed by value .

Java , passes Objects by reference .


\subsection{Fortran}

the default method of parameter passing is call-by-reference .

\subsection{ALGOL}

supports both call-by-value and call-by-reference .


\section{Parameter Passing Modes in C}

\begin{itemize}
	\item Call by value parameter passing only
	\item Objects can be modified in a function by passing pointers to the object to the function
	\item Arrays and pointers are exchangeble in C: an array is
automatically passed as a pointer to the array
\end{itemize}



\section{Parameter Passing Modes in Fortran}

\begin{itemize}
	\item Call by reference parameter passing only
\end{itemize}



\section{Parameter Passing Modes in Pascal}

\begin{itemize}
	\item Call by value and call by reference parameter passing
	\item Call by value is similar to C
	\item Call by reference is indicated by var parameters
\begin{lstlisting}[language=Pascal]
swap(var a:integer, var b:integer)
var t;
begin
t := a; a := b; b := t
end
\end{lstlisting}
\end{itemize}





\section{Parameter Passing Modes in C++}

\begin{itemize}
	\item Call by value and call by reference parameter passing
	\item Call by value is similar to C
	\item Call by reference is indicated by using \& for formal parameters
\begin{lstlisting}[language=C]
swap(int &a, int &b)
{ 
	int t = a; 
	a = b; 
	b = t; 
}
\end{lstlisting}

	\item Large objects should be passed by reference instead of by value to increase efficiency
	\item Arrays are automatically passed by reference (like in C)
	\item To avoid objects to be modified when passed by reference, const parameters can be used
\begin{lstlisting}[language=C]
store_record_in_file(const huge_record &r)
{ 
	... 
}
\end{lstlisting}

\end{itemize}






\section{Parameter Passing Modes in Java}

\begin{itemize}
	\item Call by value and call by reference/sharing parameter passing
	\item Java adopts both value and reference models of variables
	\begin{itemize}
		\item Variables of built-in types are passed by value
		\item Class instances are passed by sharing
	\end{itemize}
\end{itemize}





\section{Parameter Passing Modes in Ada}

\begin{itemize}
	\item Call by value, call by result, and call by value/result parameter passing
	\item Indicated by Ada’s in(by value), out(by result), and in out(by value/result) modes for formal parameters
\begin{lstlisting}[language=Ada]
procedure shift(a:out integer, b:in out integer,
c:in integer) is
begin
	a := b; b := c;
end shift;
\end{lstlisting}
	\item in mode parameters can be read but not written in the subroutine
	\item out mode parameters can be written but not read in the subroutine
	\item in out mode parameters can be read and written in the subroutine
\end{itemize}




\section{Parameter Passing Modes in Python}

Python uses a mechanism, which is known as "Call-by-Object", sometimes also called "Call by Object Reference" or "Call by Sharing".







\section{C++ Variables}


\subsection{Declaring (Creating) Variables}

To create a variable, you must specify the type and assign it a value:

\lstset{style=mystyle}
\begin{lstlisting}[language=C++, caption=C++ example]
type variable = value;
\end{lstlisting}


\newpage

\section{Python Variables}

\subsection{Creating Variables}

Python has no command for declaring a variable.

A variable is created the moment you first assign a value to it.

Variables do not need to be declared with any particular type and can even change type after they have been set.

\lstset{style=mystyle}
\begin{lstlisting}[language=Python, caption=Python example]
x = 5
y = "John"
print(x)
print(y)
\end{lstlisting}


\lstset{style=mystyle}
\begin{lstlisting}[language=Python, caption=Python example]
x = 4 # x is of type int
x = "Sally" # x is now of type str
print(x)
\end{lstlisting}



\subsection{Assign Value to Multiple Variables}

Python allows you to assign values to multiple variables in one line:


\lstset{style=mystyle}
\begin{lstlisting}[language=Python, caption=Python example]
x, y, z = "Orange", "Banana", "Cherry"
print(x)
print(y)
print(z)
\end{lstlisting}



\newpage

\section{Ada Variables}


\subsection{Declaring a Variable}


To declare a variable, use the following formula:

\lstset{style=mystyle}
\begin{lstlisting}[language=Ada, caption=Ada example]
VariableName : DataType;
\end{lstlisting}


declare various variables :

\lstset{style=mystyle}
\begin{lstlisting}[language=Ada, caption=Ada example]
VariableName1, VariableName2 : DataType1;
\end{lstlisting}



\subsection{Initializing a Variable}


To initialize a variable :


\lstset{style=mystyle}
\begin{lstlisting}[language=Ada, caption=Ada example]
VariableName : DataType := Value;
\end{lstlisting}



assigning a value after declaring it :


\lstset{style=mystyle}
\begin{lstlisting}[language=Ada, caption=Ada example]
VariableName : DataType
begin
    VariableName:= Value;
end
\end{lstlisting}


\newpage

\section{Pascal}


\subsection{Variable Declaration in Pascal}

All variable declarations are followed by the var keyword. A declaration specifies a list of variables, followed by a colon (:) and the type. Syntax of variable declaration is -


\lstset{style=mystyle}
\begin{lstlisting}[language=Pascal, caption=Pascal example]
var
variable_list : type;
\end{lstlisting}


valid Pascal data type including :
 character, integer,  real, boolean,  any user-defined data type .


Some valid variable declarations are shown here - 


\lstset{style=mystyle}
\begin{lstlisting}[language=Pascal, caption=Pascal example]
var
age, weekdays : integer;
taxrate, net_income: real;
choice, isready: boolean;
initials, grade: char;
name, surname : string;
\end{lstlisting}


\newpage

\subsection{Variable Initialization in Pascal}


Variables are assigned a value with a colon and the equal sign, followed by a constant expression. The general form of assigning a value is -


\lstset{style=mystyle}
\begin{lstlisting}[language=Pascal, caption=Pascal example]
variable_name := value;
\end{lstlisting}


Variables can be initialized (assigned an initial value) in their declaration :



\lstset{style=mystyle}
\begin{lstlisting}[language=Pascal, caption=Pascal example]
var
variable_name : type = value;
\end{lstlisting}


Some examples are -



\lstset{style=mystyle}
\begin{lstlisting}[language=Pascal, caption=Pascal example]
age: integer = 15;
taxrate: real = 0.5;
grade: char = 'A';
name: string = 'John Smith';
\end{lstlisting}



\newpage

\section{Lisp}


\subsection{Global Variables}


Global variables are generally declared using the defvar construct.

\lstset{style=mystyle}
\begin{lstlisting}[language=Lisp, caption=Lisp example]
(defvar x 234)
(write x)
\end{lstlisting}


you can directly specify a value for a symbol with the setq construct :



\lstset{style=mystyle}
\begin{lstlisting}[language=Lisp, caption=Lisp example]
->(setq x 10)
\end{lstlisting}



\subsection{Local Variables}


There are two other constructs - let and prog for creating local variables :


\subsubsection{Example}


\lstset{style=mystyle}
\begin{lstlisting}[language=Lisp, caption=Lisp example]
(let ((y 1)
      (z y))
      (list y z))
\end{lstlisting}

Result  $\to$ (1 2)\newline

             
\subsubsection{Example}


\lstset{style=mystyle}
\begin{lstlisting}[language=Lisp, caption=Lisp example]
(let ((str "Hello, world!"))
      (string-upcase str))
\end{lstlisting}

Result $\to$ "HELLO, WORLD!"

\newpage


\section{Definition of Compiler}

\begin{itemize}

	\item A compiler is a computer program that translates computer code written in one programming language (the source language) into another language (the target language). 
	
	\item The name compiler is primarily used for programs that translate source code from a high-level programming language to a lower level language (e.g., assembly language, object code, or machine code) to create an executable program.

\end{itemize}



\section{What is Decompiler?}

A program that translates from a low-level language to a higher level one is a decompiler.


\section{Source-to-Source compiler or transpiler}

A program that translates between high-level languages is usually called a source-to-source compiler or transpiler

\section{Alphabet, String, Language}

\begin{itemize}
	\item [Alphabet] finite set of symbols
	\item [String] finite sequence of sumbols
	\item [Language] set of strings on an alphabet 
\end{itemize}

\section{What is Parser?}

Syntax analysis (also known as parsing) involves parsing the token sequence to identify the syntactic structure of the program. This phase typically builds a parse tree, which replaces the linear sequence of tokens with a tree structure built according to the rules of a formal grammar which define the language's syntax. The parse tree is often analyzed, augmented, and transformed by later phases in the compiler.


\section{Types Of Compilers}

\subsection{native or hosted compiler}

A native or hosted compiler is one whose output is intended to directly run on the same type of computer and operating system that the compiler itself runs on.


\subsection{Cross compiler}

The output of a cross compiler is designed to run on a different platform. 

\subsection{Source-to-Source compiler}

Source-to-source compilers are a type of compiler that takes a high-level language as its input and outputs a high-level language.

\subsection{Bytecode compiler}

Bytecode compilers that compile to assembly language


\subsection{Just-in-time compilers}

Just-in-time compilers (JIT compiler) defer compilation until runtime. 


\subsection{Decompiler}

A program that translates from a low-level language to a higher level one is a decompiler.














\end{document}
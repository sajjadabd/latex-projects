\documentclass[12pt]{article}

\usepackage[most]{tcolorbox}

\tcbset{
    frame code={}
    center title,
    left=10pt,
    right=10pt,
    top=10pt,
    bottom=10pt,
    colback=gray!5,
    colframe=gray,
    width=\dimexpr\textwidth\relax,
    enlarge left by=0mm,
    boxsep=5pt,
    arc=0pt,outer arc=0pt,
}

\usepackage[margin=1.1in,footskip=.25in]{geometry}

\usepackage{amsmath, amssymb}
\usepackage{mathtools}

\usepackage{hyperref}
\hypersetup{
    colorlinks=true, %set true if you want colored links
    linktoc=all,     %set to all if you want both sections and subsections linked
    linkcolor=black,  %choose some color if you want links to stand out
}

\renewcommand{\baselinestretch}{1.3} 

\begin{document}

\tableofcontents


\newpage

\section{WordPress.com}

WordPress.com is a platform for self-publishing that is popular for blogging and other works. It is owned and operated by Automattic, Inc.[4] It is run on a modified version of WordPress.org, an open source piece of software used by bloggers.[5] This website provides free blog hosting for registered users and is financially supported via paid upgrades,[6] "VIP" services and advertising. 



\subsection{History}

The site opened to beta testers on August 8, 2005[7] and opened to the public on November 21, 2005.[4] It was initially launched as an invitation-only service, although at one stage, accounts were also available to users of the Flock web browser.[8] As of February 2017, over 77 million new posts and 42.7 million new comments are published monthly on the service.[9]

Registration is not required to read or comment on blogs hosted on the site, except if chosen by the blog owner. Registration is required to own, or post in, a weblog. All the basic and original features of the site are free-to-use. However, some features are not available in the free plan: install PHP plugins, customize theme CSS, write JavaScript, domain mapping, domain registration, removal of ads, website redirection, video upload, storage upgrades...[10]

Some notable clients include CNN, CBS, Sony, Fortune.com, and Volkswagen.[11][12][13] It is estimated that more than 30\% of internet bloggers use WordPress as their publishing platform.[14]

In September 2010, it was announced that Windows Live Spaces, Microsoft's blogging service, would be closing, and that Microsoft would partner with WordPress.com for blogging services.[15]

In December 2019, WordPress.com gave SFTP and PHPMyAdmin access on Business and eCommerce plans.[16] 


\subsection{WordPress.com vs WordPress.org}


Due to their similar names, WordPress.com and WordPress.org are often confused for each other. The main difference between the two is that WordPress.org hosts and creates the WordPress software, while WordPress.com provides freemium hosting using the WordPress software. WordPress the software is able to be self hosted or hosted with assistance on any website or server, while WordPress.com is simply one such provider for WordPress software hosting.

WordPress.com has several content stipulations due to how the platform works. For example, content hosted on their service isn't able to be copyrighted with an All Rights Reserved license, otherwise content hosted on WordPress.com would not be able to be shown by the hoster. Advertising is also limited on WordPress.com, whereas self hosted websites using WordPress from WordPress.org don't have such limitations. Themes and plugins are similarly limited, with some requiring specific hosting plans to be used on WordPress.com.

WordPress.com provides many services which would otherwise require hand tuning, like hosting of WordPress, as well as security updates and server maintenance. 



\newpage

\section{Wix.com}

Wix.com Ltd. is an Israeli software company, providing cloud-based web development services. It allows users to create HTML5 websites and mobile sites through the use of online drag and drop tools.[4] Along with its headquarters and other offices in Israel, Wix also has offices in Brazil, Canada, Germany, India, Ireland, Lithuania, United States, and Ukraine.[3]

Users can add social plug-ins, e-commerce, online marketing, contact forms, e-mail marketing, and community forums to their web sites using a variety of Wix-developed and third-party applications.[5] The Wix website builder is built on a freemium business model, earning its revenues through premium upgrades.[6] 



\subsection{History}

\subsubsection{Product development}

Wix was founded in 2006 by Israeli developers Avishai Abrahami, Nadav Abrahami, and Giora Kaplan. Headquartered in Tel Aviv, Wix was backed by investors Insight Venture Partners, Mangrove Capital Partners, Bessemer Venture Partners, DAG Ventures, and Benchmark Capital.[7] The company entered an open beta phase in 2007 using a platform based on Adobe Flash.[8]

By April 2010 Wix had 3.5 million users and raised US \$10 million in Series C funding provided by Benchmark Capital and existing investors Bessemer Venture Partners and Mangrove Capital Partners.[9] In March 2011, Wix had 8.5 million users and raised US \$40 million in Series D funding, bringing its total funding to that date to US \$61 million.[10]

In June 2011, Wix launched the Facebook store module, making its first step into the social commerce trend.[11] In March 2012, Wix launched a new HTML5 site builder, replacing the Adobe Flash technology.[12][13] In October 2012, Wix launched an app market for users to sell applications built with the company's automated web development technology. Wix's software development kit lets app developers create and offer web apps to the millions of Wix users around the globe.[14]

By August 2013, the Wix platform had more than 34 million registered users.[15]

On 15 May 2014, Wix launched the WixHive API which allows Wix apps within a user web site to capture and share their visitor data (such as contact information, messages, purchases and bookings) with other installed apps within the same web site.[16]

In December 2017, Wix Code, an API for adding database collections and custom JavaScript scripting, was released.[17] In March 2020, Wix Code was re-branded Corvid API.[18] 



\subsection{Use of WordPress code}

In October 2016, there was a controversy over Wix's use of WordPress's GPL-licensed code.[31] In response, Avishai Abrahami, Wix's CEO, published an answer explaining which open-source code was used and how Wix collaborates with the open-source community.[32] However, it was subsequently noted that collaboration with the open-source community was not sufficient under the terms of the GPL license, which requires any code built on GPL-licensed code to be released under the same license.[33]





\section{Joomla}

Joomla is a free and open-source content management system (CMS) for publishing web content, developed by Open Source Matters, Inc. It is built on a model–view–controller web application framework that can be used independently of the CMS.

Joomla is written in PHP, uses object-oriented programming techniques (since version 1.5) and software design patterns, stores data in a MySQL, MS SQL (since version 2.5), or PostgreSQL (since version 3.0) database, and includes features such as page caching, RSS feeds, printable versions of pages, news flashes, blogs, search, and support for language internationalization.[4][5][6]

Over 8,000 free and commercial extensions are available from the official Joomla Extensions Directory, and more are available from other sources.[7] As of 2019, it was estimated to be the fourth most used content management system on the Internet, after WordPress and Drupal.[8] 



\subsection{Historical background}

Joomla was the result of a fork of Mambo on August 17, 2005. At that time, the Mambo name was a trademark of Miro International Pvt. Ltd, who formed a non-profit foundation with the stated purpose of funding the project and protecting it from lawsuits.[9] The Joomla development team claimed that many of the provisions of the foundation structure violated previous agreements made by the elected Mambo Steering Committee, lacked the necessary consultation with key stakeholders and included provisions that violated core open source values.[10]

Joomla developers created a website called OpenSourceMatters.org (OSM) to distribute information to the software community. Project leader Andrew Eddie wrote a letter that appeared on the announcements section of the public forum at mamboserver.com.[11] Over one thousand people joined OpenSourceMatters.org within a day, most posting words of encouragement and support. Miro CEO Peter Lamont responded publicly to the development team in an article titled "The Mambo Open Source Controversy — 20 Questions With Miro".[12] This event created controversy within the free software community about the definition of open source. Forums of other open-source projects were active with postings about the actions of both sides.

In the two weeks following Eddie's announcement, teams were re-organized and the community continued to grow. Eben Moglen and the Software Freedom Law Center (SFLC) assisted the Joomla core team beginning in August 2005, as indicated by Moglen's blog entry from that date and a related OSM announcement.[13][14] The SFLC continue to provide legal guidance to the Joomla Project.[15]

On August 18, Andrew Eddie called for community input to suggest a name for the project. The core team reserved the right for the final naming decision and chose a name not suggested by the community. On September 22, the new name, Joomla!, was announced. It is the anglicised spelling of the Swahili word jumla, meaning all together or as a whole that also has a similar meaning in at least Amharic, Arabic and Urdu.[16] On September 26, the development team called for logo submissions from the community and invited the community to vote on the logo; the team announced the community's decision on September 29. On October 2, brand guidelines, a brand manual, and a set of logo resources were published.[17] 



\newpage

\section{Squarespace}


Squarespace, Inc. is a private American company, based in New York City, that provides software as a service for website building and hosting. Its customers use pre-built website templates and drag and drop elements to create webpages.

Anthony Casalena developed Squarespace as a blog-hosting service while attending the University of Maryland. He founded it as a company in 2004, and was its only employee until 2006, when it reached \$1 million in revenue. The company grew from 30 employees in 2010 to 550 by 2015. By 2014, it raised a total of \$78.5 million in venture capital; added e-commerce tools, domain name services, and analytics; and replaced its coding backend with drag and drop features. 



\subsection{Company history}

Casalena began developing Squarespace for his personal use while attending the University of Maryland.[2][3] He started sharing it with friends and family members[2] and participated in a "business incubator" program at the university.[3] He launched Squarespace publicly in January 2004,[3][4] initially funded by \$30,000 from his father, a small grant from the university, and 300 beta testers who paid a discounted rate.[3][5][6][7] At that time, Casalena was the company's sole developer and employee, and worked out of his dorm room.[3][6]

By the time Casalena graduated in 2007, Squarespace was making annual revenues of \$1 million.[4] He moved to New York City, began hiring, and had 30 employees by 2010.[4][7] That year, Squarespace received \$38.5 million in its first round of venture capital funding, enabling it to hire more staff, continue to develop its software,[8] and double its marketing budget.[2] From 2009 to 2012, it grew an average of 266\% in yearly revenue.[9] In April 2014, it received another \$40 million in funding.[10] By 2015, it had reached \$100 million in revenue and 550 employees.[4]

Squarespace has purchased Super Bowl advertising spots in 2014,[2] 2015,[11] 2016,[12] 2017[13] and 2018.[14] Its 2017 ad won an Emmy Award for Outstanding Commercial.[13] In 2017, it signed a sponsorship deal with the New York Knicks to add the Squarespace logo to their uniforms.[15]

After the Unite the Right rally in 2017, Squarespace received a petition with 58,000 signatures and removed a group of websites for violating its terms of service against "bigotry or hatred" towards demographic groups.[16][17] In 2017, it raised an additional \$200 million in funding, boosting its value to \$1.7 billion.[18] This funding was earmarked for reacquiring interests from investors.[18]

In 2018, Squarespace partnered with the Madison Square Garden Company to launch the "Make It Awards", which award \$30,000 to entrepreneurs (4 winners, totaling \$120,000).




\section{Blogger (service)}


Blogger is a blog-publishing service that allows multi-user blogs with time-stamped entries. It was developed by Pyra Labs, which was bought by Google in 2003. The blogs are hosted by Google and generally accessed from a subdomain of blogspot.com. Blogs can also be served from a custom domain owned by the user (like www.example.com) by using DNS facilities to direct a domain to Google's servers.[4][5][6] A user can have up to 100 blogs per account.[7]

Up until May 1, 2010, Blogger also allowed users to publish blogs to their own web hosting server, via FTP. All such blogs had to be changed to either use a blogspot.com subdomain, or point their own domain to Google's servers through DNS.[8] 






\subsection{History}


On August 23, 1999, Blogger was launched by Pyra Labs. As one of the earliest dedicated blog-publishing tools, it is credited for helping popularize the format. In February 2003, Pyra Labs was acquired by Google under undisclosed terms. The acquisition allowed premium features (for which Pyra had charged) to become free. In October 2004, Pyra Labs' co-founder, Evan Williams, left Google. In 2004, Google purchased Picasa; it integrated Picasa and its photo sharing utility Hello into Blogger, allowing users to post photos to their blogs.[citation needed]

On May 9, 2004, Blogger introduced a major redesign, adding features such as web standards-compliant templates, individual archive pages for posts, comments, and posting by email. On August 14, 2006, Blogger launched its latest version in beta, codenamed "Invader", alongside the gold release. This migrated users to Google servers and had some new features, including interface language in French, Italian, German and Spanish.[9] In December 2006, this new version of Blogger was taken out of beta. By May 2007, Blogger had completely moved over to Google-operated servers. Blogger was ranked 16 on the list of top 50 domains in terms of number of unique visitors in 2007.[10]

On February 24, 2015, Blogger announced that as of late March it will no longer allow its users to post sexually explicit content, unless the nudity offers "substantial public benefit," for example in "artistic, educational, documentary, or scientific contexts."[11] On February 28, 2015, accounting for severe backlash from long-term bloggers, Blogger reversed its decision on banning sexual content, going back to the previous policy that allowed explicit images and videos if the blog was marked as "adult".[12] 




\section{Tumblr}



Tumblr (stylized as tumblr and pronounced "tumbler") is an American microblogging and social networking website founded by David Karp in 2007 and currently owned by Automattic. The service allows users to post multimedia and other content to a short-form blog. Users can follow other users' blogs. Bloggers can also make their blogs private.[4][5] For bloggers many of the website's features are accessed from a "dashboard" interface.

As of August 12, 2019, Tumblr hosts over 475 million blogs.[6] As of January 2016, the website had over 500 million monthly visitors,[3] which dropped to less than 400 million by August 2019.[7] 





\subsection{History}




Development of Tumblr began in 2006 during a two-week gap between contracts at David Karp's software consulting company, Davidville (housed at Karp's former internship with producer-incubator Fred Seibert's Frederator Studios, which was located a block from Tumblr's current headquarters).[8][9] Karp had been interested in tumblelogs (short-form blogs, hence the name Tumblr[10]) for some time and was waiting for one of the established blogging platforms to introduce their own tumblelogging platform. As no one had done so after a year of waiting, Karp and developer Marco Arment began working on their own tumblelogging platform.[11][12] Tumblr was launched in February 2007,[13][14] and within two weeks the service had gained 75,000 users.[15] Arment left the company in September 2010 to focus on Instapaper.[16]

In early June 2012, Tumblr featured its first major brand advertising campaign in conjunction with Adidas, who launched an official soccer Tumblr blog and bought placements on the user dashboard. This launch came only two months after Tumblr announced it would be moving towards paid advertising on its site.[17]

On May 20, 2013, it was announced that Yahoo and Tumblr had reached an agreement for Yahoo! Inc. to acquire Tumblr for \$1.1 billion in cash.[18][19] Many of Tumblr's users were unhappy with the news, causing some to start a petition, achieving nearly 170,000 signatures.[20] David Karp remained CEO and the deal was finalized on June 20, 2013.[21][22] Advertising sales goals were not met and in 2016 Yahoo wrote down \$712 million of Tumblr's value.[23]

Verizon Communications acquired Yahoo in June 2017, and placed Yahoo and Tumblr under its Oath subsidiary.[24][1][25][26][27][28]

Karp announced in November 2017 that he would be leaving Tumblr by the end of the year. Jeff D'Onofrio, Tumblr's President and COO, took over leading the company.[29]

The site, along with the rest of the Oath division (renamed Verizon Media Group in 2019), continued to struggle under Verizon. In March 2019, SimilarWeb estimated Tumblr had lost 30\% of its user traffic since December 2018, when the site had introduced a stricter content policy with heavier restrictions on adult content (which had been a notable draw to the service).[30] In May 2019, it was reported that Verizon was considering selling the site due to its continued struggles since the purchase (as it had done with another Yahoo property, Flickr, via its sale to SmugMug). Following this news, Pornhub's vice president publicly expressed interest in purchasing Tumblr, with a promise to reinstate the previous adult content policies.[31][32]

On August 12, 2019, Verizon Media announced that it would sell Tumblr to Automattic—operator of blog service WordPress.com and corporate backer of the open source blog software of the same name—for an undisclosed amount. Axios reported that the sale price was less than \$3 million, a significant decrease over Yahoo's original purchase price. Automattic CEO Matt Mullenweg stated that the site will operate as a complementary service to WordPress.com, and that there were no plans to reverse the content policy decisions made during Verizon ownership.[33] 









\newpage

\section{Facebook}



Facebook is an American online social media and social networking service based in Menlo Park, California and a flagship service of the namesake company Facebook, Inc. It was founded by Mark Zuckerberg, along with fellow Harvard College students and roommates Eduardo Saverin, Andrew McCollum, Dustin Moskovitz and Chris Hughes.

The founders initially limited Facebook membership to Harvard students. Membership was expanded to Ivy League universities, MIT, and higher education institutions in the Boston area, then various other universities, and lastly high school students. Since 2006, anyone who claims to be at least 13 years old has been allowed to become a registered user of Facebook, though this may vary depending on local laws. The name comes from the face book directories often given to American university students.

The Facebook service can be accessed from devices with Internet connectivity, such as personal computers, tablets and smartphones. After registering, users can create a profile revealing information about themselves. They can post text, photos and multimedia which is shared with any other users that have agreed to be their "friend", or, with a different privacy setting, with any reader. Users can also use various embedded apps, join common-interest groups, buy and sell items or services on Marketplace, and receive notifications of their Facebook friends' activities and activities of Facebook pages they follow. Facebook claimed that it had more than 2.3 billion monthly active users as of December 2018.[8]

Facebook has been subject to extensive media coverage and many controversies. These often involve user privacy (as with the Cambridge Analytica data scandal), political manipulation (as with the 2016 U.S. elections), psychological effects such as addiction and low self-esteem, and content that some users find objectionable, including fake news, conspiracy theories, and copyright infringement.[9] Commentators have accused Facebook of helping to spread false information and fake news.[10][11][12][13] In 2017, Facebook partnered with fact checkers from the Poynter Institute's International Fact-Checking Network to identify and mark false content, though most ads from political candidates are exempt from this program.[14][15] Critics of the program accuse Facebook of not doing enough to remove false information from its website.[16] Facebook was the most downloaded mobile app of the 2010s globally.[17] 





\subsection{History}



Zuckerberg built a website called "Facemash" in 2003 while attending Harvard University. The site was comparable to Hot or Not and used "photos compiled from the online face books of nine Houses, placing two next to each other at a time and asking users to choose the "hotter" person".[18] Facemash attracted 450 visitors and 22,000 photo-views in its first four hours.[19] The site was sent to several campus group list-servers, but was shut down a few days later by Harvard administration. Zuckerberg faced expulsion and was charged with breaching security, violating copyrights and violating individual privacy. Ultimately, the charges were dropped.[18] Zuckerberg expanded on this project that semester by creating a social study tool ahead of an art history final exam. He uploaded all art images to a website, each of which was accompanied by a comments section, then shared the site with his classmates.[20]



A "face book" is a student directory featuring photos and personal information.[19] In 2003, Harvard had only a paper version[22] along with private online directories.[18][23] Zuckerberg told the Crimson, "Everyone's been talking a lot about a universal face book within Harvard. ... I think it's kind of silly that it would take the University a couple of years to get around to it. I can do it better than they can, and I can do it in a week."[23] In January 2004, Zuckerberg coded a new website, known as "TheFacebook", inspired by a Crimson editorial about Facemash, stating, "It is clear that the technology needed to create a centralized Website is readily available ... the benefits are many." Zuckerberg met with Harvard student Eduardo Saverin, and each of them agreed to invest \$1,000 in the site.[24] On February 4, 2004, Zuckerberg launched "TheFacebook", originally located at thefacebook.com.[25]

Six days after the site launched, Harvard seniors Cameron Winklevoss, Tyler Winklevoss, and Divya Narendra accused Zuckerberg of intentionally misleading them into believing that he would help them build a social network called HarvardConnection.com. They claimed that he was instead using their ideas to build a competing product.[26] The three complained to the Crimson and the newspaper began an investigation. They later sued Zuckerberg, settling in 2008[27] for 1.2 million shares (worth \$300 million at Facebook's IPO).[28]

Membership was initially restricted to students of Harvard College. Within a month, more than half the undergraduates had registered.[29] Dustin Moskovitz, Andrew McCollum, and Chris Hughes joined Zuckerberg to help manage the growth of the website.[30] In March 2004, Facebook expanded to Columbia, Stanford and Yale.[31] It then became available to all Ivy League colleges, Boston University, New York University, MIT, and successively most universities in the United States and Canada.[32][33]

In mid-2004, Napster co-founder and entrepreneur Sean Parker—an informal advisor to Zuckerberg—became company president.[34] In June 2004, the company moved to Palo Alto, California.[35] It received its first investment later that month from PayPal co-founder Peter Thiel.[36] In 2005, the company dropped "the" from its name after purchasing the domain name Facebook.com for US \$200,000.[37] The domain had belonged to AboutFace Corporation. 




In May 2005, Accel Partners invested \$12.7 million in Facebook, and Jim Breyer[38] added \$1 million of his own money. A high-school version of the site launched in September 2005.[39] Eligibility expanded to include employees of several companies, including Apple Inc. and Microsoft.[40]





\subsection{Website}

\subsubsection{2012 architecture}

Facebook is developed as one monolithic application. According to an interview in 2012 with Chuck Rossi, a build engineer at Facebook, Facebook compiles into a 1.5 GB binary blob which is then distributed to the servers using a custom BitTorrent-based release system. Rossi stated that it takes about 15 minutes to build and 15 minutes to release to the servers. The build and release process has zero downtime. Changes to Facebook are rolled out daily.[158]

Facebook used a combination platform based on HBase to store data across distributed machines. Using a tailing architecture, events are stored in log files, and the logs are tailed. The system rolls these events up and writes them to storage. The user interface then pulls the data out and displays it to users. Facebook handles requests as AJAX behavior. These requests are written to a log file using Scribe (developed by Facebook).[159]

Data is read from these log files using Ptail, an internally built tool to aggregate data from multiple Scribe stores. It tails the log files and pulls data out. Ptail data are separated into three streams and sent to clusters in different data centers (Plugin impression, News feed impressions, Actions (plugin + news feed)). Puma is used to manage periods of high data flow (Input/Output or IO). Data is processed in batches to lessen the number of times needed to read and write under high demand periods (A hot article generates many impressions and news feed impressions that cause huge data skews). Batches are taken every 1.5 seconds, limited by memory used when creating a hash table.[159]

Data is then output in PHP format. The backend is written in Java. Thrift is used as the messaging format so PHP programs can query Java services. Caching solutions display pages more quickly. The data is then sent to MapReduce servers where it is queried via Hive. This serves as a backup as the data can be recovered from Hive.[159] 






\subsubsection{User profile/personal timeline}



Each registered user on Facebook has a personal profile that shows their posts and content.[163] The format of individual user pages was revamped in September 2011 and became known as "Timeline", a chronological feed of a user's stories,[164][165] including status updates, photos, interactions with apps and events.[166] The layout let users add a "cover photo".[166] Users were given more privacy settings.[166] In 2007, Facebook launched Facebook Pages for brands and celebrities to interact with their fanbase.[167][168] 100,000 Pages launched in November.[169] In June 2009, Facebook introduced a "Usernames" feature, allowing users to choose a unique nickname used in the URL for their personal profile, for easier sharing.[170][171]

In February 2014, Facebook expanded the gender setting, adding a custom input field that allows users to choose from a wide range of gender identities. Users can also set which set of gender-specific pronoun should be used in reference to them throughout the site.[172][173][174] In May 2014, Facebook introduced a feature to allow users to ask for information not disclosed by other users on their profiles. If a user does not provide key information, such as location, hometown, or relationship status, other users can use a new "ask" button to send a message asking about that item to the user in a single click.[175][176] 








\subsubsection{News Feed}



News Feed appears on every user's homepage and highlights information including profile changes, upcoming events and friends' birthdays.[177] This enabled spammers and other users to manipulate these features by creating illegitimate events or posting fake birthdays to attract attention to their profile or cause.[178] Initially, the News Feed caused dissatisfaction among Facebook users; some complained it was too cluttered and full of undesired information, others were concerned that it made it too easy for others to track individual activities (such as relationship status changes, events, and conversations with other users).[179] Zuckerberg apologized for the site's failure to include appropriate privacy features. Users then gained control over what types of information are shared automatically with friends. Users are now able to prevent user-set categories of friends from seeing updates about certain types of activities, including profile changes, Wall posts and newly added friends.[180]

On February 23, 2010, Facebook was granted a patent[181] on certain aspects of its News Feed. The patent covers News Feeds in which links are provided so that one user can participate in the activity of another user.[182] The sorting and display of stories in a user's News Feed is governed by the EdgeRank algorithm.[183]

The Photos application allows users to upload albums and photos.[184] Each album can contain 200 photos.[185] Privacy settings apply to individual albums. Users can "tag", or label, friends in a photo. The friend receives a notification about the tag with a link to the photo.[186] This photo tagging feature was developed by Aaron Sittig, now a Design Strategy Lead at Facebook, and former Facebook engineer Scott Marlette back in 2006 and was only granted a patent in 2011.[187][188]

On June 7, 2012, Facebook launched its App Center to help users find games and other applications.[189]

On May 13, 2015, Facebook in association with major news portals launched "Instant Articles" to provide news on the Facebook news feed without leaving the site.[190][191]

In January 2017, Facebook launched Facebook Stories for iOS and Android in Ireland. The feature, following the format of Snapchat and Instagram stories, allows users to upload photos and videos that appear above friends' and followers' News Feeds and disappear after 24 hours.[192]

On October 11, 2017, Facebook introduced the 3D Posts feature to allow for uploading interactive 3D assets.[193] On January 11, 2018, Facebook announced that it would change News Feed to prioritize friends/family content and de-emphasize content from media companies.[194] 





\subsubsection{Like button}


The "like" button, stylized as a "thumbs up" icon, was first enabled on February 9, 2009,[195] and enables users to easily interact with status updates, comments, photos and videos, links shared by friends, and advertisements. Once clicked by a user, the designated content is more likely to appear in friends' News Feeds.[196][197] The button displays the number of other users who have liked the content.[198] The like button was extended to comments in June 2010.[199] In February 2016, Facebook expanded Like into "Reactions", choosing among five pre-defined emotions, including "Love", "Haha", "Wow", "Sad", or "Angry".[200][201][202][203]







\subsubsection{Instant messaging}


Facebook Messenger is an instant messaging service and software application. It began as Facebook Chat in 2008,[204] was revamped in 2010[205] and eventually became a standalone mobile app in August 2011, while remaining part of the user page on browsers.[206]

Complementing regular conversations, Messenger lets users make one-to-one[207] and group[208] voice[209] and video calls.[210] Its Android app has integrated support for SMS[211] and "Chat Heads", which are round profile photo icons appearing on-screen regardless of what app is open,[212] while both apps support multiple accounts,[213] conversations with optional end-to-end encryption[214] and "Instant Games".[215] Some features, including sending money[216] and requesting transportation,[217] are limited to the United States.[216] In 2017, Facebook added "Messenger Day", a feature that lets users share photos and videos in a story-format with all their friends with the content disappearing after 24 hours;[218] Reactions, which lets users tap and hold a message to add a reaction through an emoji;[219] and Mentions, which lets users in group conversations type @ to give a particular user a notification.[219]

Businesses and users can interact through Messenger with features such as tracking purchases and receiving notifications, and interacting with customer service representatives. Third-party developers can integrate apps into Messenger, letting users enter an app while inside Messenger and optionally share details from the app into a chat.[220] Developers can build chatbots into Messenger, for uses such as news publishers building bots to distribute news.[221] The M virtual assistant (U.S.) scans chats for keywords and suggests relevant actions, such as its payments system for users mentioning money.[222][223] Group chatbots appear in Messenger as "Chat Extensions". A "Discovery" tab allows finding bots, and enabling special, branded QR codes that, when scanned, take the user to a specific bot.[224] 






\subsubsection{Following}

Users can "Follow" content posted by other users without needing to friend them.[225] Accounts can be "verified", confirming a user's identity.[226]






\section{YouTube}



YouTube is an American online video-sharing platform headquartered in San Bruno, California. Three former PayPal employees—Chad Hurley, Steve Chen, and Jawed Karim—created the service in February 2005. Google bought the site in November 2006 for US \$1.65 billion; YouTube now operates as one of Google's subsidiaries.

YouTube allows users to upload, view, rate, share, add to playlists, report, comment on videos, and subscribe to other users. It offers a wide variety of user-generated and corporate media videos. Available content includes video clips, TV show clips, music videos, short and documentary films, audio recordings, movie trailers, live streams, and other content such as video blogging, short original videos, and educational videos. Most content on YouTube is uploaded by individuals, but media corporations including CBS, the BBC, Vevo, and Hulu offer some of their material via YouTube as part of the YouTube partnership program. Unregistered users can only watch (but not upload) videos on the site, while registered users are also permitted to upload an unlimited number of videos and add comments to videos. Videos deemed potentially inappropriate are available only to registered users affirming themselves to be at least 18 years old.

YouTube and selected creators earn advertising revenue from Google AdSense, a program which targets ads according to site content and audience. The vast majority of its videos are free to view, but there are exceptions, including subscription-based premium channels, film rentals, as well as YouTube Music and YouTube Premium, subscription services respectively offering premium and ad-free music streaming, and ad-free access to all content, including exclusive content commissioned from notable personalities. As of February 2017, there were more than 400 hours of content uploaded to YouTube each minute, and one billion hours of content being watched on YouTube every day. As of August 2018, the website is ranked as the second-most popular site in the world, according to Alexa Internet, just behind Google.[2] As of May 2019, more than 500 hours of video content are uploaded to YouTube every minute.[7] Based on reported quarterly advertising revenue, YouTube is estimated to have US \$15 billion in annual revenues.

YouTube has faced criticism over aspects of its operations, including its handling of copyrighted content contained within uploaded videos,[8] its recommendation algorithms perpetuating videos that promote conspiracy theories and falsehoods,[9] hosting videos ostensibly targeting children but containing violent and/or sexually suggestive content involving popular characters,[10] videos of minors attracting pedophilic activities in their comment sections,[11] and fluctuating policies on the types of content that is eligible to be monetized with advertising.[8] 



\subsection{History}

\subsubsection{Founding and initial growth (2005–2006)}



YouTube was founded by Steve Chen, Chad Hurley, and Jawed Karim, who were all early employees of PayPal.[12] Hurley had studied design at Indiana University of Pennsylvania, and Chen and Karim studied computer science together at the University of Illinois at Urbana–Champaign.[13]

Karim said the inspiration for YouTube first came from Janet Jackson's role in the 2004 Super Bowl incident when her breast was exposed during her performance, and later from the 2004 Indian Ocean tsunami. Karim could not easily find video clips of either event online, which led to the idea of a video sharing site.[14] Hurley and Chen said that the original idea for YouTube was a video version of an online dating service, and had been influenced by the website Hot or Not.[15][16] Difficulty in finding enough dating videos led to a change of plans, with the site's founders deciding to accept uploads of any type of video.[17]

According to a story that has often been repeated in the media, Hurley and Chen developed the idea for YouTube during the early months of 2005, after they had experienced difficulty sharing videos that had been shot at a dinner party at Chen's apartment in San Francisco. Karim did not attend the party and denied that it had occurred, but Chen commented that the idea that YouTube was founded after a dinner party "was probably very strengthened by marketing ideas around creating a story that was very digestible".[15] 



YouTube began as a venture capital–funded technology startup, primarily from an $11.5 million investment by Sequoia Capital and an $8 million investment from Artis Capital Management between November 2005 and April 2006.[18][19] YouTube's early headquarters were situated above a pizzeria and Japanese restaurant in San Mateo, California.[20] The domain name www.youtube.com was activated on February 14, 2005, and the website was developed over the subsequent months.[21] The first YouTube video, titled Me at the zoo, shows co-founder Jawed Karim at the San Diego Zoo.[22] The video was uploaded on April 23, 2005, and can still be viewed on the site.[23] YouTube offered the public a beta test of the site in May 2005. The first video to reach one million views was a Nike advertisement featuring Ronaldinho in November 2005.[24][25] Following a \$3.5 million investment from Sequoia Capital in November, the site launched officially on December 15, 2005, by which time the site was receiving 8 million views a day.[26][27]

At the time of the official launch, YouTube did not have much market recognition. The week of YouTube's launch, NBC-Universal's Saturday Night Live ran a skit "Lazy Sunday" by The Lonely Island. Besides helping to bolster ratings and long-term viewership for Saturday Night Live, "Lazy Sunday" established YouTube as an important website as one of the first viral videos.[28] Unofficial uploads of the skit to YouTube drew in more than five million collective views by February 2006 before they were removed at request of NBC-Universal about two months later, raising questions of copyright related to viral content.[29] Even their short presence at the site not only helped spread the name of YouTube, but allowed users to see it was not just for video lovers and people who create their own video content, but a means to share other types of videos, such as television clips and music videos.[30][31] The site grew rapidly and, in July 2006, the company announced that more than 65,000 new videos were being uploaded every day, and that the site was receiving 100 million video views per day.[32] According to data published by market research company comScore, YouTube is the dominant provider of online video in the United States, with a market share of around 43\% and more than 14 billion views of videos in May 2010.[33]

In May 2011, 48 hours of new videos were uploaded to the site every minute,[34] which increased to 60 hours every minute in January 2012,[34] 100 hours every minute in May 2013,[35][36] 300 hours every minute in November 2014,[37] and 400 hours every minute in February 2017.[38] As of January 2012, the site had 800 million unique users a month.[39] It is has been claimed, by The Daily Telegraph in 2008, that in 2007, YouTube consumed as much bandwidth as the entire Internet in 2000.[40] According to third-party web analytics providers, Alexa and SimilarWeb, YouTube is the second-most visited website in the world, as of December 2016; SimilarWeb also lists YouTube as the top TV and video website globally, attracting more than 15 billion visitors per month.[2][41][42] In October 2006, YouTube moved to a new office in San Bruno, California.[43]

The choice of the name www.youtube.com led to problems for a similarly named website, www.utube.com. The site's owner, Universal Tube \& Rollform Equipment, filed a lawsuit against YouTube in November 2006 after being regularly overloaded by people looking for YouTube. Universal Tube has since changed the name of its website to www.utubeonline.com.[44][45] 



\subsubsection{Acquisition by Google (2006–2013)}



On October 9, 2006, Google Inc. announced that it had acquired YouTube for \$1.65 billion in Google stock,[46][47] and the deal was finalized on November 13, 2006.[48][49]


In March 2010, YouTube began free streaming of certain content, including 60 cricket matches of the Indian Premier League. According to YouTube, this was the first worldwide free online broadcast of a major sporting event.[50] On March 31, 2010, the YouTube website launched a new design, with the aim of simplifying the interface and increasing the time users spend on the site. Google product manager Shiva Rajaraman commented: "We really felt like we needed to step back and remove the clutter."[51] In May 2010, YouTube videos were watched more than two billion times per day.[52][53][54] This increased to three billion in May 2011,[55][56][57] and four billion in January 2012.[34][58] In February 2017, one billion hours of YouTube was watched every day.[59][60][61]

In October 2010, Hurley announced that he would be stepping down as chief executive officer of YouTube to take an advisory role, and that Salar Kamangar would take over as head of the company.[62] In April 2011, James Zern, a YouTube software engineer, revealed that 30\% of videos accounted for 99\% of views on the site.[63] In November 2011, the Google+ social networking site was integrated directly with YouTube and the Chrome web browser, allowing YouTube videos to be viewed from within the Google+ interface.[64] 


In December 2011, YouTube launched a new version of the site interface, with the video channels displayed in a central column on the home page, similar to the news feeds of social networking sites.[65] At the same time, a new version of the YouTube logo was introduced with a darker shade of red, the first change in design since October 2006.[66]

In early March 2013, YouTube finalized the transition for all channels to the previously[when?] optional "One Channel Layout," which removed many customization options and custom background images for consistency, and split up the channel information to different tabs (Home/Feed, Videos Playlists, Discussion, About) rather than one unified page.[67] 




\subsection{Features}


\subsubsection{Video technology}

YouTube primarily uses the VP9 and H.264/MPEG-4 AVC video formats, and the Dynamic Adaptive Streaming over HTTP protocol.[86] By January 2019, YouTube had begun rolling out videos in AV1 format.[87]





\subsubsection{Playback}

Previously, viewing YouTube videos on a personal computer required the Adobe Flash Player plug-in to be installed in the browser.[88] In January 2010, YouTube launched an experimental version of the site that used the built-in multimedia capabilities of web browsers supporting the HTML5 standard.[89] This allowed videos to be viewed without requiring Adobe Flash Player or any other plug-in to be installed.[90][91] The YouTube site had a page that allowed supported browsers to opt into the HTML5 trial. Only browsers that supported HTML5 Video using the MP4 (with H.264 video) or WebM (with VP8 video) formats could play the videos, and not all videos on the site were available.[92][93]

On January 27, 2015, YouTube announced that HTML5 would be the default playback method on supported browsers. YouTube used to employ Adobe Dynamic Streaming for Flash,[94] but with the switch to HTML5 video now streams video using Dynamic Adaptive Streaming over HTTP (MPEG-DASH), an adaptive bit-rate HTTP-based streaming solution optimizing the bitrate and quality for the available network.[95] 


\subsubsection{Uploading}

All YouTube users can upload videos up to 15 minutes each in duration. Users who have a good track record of complying with the site's Community Guidelines may be offered the ability to upload videos up to 12 hours in length, as well as live streams, which requires verifying the account, normally through a mobile phone.[96][97] When YouTube was launched in 2005, it was possible to upload longer videos, but a ten-minute limit was introduced in March 2006 after YouTube found that the majority of videos exceeding this length were unauthorized uploads of television shows and films.[98] The 10-minute limit was increased to 15 minutes in July 2010.[99] In the past, it was possible to upload videos longer than 12 hours. Videos can be at most 128 GB in size.[96] Video captions are made using speech recognition technology when uploaded. Such captioning is usually not perfectly accurate, so YouTube provides several options for manually entering the captions for greater accuracy.[100]

YouTube accepts videos that are uploaded in most container formats, including AVI, MP4, MPEG-PS, QuickTime File Format and FLV. It supports WebM files and also 3GP, allowing videos to be uploaded from mobile phones.[101]

Videos with progressive scanning or interlaced scanning can be uploaded, but for the best video quality, YouTube suggests interlaced videos be deinterlaced before uploading. All the video formats on YouTube use progressive scanning.[102] YouTube's statistics shows that interlaced videos are still being uploaded to YouTube, and there is no sign of that actually dwindling. YouTube attributes this to uploading of made-for-TV content.[103] 

\subsubsection{Quality and formats}

YouTube originally offered videos at only one quality level, displayed at a resolution of 320$\times$240 pixels using the Sorenson Spark codec (a variant of H.263),[104][105] with mono MP3 audio.[106] In June 2007, YouTube added an option to watch videos in 3GP format on mobile phones.[107] In March 2008, a high-quality mode was added, which increased the resolution to 480$\times$360 pixels.[108] In December 2008, 720p HD support was added. At the time of the 720p launch, the YouTube player was changed from a 4:3 aspect ratio to a widescreen 16:9.[109] With this new feature, YouTube began a switchover to H.264/MPEG-4 AVC as its default video compression format. In November 2009, 1080p HD support was added. In July 2010, YouTube announced that it had launched a range of videos in 4K format, which allows a resolution of up to 4096$\times$3072 pixels.[110][111] In March 2015, support for 4K resolution was added, with the videos playing at 3840$\times$2160 pixels. In June 2015, support for 8K resolution was added, with the videos playing at 7680$\times$4320 pixels.[112] In November 2016, support for HDR video was added which can be encoded with Hybrid Log-Gamma (HLG) or Perceptual Quantizer (PQ).[113] HDR video can be encoded with the Rec. 2020 color space.[114]

In June 2014, YouTube began to deploy support for high frame rate videos up to 60 frames per second (as opposed to 30 before), becoming available for user uploads in October. YouTube stated that this would enhance "motion-intensive" videos, such as video game footage.[115][116][117][118]

YouTube videos are available in a range of quality levels. The former names of standard quality (SQ), high quality (HQ), and high definition (HD) have been replaced by numerical values representing the vertical resolution of the video. The default video stream is encoded in the VP9 format with stereo Opus audio; if VP9/WebM is not supported in the browser/device or the browser's user agent reports Windows XP, then H.264/MPEG-4 AVC video with stereo AAC audio is used instead.[119] 



\subsubsection{Live streaming}


YouTube carried out early experiments with live streaming, including a concert by U2 in 2009, and a question-and-answer session with US President Barack Obama in February 2010.[126] These tests had relied on technology from 3rd-party partners, but in September 2010, YouTube began testing its own live streaming infrastructure.[127] In April 2011, YouTube announced the rollout of YouTube Live, with a portal page at the URL "www.youtube.com/live". The creation of live streams was initially limited to select partners.[128] It was used for real-time broadcasting of events such as the 2012 Olympics in London.[129] In October 2012, more than 8 million people watched Felix Baumgartner's jump from the edge of space as a live stream on YouTube.[130]

In May 2013, creation of live streams was opened to verified users with at least 1,000 subscribers; in August of that year the number was reduced to 100 subscribers,[131] and in December the limit was removed.[132] In February 2017, live streaming was introduced to the official YouTube mobile app. Live streaming via mobile was initially restricted to users with at least 10,000 subscribers,[133] but as of mid-2017 it has been reduced to 100 subscribers.[134] Live streams can be up to 4K resolution at 60 fps, and also support $360^{\circ}$ video.[135] In February 2017, a live streaming feature called Super Chat was introduced, which allows viewers to donate between \$1 and \$500 to have their comment highlighted.[136] 



\subsubsection{3D videos}

In a video posted on July 21, 2009,[137] YouTube software engineer Peter Bradshaw announced that YouTube users could now upload 3D videos. The videos can be viewed in several different ways, including the common anaglyph (cyan/red lens) method which utilizes glasses worn by the viewer to achieve the 3D effect.[138][139][140] The YouTube Flash player can display stereoscopic content interleaved in rows, columns or a checkerboard pattern, side-by-side or anaglyph using a red/cyan, green/magenta or blue/yellow combination. In May 2011, an HTML5 version of the YouTube player began supporting side-by-side 3D footage that is compatible with Nvidia 3D Vision.[141] The feature set has since been reduced, and the 3D feature currently only supports red/cyan anaglyph with no side-by-side support. 




\subsubsection{360-degree videos}

In January 2015, Google announced that 360-degree video would be natively supported on YouTube. On March 13, 2015, YouTube enabled $360^{\circ}$ videos which can be viewed from Google Cardboard, a virtual reality system. YouTube 360 can also be viewed from all other virtual reality headsets.[142] Live streaming of $360^{\circ}$ video at up to 4K resolution is also supported.[135]

In 2017, YouTube began to promote an alternative stereoscopic video format known as VR180, which is limited to a 180-degree field of view but is promoted as being easier to produce than 360-degree video and allowing more depth to be maintained by not subjecting the video to equirectangular projection.[143] 




\subsection{Social impact}


Both private individuals[280] and large production companies[281] have used YouTube to grow audiences. Independent content creators have built grassroots followings numbering in the thousands at very little cost or effort, while mass retail and radio promotion proved problematic.[280] Concurrently, old media celebrities moved into the website at the invitation of a YouTube management that witnessed early content creators accruing substantial followings, and perceived audience sizes potentially larger than that attainable by television.[281] While YouTube's revenue-sharing "Partner Program" made it possible to earn a substantial living as a video producer—its top five hundred partners each earning more than \$100,000 annually[282] and its ten highest-earning channels grossing from \$2.5 million to \$12 million[283]—in 2012 CMU business editor characterized YouTube as "a free-to-use ... promotional platform for the music labels."[284] In 2013 Forbes' Katheryn Thayer asserted that digital-era artists' work must not only be of high quality, but must elicit reactions on the YouTube platform and social media.[285] Videos of the 2.5\% of artists categorized as "mega", "mainstream" and "mid-sized" received 90.3\% of the relevant views on YouTube and Vevo in that year.[286] By early 2013 Billboard had announced that it was factoring YouTube streaming data into calculation of the Billboard Hot 100 and related genre charts.[287]




Observing that face-to-face communication of the type that online videos convey has been "fine-tuned by millions of years of evolution," TED curator Chris Anderson referred to several YouTube contributors and asserted that "what Gutenberg did for writing, online video can now do for face-to-face communication."[288] Anderson asserted that it is not far-fetched to say that online video will dramatically accelerate scientific advance, and that video contributors may be about to launch "the biggest learning cycle in human history."[288] In education, for example, the Khan Academy grew from YouTube video tutoring sessions for founder Salman Khan's cousin into what Forbes' Michael Noer called "the largest school in the world," with technology poised to disrupt how people learn.[289] YouTube was awarded a 2008 George Foster Peabody Award,[290] the website being described as a Speakers' Corner that "both embodies and promotes democracy."[291] The Washington Post reported that a disproportionate share of YouTube's most subscribed channels feature minorities, contrasting with mainstream television in which the stars are largely white.[292] A Pew Research Center study reported the development of "visual journalism," in which citizen eyewitnesses and established news organizations share in content creation.[293] The study also concluded that YouTube was becoming an important platform by which people acquire news.[294]

YouTube has enabled people to more directly engage with government, such as in the CNN/YouTube presidential debates (2007) in which ordinary people submitted questions to U.S. presidential candidates via YouTube video, with a techPresident co-founder saying that Internet video was changing the political landscape.[295] Describing the Arab Spring (2010–2012), sociologist Philip N. Howard quoted an activist's succinct description that organizing the political unrest involved using "Facebook to schedule the protests, Twitter to coordinate, and YouTube to tell the world."[296] In 2012, more than a third of the U.S. Senate introduced a resolution condemning Joseph Kony 16 days after the "Kony 2012" video was posted to YouTube, with resolution co-sponsor Senator Lindsey Graham remarking that the video "will do more to lead to (Kony's) demise than all other action combined."[297] 




Conversely, YouTube has also allowed government to more easily engage with citizens, the White House's official YouTube channel being the seventh top news organization producer on YouTube in 2012[300] and in 2013 a healthcare exchange commissioned Obama impersonator Iman Crosson's YouTube music video spoof to encourage young Americans to enroll in the Affordable Care Act (Obamacare)-compliant health insurance.[301] In February 2014, U.S. President Obama held a meeting at the White House with leading YouTube content creators to not only promote awareness of Obamacare[302] but more generally to develop ways for government to better connect with the "YouTube Generation."[298] Whereas YouTube's inherent ability to allow presidents to directly connect with average citizens was noted, the YouTube content creators' new media savvy was perceived necessary to better cope with the website's distracting content and fickle audience.[298]

Some YouTube videos have themselves had a direct effect on world events, such as Innocence of Muslims (2012) which spurred protests and related anti-American violence internationally.[303] TED curator Chris Anderson described a phenomenon by which geographically distributed individuals in a certain field share their independently developed skills in YouTube videos, thus challenging others to improve their own skills, and spurring invention and evolution in that field.[288] Journalist Virginia Heffernan stated in The New York Times that such videos have "surprising implications" for the dissemination of culture and even the future of classical music.[304]

The Legion of Extraordinary Dancers[305] and the YouTube Symphony Orchestra[306] selected their membership based on individual video performances.[288][306] Further, the cybercollaboration charity video "We Are the World 25 for Haiti (YouTube edition)" was formed by mixing performances of 57 globally distributed singers into a single musical work,[307] with The Tokyo Times noting the "We Pray for You" YouTube cyber-collaboration video as an example of a trend to use crowdsourcing for charitable purposes.[308] The anti-bullying It Gets Better Project expanded from a single YouTube video directed to discouraged or suicidal LGBT teens,[309] that within two months drew video responses from hundreds including U.S. President Barack Obama, Vice President Biden, White House staff, and several cabinet secretaries.[310] Similarly, in response to fifteen-year-old Amanda Todd's video "My story: Struggling, bullying, suicide, self-harm," legislative action was undertaken almost immediately after her suicide to study the prevalence of bullying and form a national anti-bullying strategy.[311] In May 2018, London Metropolitan Police claimed that the drill videos that talk about violence give rise to the gang-related violence. YouTube deleted 30 music videos after the complaint.[312] 




\section{Instagram}



Instagram is an American photo and video-sharing social networking service owned by Facebook, Inc. It was created by Kevin Systrom and Mike Krieger, and launched in October 2010 on iOS. A version for Android devices was released in April 2012, followed by a feature-limited website interface in November 2012, a Fire OS app on June 15, 2014 and an app for Windows 10 tablets and computers in October 2016. The app allows users to upload photos and videos, which can be edited with filters and organized with tags and location information. Posts can be shared publicly or with pre-approved followers. Users can browse other users' content by tags and locations, and view trending content. Users can like photos and follow other users to add their content to a feed.

Instagram was originally distinguished by only allowing content to be framed in a square (1:1) aspect ratio with 640 pixels to match the display width of the iPhone at the time. These restrictions were eased in 2015, with an increase to 1080 pixels. The service also added messaging features, the ability to include multiple images or videos in a single post, as well as "Stories"—similar to its main competitor Snapchat—which allows users to post photos and videos to a sequential feed, with each post accessible by others for 24 hours each. As of January 2019, the Stories feature is used by 500 million users daily.[10]

After its launch in 2010, Instagram rapidly gained popularity, with one million registered users in two months, 10 million in a year, and 1 billion as of May 2019. In April 2012, Facebook acquired the service for approximately US\$1 billion in cash and stock. As of October 2015, over 40 billion photos had been uploaded. Although praised for its influence, Instagram has been the subject of criticism, most notably for policy and interface changes, allegations of censorship, and illegal or improper content uploaded by users.

As of April 2020, the most followed person is footballer Cristiano Ronaldo with over 211 million followers, and the most followed woman is singer Ariana Grande with over 180 million followers. As of January 14, 2019, the most liked photo on Instagram is a picture of an egg, posted by the account @world\_record\_egg, created with the sole purpose of surpassing the previous record of 18 million likes on a Kylie Jenner post. The picture currently has over 54 million likes.[11] Instagram was the 4th most downloaded mobile app of the 2010s.[12]





\subsection{History}

Instagram began development in San Francisco as Burbn, a mobile check-in app created by Kevin Systrom and Mike Krieger.[13] Realizing that Burbn was too similar to Foursquare, Systrom and Krieger refocused their app on photo-sharing, which had become a popular feature among Burbn users.[14] They renamed the app Instagram, a portmanteau of instant camera and telegram.[15]




\subsubsection{2010–2011: Beginnings and major funding}


On March 5, 2010, Systrom closed a \$500,000 seed funding round with Baseline Ventures and Andreessen Horowitz while working on Burbn.[16] Josh Riedel joined the company in October as Community Manager,[17] Shayne Sweeney joined in November as an engineer,[17] and Jessica Zollman joined as a Community Evangelist in August 2011.[17][18]

The first Instagram post was a photo of South Beach Harbor at Pier 38, posted by Mike Krieger on July 16, 2010.[19][20] Systrom shared his first post, a picture of a dog and his girlfriend's foot, a few hours later (at 9:24 PM). It has been wrongly attributed as the first Instagram photo due to the earlier letter of the alphabet in its URL.[21][22][better source needed] On October 6, 2010, the Instagram iOS app was officially released through the App Store.[23]

In February 2011, it was reported that Instagram had raised \$7 million in Series A funding from a variety of investors, including Benchmark Capital, Jack Dorsey, Chris Sacca (through Capital fund), and Adam D'Angelo.[24] The deal valued Instagram at around \$20 million.[25] In April 2012, Instagram raised \$50 million from venture capitalists with a \$500 million valuation.[26] Joshua Kushner was the second largest investor in Instagram's Series B fundraising round, leading his investment firm, Thrive Capital, to double its money after the sale to Facebook.[27] 




\subsubsection{2012–2014: Additional platforms and acquisition by Facebook}



On April 3, 2012, Instagram released a version of its app for Android phones,[28][29] and it was downloaded more than one million times in less than one day.[30] The Android app has since received two significant updates: first, in March 2014, which cut the file size of the app by half and added performance improvements;[31][32] then in April 2017, to add an offline mode that allows users to view and interact with content without an Internet connection. At the time of the announcement, it was reported that 80\% of Instagram's 600 million users are located outside the U.S., and while the aforementioned functionality was live at its announcement, Instagram also announced its intention to make more features available offline, and that they were "exploring an iOS version".[33][34][35]

On April 9, 2012, Facebook, Inc. bought Instagram for \$1 billion in cash and stock,[36][37][38] with a plan to keep the company independently managed.[39][40][41] Britain's Office of Fair Trading approved the deal on August 14, 2012,[42] and on August 22, 2012, the Federal Trade Commission in the U.S. closed its investigation, allowing the deal to proceed.[43] On September 6, 2012, the deal between Instagram and Facebook officially closed with a purchase price of \$300 million in cash and 23 million shares of stock.[44][45]

The deal closed just before Facebook's scheduled initial public offering according to CNN. [41] The deal price was compared to the \$35 million Yahoo! paid for Flickr in 2005.[41] Mark Zuckerberg said Facebook was "committed to building and growing Instagram independently." [41] According to Wired, the deal netted Systrom \$400 million.[46]

In November 2012, Instagram launched website profiles, allowing anyone to see user feeds from a web browser with limited functionality. [47]

Since the app's launch it had used the Foursquare API technology to provide named location tagging. In March 2014, Instagram started to test and switch the technology to use Facebook Places.[48][49] 






\subsection{Features and tools}


Users can upload photographs and short videos, follow other users' feeds,[86] and geotag images with the name of a location.[87] Users can set their account as "private", thereby requiring that they approve any new follower requests.[88] Users can connect their Instagram account to other social networking sites, enabling them to share uploaded photos to those sites.[89] In September 2011, a new version of the app included new and live filters, instant tilt–shift, high-resolution photographs, optional borders, one-click rotation, and an updated icon.[90][91] Photos were initially restricted to a square, 1:1 aspect ratio; since August 2015, the app supports portrait and widescreen aspect ratios as well.[92][93][94] Users could formerly view a map of a user's geotagged photos. The feature was removed in September 2016, citing low usage.[95][96]

Since December 2016, posts can be "saved" into a private area of the app.[97][98] The feature was updated in April 2017 to let users organize saved posts into named collections.[99][100] Users can also "archive" their posts in a private storage area, out of visibility for the public and other users. The move was seen as a way to prevent users from deleting photos that don't garner a desired number of "likes" or are deemed boring, but also as a way to limit the "emergent behavior" of deleting photos, which deprives the service of content.[101][102] In August, Instagram announced that it would start organizing comments into threads, letting users more easily interact with replies.[103][104]

Since February 2017, up to ten pictures or videos can be included in a single post, with the content appearing as a swipeable carousel.[105][106] The feature originally limited photos to the square format, but received an update in August to enable portrait and landscape photos instead.[107][108]

In April 2018, Instagram launched its version of a portrait mode called "focus mode," which gently blurs the background of a photo or video while keeping the subject in focus when selected.[109] In November, Instagram began to support Alt text to add descriptions of photos for the visually impaired. They are either generated automatically using object recognition (using existing Facebook technology) or manually specified by the uploader.[110] 






\subsubsection{Hashtags}

In January 2011, Instagram introduced hashtags to help users discover both photos and each other.[111][112] Instagram encourages users to make tags both specific and relevant, rather than tagging generic words like "photo", to make photographs stand out and to attract like-minded Instagram users.[113]

Users on Instagram have created "trends" through hashtags. The trends deemed the most popular on the platform often highlight a specific day of the week to post the material on. Examples of popular trends include \#SelfieSunday, in which users post a photo of their faces on Sundays; \#MotivationMonday, in which users post motivational photos on Mondays; \#TransformationTuesday, in which users post photos highlighting differences from the past to the present; \#WomanCrushWednesday, in which users post photos of women they have a romantic interest in or view favorably, as well as its \#ManCrushMonday counterpart centered on men; and \#ThrowbackThursday, in which users post a photo from their past, highlighting a particular moment.[114][115]

In December 2017, Instagram began to allow users to follow hashtags, which display relevant highlights of the topic in their feeds.[116][117] 




\subsubsection{Explore}


In June 2012, Instagram introduced "Explore", a tab inside the app that displays popular photos, photos taken at nearby locations, and search.[118] The tab was updated in June 2015 to feature trending tags and places, curated content, and the ability to search for locations.[119] In April 2016, Instagram added a "Videos You Might Like" channel to the tab,[120][121] followed by an "Events" channel in August, featuring videos from concerts, sports games, and other live events,[122][123] followed by the addition of Instagram Stories in October.[124][125] The tab was later expanded again in November 2016 after Instagram Live launched to display an algorithmically-curated page of the "best" Instagram Live videos currently airing.[126] In May 2017, Instagram once again updated the Explore tab to promote public Stories content from nearby places.[127]




\subsubsection{Video}

Initially a purely photo-sharing service, Instagram incorporated 15-second video sharing in June 2013.[140][141] The addition was seen by some in the technology media as Facebook's attempt at competing with the then-popular video-sharing application Vine.[142][143] In August 2015, Instagram added support for widescreen videos.[144][145] In March 2016, Instagram increased the 15-second video limit to 60 seconds.[146][147] Albums were introduced in February 2017, which allow up to 10 minutes of video to be shared in one post.[105][106][148]








\subsubsection{IGTV}


IGTV is a vertical video application launched by Instagram[149] in June 2018. Basic functionality is also available within the Instagram app and website. IGTV allows uploads of up to 10 minutes in length with a file size of up to 650 MB, with verified and popular users allowed to upload videos of up to 60 minutes in length with a file size of up to 5.4 GB.[150] The app automatically begins playing videos as soon as it is launched, which CEO Kevin Systrom contrasted to video hosts where one must first locate a video.[151][152][153]








\subsubsection{Instagram Direct}



In December 2013, Instagram announced Instagram Direct, a feature that lets users interact through private messaging. Users who follow each other can send private messages with photos and videos, in contrast to the public-only requirement that was previously in place. When users receive a private message from someone they don't follow, the message is marked as pending and the user must accept to see it. Users can send a photo to a maximum of 15 people.[154][155][156] The feature received a major update in September 2015, adding conversation threading and making it possible for users to share locations, hashtag pages, and profiles through private messages directly from the news feed. Additionally, users can now reply to private messages with text, emoji or by clicking on a heart icon. A camera inside Direct lets users take a photo and send it to the recipient without leaving the conversation.[157][158][159] A new update in November 2016 let users make their private messages "disappear" after being viewed by the recipient, with the sender receiving a notification if the recipient takes a screenshot.[160][161] In April 2017, Instagram redesigned Direct to combine all private messages, both permanent and ephemeral, into the same message threads.[162][163][164] In May, Instagram made it possible to send website links in messages, and also added support for sending photos in their original portrait or landscape orientation without cropping.[165][166]





\subsubsection{Instagram Stories}




In August 2016, Instagram launched Instagram Stories, a feature that allows users to take photos, add effects and layers, and add them to their Instagram story. Images uploaded to a user's story expire after 24 hours. The media noted the feature's similarities to Snapchat.[167][168] In response to criticism that it copied functionality from Snapchat, CEO Kevin Systrom told Recode that "Day One: Instagram was a combination of Hipstamatic, Twitter [and] some stuff from Facebook like the 'Like' button. You can trace the roots of every feature anyone has in their app, somewhere in the history of technology". Although Systrom acknowledged the criticism as "fair", Recode wrote that "he likened the two social apps' common features to the auto industry: Multiple car companies can coexist, with enough differences among them that they serve different consumer audiences". Systrom further stated that "When we adopted [Stories], we decided that one of the really annoying things about the format is that it just kept going and you couldn't pause it to look at something, you couldn't rewind. We did all that, we implemented that." He also told the publication that Snapchat "didn't have filters, originally. They adopted filters because Instagram had filters and a lot of others were trying to adopt filters as well."[169][170]

In November, Instagram added live video functionality to Instagram Stories, allowing users to broadcast themselves live, with the video disappearing immediately after ending.[171][126]

In January 2017, Instagram launched skippable ads, where five-second photo and 15-second video ads appear in-between different stories.[172][173]

In April 2017, Instagram Stories incorporated augmented reality stickers, a "clone" of Snapchat's functionality.[174][175][176]

In May 2017, Instagram expanded the augmented reality sticker feature to support face filters, letting users add specific visual features onto their faces.[177][178]

Later in May, TechCrunch reported about tests of a Location Stories feature in Instagram Stories, where public Stories content at a certain location are compiled and displayed on a business, landmark or place's Instagram page.[179] A few days later, Instagram announced "Story Search", in which users can search for geographic locations or hashtags and the app displays relevant public Stories content featuring the search term.[127][180]

In June 2017, Instagram revised its live-video functionality to allow users to add their live broadcast to their story for availability in the next 24 hours, or discard the broadcast immediately.[181] In July, Instagram started allowing users to respond to Stories content by sending photos and videos, complete with Instagram effects such as filters, stickers, and hashtags.[182][183]

Stories were made available for viewing on Instagram's mobile and desktop websites in late August 2017.[184][185]

On December 5, 2017, Instagram introduced “Story Highlights”,[186] also known as “Permanent Stories”, which are similar to Instagram Stories, but don't expire. They appear as circles below the profile picture and biography and are accessible from the desktop website as well.

In June 2018, the daily active story users of Instagram had reached 400 million users, and monthly active users had reached 1 billion active users.[187] 



\newpage

\section{Twitter}


Twitter is an American microblogging and social networking service on which users post and interact with messages known as "tweets". Registered users can post, like, and retweet tweets, but unregistered users can only read them. Users access Twitter through its website interface, through Short Message Service (SMS) or its mobile-device application software ("app").[15] Twitter, Inc. is based in San Francisco, California, and has more than 25 offices around the world.[16] Tweets were originally restricted to 140 characters, but was doubled to 280 for non-Asian languages in November 2017.[17]

Twitter was created in March 2006 by Jack Dorsey, Noah Glass, Biz Stone, and Evan Williams, launched in July of that year. The service rapidly gained worldwide popularity. In 2012, more than 100 million users posted 340 million tweets a day,[18] and the service handled an average of 1.6 billion search queries per day.[19][20][21] In 2013, it was one of the ten most-visited websites and has been described as "the SMS of the Internet".[22][23] As of 2018, Twitter had more than 321 million monthly active users.[11] 







\subsection{History}


\subsubsection{Creation and initial reaction}


Twitter's origins lie in a "daylong brainstorming session" held by board members of the podcasting company Odeo. Jack Dorsey, then an undergraduate student at New York University, introduced the idea of an individual using an SMS service to communicate with a small group.[24][25] The original project code name for the service was twttr, an idea that Williams later ascribed to Noah Glass,[26] inspired by Flickr and the five-character length of American SMS short codes. The decision was also partly due to the fact that the domain twitter.com was already in use, and it was six months after the launch of twttr that the crew purchased the domain and changed the name of the service to Twitter.[27] The developers initially considered "10958" as a short code, but later changed it to "40404" for "ease of use and memorability".[28] Work on the project started on March 21, 2006, when Dorsey published the first Twitter message at 9:50 p.m. Pacific Standard Time (PST): "just setting up my twttr".[3] Dorsey has explained the origin of the "Twitter" title: 




\begin{tcolorbox}
...we came across the word 'twitter', and it was just perfect. The definition was 'a short burst of inconsequential information,' and 'chirps from birds'. And that's exactly what the product was.[29]
\end{tcolorbox}





The first Twitter prototype, developed by Dorsey and contractor Florian Weber, was used as an internal service for Odeo employees[30] and the full version was introduced publicly on July 15, 2006.[12] In October 2006, Biz Stone, Evan Williams, Dorsey, and other members of Odeo formed Obvious Corporation and acquired Odeo, together with its assets — including Odeo.com and Twitter.com — from the investors and shareholders.[31] Williams fired Glass, who was silent about his part in Twitter's startup until 2011.[32] Twitter spun off into its own company in April 2007.[33] Williams provided insight into the ambiguity that defined this early period in a 2013 interview: 




\begin{tcolorbox}
With Twitter, it wasn't clear what it was. They called it a social network, they called it microblogging, but it was hard to define, because it didn't replace anything. There was this path of discovery with something like that, where over time you figure out what it is. Twitter actually changed from what we thought it was in the beginning, which we described as status updates and a social utility. It is that, in part, but the insight we eventually came to was Twitter was really more of an information network than it is a social network.[34]
\end{tcolorbox}





The tipping point for Twitter's popularity was the 2007 South by Southwest Interactive (SXSWi) conference. During the event, Twitter usage increased from 20,000 tweets per day to 60,000.[35][36] "The Twitter people cleverly placed two 60-inch plasma screens in the conference hallways, exclusively streaming Twitter messages," remarked Newsweek's Steven Levy. "Hundreds of conference-goers kept tabs on each other via constant twitters. Panelists and speakers mentioned the service, and the bloggers in attendance touted it."[37] Reaction at the conference was highly positive. Blogger Scott Beale said that Twitter was "absolutely ruling" SXSWi. Social software researcher danah boyd said Twitter was "owning" the conference.[38] Twitter staff received the festival's Web Award prize with the remark "we'd like to thank you in 140 characters or less. And we just did!"[39]

The first unassisted off-Earth Twitter message was posted from the International Space Station by NASA astronaut T. J. Creamer on January 22, 2010.[40] By late November 2010, an average of a dozen updates per day were posted on the astronauts communal account, @NASA\_Astronauts. NASA has also hosted over 25 "tweetups", events that provide guests with VIP access to NASA facilities and speakers with the goal of leveraging participants' social networks to further the outreach goals of NASA. In August 2010, the company appointed Adam Bain from News Corp.'s Fox Audience Network as president of revenue.[41] 








\subsubsection{2007–2010}


The company experienced rapid initial growth. It had 400,000 tweets posted per quarter in 2007. This grew to 100 million tweets posted per quarter in 2008. In February 2010, Twitter users were sending 50 million tweets per day.[42] By March 2010, the company recorded over 70,000 registered applications.[43] As of June 2010, about 65 million tweets were posted each day, equaling about 750 tweets sent each second, according to Twitter.[44] As of March 2011, that was about 140 million tweets posted daily.[45] As noted on Compete.com, Twitter moved up to the third-highest-ranking social networking site in January 2009 from its previous rank of twenty-second.[46]




Twitter's usage spikes during prominent events. For example, a record was set during the 2010 FIFA World Cup when fans wrote 2,940 tweets per second in the thirty-second period after Japan scored against Cameroon on June 14. The record was broken again when 3,085 tweets per second were posted after the Los Angeles Lakers' victory in the 2010 NBA Finals on June 17,[47] and then again at the close of Japan's victory over Denmark in the World Cup when users published 3,283 tweets per second.[48] The record was set again during the 2011 FIFA Women's World Cup Final between Japan and the United States, when 7,196 tweets per second were published.[49] When American singer Michael Jackson died on June 25, 2009, Twitter servers crashed after users were updating their status to include the words "Michael Jackson" at a rate of 100,000 tweets per hour.[50] The current record as of August 3, 2013 was set in Japan, with 143,199 tweets per second during a television screening of the movie Castle in the Sky[51] (beating the previous record of 33,388, also set by Japan for the television screening of the same movie).[52]

Twitter acquired application developer Atebits on April 11, 2010. Atebits had developed the Apple Design Award-winning Twitter client Tweetie for the Mac and iPhone. The application, now called "Twitter" and distributed free of charge, is the official Twitter client for the iPhone, iPad and Mac.[53]

From September through October 2010, the company began rolling out "New Twitter", an entirely revamped edition of twitter.com. Changes included the ability to see pictures and videos without leaving Twitter itself by clicking on individual tweets which contain links to images and clips from a variety of supported websites including YouTube and Flickr, and a complete overhaul of the interface, which shifted links such as '@mentions' and 'Retweets' above the Twitter stream, while 'Messages' and 'Log Out' became accessible via a black bar at the very top of twitter.com. As of November 1, 2010, the company confirmed that the "New Twitter experience" had been rolled out to all users. In 2019, Twitter was announced to be the 10th most downloaded mobile app of the decade, from 2010 to 2019.[54] 






\subsubsection{2011–2014}




On April 5, 2011, Twitter tested a new homepage and phased out the "Old Twitter".[55] However, a glitch came about after the page was launched, so the previous "retro" homepage was still in use until the issues were resolved; the new homepage was reintroduced on April 20.[56][57] On December 8, 2011, Twitter overhauled its website once more to feature the "Fly" design, which the service says is easier for new users to follow and promotes advertising. In addition to the Home tab, the Connect and Discover tabs were introduced along with a redesigned profile and timeline of Tweets. The site's layout has been compared to that of Facebook.[58][59] On February 21, 2012, it was announced that Twitter and Yandex agreed to a partnership. Yandex, a Russian search engine, finds value within the partnership due to Twitter's real time news feeds. Twitter's director of business development explained that it is important to have Twitter content where Twitter users go.[60] On March 21, 2012, Twitter celebrated its sixth birthday while also announcing that it had 140 million users and 340 million tweets per day. The number of users was up 40\% from their September 2011 number, which was said to have been at 100 million at the time.[61]

In April 2012, Twitter announced that it was opening an office in Detroit, with the aim of working with automotive brands and advertising agencies.[62] Twitter also expanded its office in Dublin.[63] On June 5, 2012, a modified logo was unveiled through the company blog, removing the text to showcase the slightly redesigned bird as the sole symbol of Twitter.[64][65] On October 5, 2012, Twitter acquired a video clip company called Vine that launched in January 2013.[66][67] Twitter released Vine as a standalone app that allows users to create and share six-second looping video clips on January 24, 2013. Vine videos shared on Twitter are visible directly in users' Twitter feeds.[68] Due to an influx of inappropriate content, it is now rated 17+ in Apple's[needs update][69] app store.[70] On December 18, 2012, Twitter announced it had surpassed 200 million monthly active users. Twitter hit 100 million monthly active users in September 2011.[71]

On January 28, 2013, Twitter acquired Crashlytics in order to build out its mobile developer products.[72]

On April 18, 2013, Twitter launched a music app called Twitter Music for the iPhone.[73] On August 28, 2013, Twitter acquired Trendrr,[74] followed by the acquisition of MoPub on September 9, 2013.[75] As of September 2013, the company's data showed that 200 million users sent over 400 million tweets daily, with nearly 60\% of tweets sent from mobile devices.[76] On June 4, 2014, Twitter announced that it would acquire Namo Media, a technology firm specializing in "native advertising" for mobile devices.[77] On June 19, 2014, Twitter announced that it had reached an undisclosed deal to buy SnappyTV, a service that helps edit and share video from television broadcasts.[78][79] The company was helping broadcasters and rights holders to share video content both organically across social and via Twitter's Amplify program.[80] In July 2014, Twitter announced that it intended to buy a young company called CardSpring for an undisclosed sum. CardSpring enabled retailers to offer online shoppers coupons that they could automatically sync to their credit cards in order to receive discounts when they shopped in physical stores.[81] On July 31, 2014, Twitter announced that it had acquired a small password-security startup called Mitro.[82] On October 29, 2014, Twitter announced a new partnership with IBM. The partnership was intended to help businesses use Twitter data to understand their customers, businesses and other trends.[83] 






\subsubsection{2015–2016}




On February 11, 2015, Twitter announced that it had acquired Niche, an advertising network for social media stars, founded by Rob Fishman and Darren Lachtman.[84] The acquisition price was reportedly \$50 million.[85] On March 13, 2015, Twitter announced its acquisition of Periscope, an app that allows live streaming of video.[86] In April 2015, the Twitter.com desktop homepage changed.[87] Twitter announced that it had acquired TellApart, a commerce ads tech firm, with \$532 million stock.[88][89] Later in the year it became apparent that growth had slowed, according to Fortune,[90] Business Insider,[91] Marketing Land[92] and other news websites including Quartz (in 2016).[93] In June 2016, Twitter acquired an artificial intelligence startup called Magic Pony for \$150 million.[94][95]








\section{WhatsApp}




WhatsApp Messenger, or simply WhatsApp, is an American freeware, cross-platform messaging and Voice over IP (VoIP) service owned by Facebook, Inc.[45] It allows users to send text messages and voice messages,[46] make voice and video calls, and share images, documents, user locations, and other media.[47][48] WhatsApp's client application runs on mobile devices but is also accessible from desktop computers, as long as the user's mobile device remains connected to the Internet while they use the desktop app.[49] The service requires users to provide a standard cellular mobile number for registering with the service.[50] In January 2018, WhatsApp released a standalone business app targeted at small business owners, called WhatsApp Business, to allow companies to communicate with customers who use the standard WhatsApp client.[51][52]

The client application was created by WhatsApp Inc. of Mountain View, California, which was acquired by Facebook in February 2014 for approximately US\$19.3 billion.[53][54] It became the world's most popular messaging application by 2015,[55][56] and has over 2 billion users worldwide as of February 2020.[57] It has become the primary means of communication in multiple countries and locations, including Latin America, the Indian subcontinent, and large parts of Europe and Africa.[55] 





\subsection{History}


\subsubsection{2009–2014}

WhatsApp was founded in 2009 by Brian Acton and Jan Koum, former employees of Yahoo!. After leaving Yahoo! in September 2007, they took some time off in South America.[12] At one point, they applied for jobs at Facebook but were rejected.[12]

In January 2009, after purchasing an iPhone and realizing the potential of the app industry on the App Store, Koum and Acton began visiting Koum's friend Alex Fishman in West San Jose to discuss a new type of messaging app that would "show statuses next to individual names of the people".[citation needed] They realized that to take the idea further, they'd need an iPhone developer. Fishman visited RentACoder.com, found Russian developer Igor Solomennikov, and introduced him to Koum.

Koum named the app WhatsApp to sound like "what's up". On February 24, 2009, he incorporated WhatsApp Inc. in California. However, when early versions of WhatsApp kept crashing, Koum considered giving up and looking for a new job. Acton encouraged him to wait for a "few more months".[58]

In June 2009, Apple launched push notifications, allowing users to be pinged when they were not using an app. Koum changed WhatsApp so that when a user's status is changed, everyone in the user's network would be notified.[12] WhatsApp 2.0 was released with a messaging component and the number of active users suddenly increased to 250,000. Although Acton was working on another startup idea, he decided to join the company.[12] In October 2009, Acton persuaded five former friends at Yahoo! to invest \$250,000 in seed funding, and Acton became a co-founder and was given a stake. He officially joined WhatsApp on November 1.[12] After months at beta stage, the application launched in November 2009, exclusively on the App Store for the iPhone. Koum then hired a friend in Los Angeles, Chris Peiffer, to develop a BlackBerry version, which arrived two months later.[12]

To cover the primary cost of sending verification texts to users, WhatsApp was changed from a free service to a paid one. In December 2009, the ability to send photos was added to the iPhone version. By early 2011, WhatsApp was one of the top 20 apps at Apple's U.S. App Store.[12]

In April 2011, Sequoia Capital invested about \$8 million for more than 15\% of the company, after months of negotiation with Sequoia partner Jim Goetz.[59][60][61]

By February 2013, WhatsApp had about 200 million active users and 50 staff members. Sequoia invested another \$50 million, and WhatsApp was valued at \$1.5 billion.[12]

In a December 2013 blog post, WhatsApp claimed that 400 million active users used the service each month.[62] 



\newpage

\section{Telegram (software)}



Telegram is a cloud-based instant messaging and voice over IP service. Telegram client apps are available for Android, iOS, Windows Phone, Windows, macOS and Linux.[13] Users can send messages and exchange photos, videos, stickers, audio and files of any type.

Telegram's client-side code is open-source software but the source code for recent versions is not always immediately published,[16] whereas its server-side code is closed-source and proprietary.[17] The service also provides APIs to independent developers. As of April 2020, Telegram has 400 million monthly active users with at least 1.5 million new users signing up every day. The announcement included a promise to implement group video calls in 2020.[18]

Default messages and media in Telegram are encrypted when stored on its servers, but can be accessed by the Telegram service provider, who holds the encryption keys. In addition Telegram provides optional end-to-end encrypted "secret" chats between two online users, yet not for groups or channels.[19]

The client-server communication is also encrypted.[20] The service provides end-to-end encryption for voice calls.[21] 





\subsection{History}


\subsubsection{Development}


Telegram was launched in 2013 by the brothers Nikolai and Pavel Durov. Previously, the pair founded the Russian social network VK, which they left when it was taken over by the Mail.ru Group.[22][23] Nikolai Durov created the MTProto protocol that is the basis for the messenger, while Pavel provided financial support and infrastructure through his Digital Fortress fund with partner Axel Neff joining as a second co-founder.[24] Telegram Messenger LLP states that its end goal is not to bring profit,[25][26] but it is not currently structured as a non-profit organization.[27]

Telegram is registered as both an English LLP[28] and an American LLC.[29] It does not disclose where it rents offices or which legal entities it uses to rent them, citing the need to "shelter the team from unnecessary influence" and protect users from governmental data requests.[30] Pavel Durov has said that the service was headquartered in Berlin, Germany, between 2014[31] and early 2015, but moved to different jurisdictions after failing to obtain residence permits for everyone on the team.[32] Durov left Russia and is said to be moving from country to country with a small group of computer programmers consisting of 15 core members.[22][33] According to press reports, Telegram had employees in St. Petersburg.[32] The Telegram team is currently based in Dubai.[34] 








\subsubsection{Usage numbers}




In October 2013, Telegram had 100,000 daily active users.[23] On 24 March 2014, Telegram announced that it had reached 35 million monthly users and 15 million daily active users.[35] In October 2014, South Korean governmental surveillance plans drove many of its citizens to switch to Telegram.[31] In December 2014, Telegram announced that it had 50 million active users, generating 1 billion daily messages, and that it had 1 million new users signing up on its service every week[36], traffic doubled in five months with 2 billion daily messages.[37] In September 2015, an announcement stated that the app had 60 million active users and delivered 12 billion daily messages.[38] In February 2016, Telegram announced that it had 100 million monthly active users, with 350,000 new users signing up every day, delivering 15 billion messages daily.[39] In December 2017, Telegram reached 180 million monthly active users.[34] In March 2018, Telegram reached 200 million monthly active users.[40] On March 14, 2019, Pavel Durov claimed that “3 million new users signed up for Telegram within the last 24 hours.”[41] Durov didn't specify what prompted this flood of new sign-ups, but the period matched a prolonged technical outage experienced by Facebook and its family of apps, including Instagram.[42] On April 24, 2020 Telegram announced it had reached 400 million monthly active users.

According to the U.S. Securities and Exchange Commission, the number of monthly Telegram users as of October 2019 is 300 million people worldwide.[43] 






\subsection{Features}



\subsubsection{Account}


Telegram accounts are tied to telephone numbers and are verified by SMS.[44] Users can add multiple devices to their account and receive messages on each one. Connected devices can be removed individually or all at once. The associated number can be changed at any time and when doing so, the user's contacts will receive the new number automatically.[44][45][46] In addition, a user can set up an alias that allows them to send and receive messages without exposing their phone number.[47] Telegram accounts can be deleted at any time and they are deleted automatically after six months of inactivity by default, which can optionally be changed to 1 month and 12 months. Users can replace exact "last seen" timestamps with broader messages such as "last seen recently".[48]

The default method of authentication that Telegram uses for logins is SMS-based single-factor authentication.[49][50] All that is needed in order to log into an account and gain access to that user's cloud-based messages is a one-time passcode that is sent via SMS to the user's phone number.[50][51] These login SMS messages are known to have been intercepted in Iran, Russia and Germany, possibly in coordination with phone companies.[51][52][53] Pavel Durov has said that Telegram users in "troubled countries" should enable two-factor authentication by creating passwords, which Telegram allows, but does not require.[51][52] 





\subsubsection{Cloud-based messages}


Telegram's default messages are cloud-based and can be accessed on any of the user's connected devices. Users can share photos, videos, audio messages and other files (up to 1.5 gigabyte in size per file). Users can send messages to other users individually or to groups of up to 200,000 members.[54] Sent messages can be edited within 48 hours after they have been sent and deleted at any time on both sides. This gives user an ability to correct typos and retract messages that were sent by mistake.[55] The transmission of messages to Telegram Messenger LLP's servers is encrypted with the service's MTProto protocol.[56] According to Telegram's privacy policy, "all data is stored heavily encrypted and the encryption keys in each case are stored in several other DCs in different jurisdictions. This way local engineers or physical intruders cannot get access to user data".[57] This makes the messages' security roughly comparable to that of e-mail. Here, most providers employ client-server encryption as well, however usually with the standardized protocol Transport Layer Security. E-mails may or may not be encrypted on the servers. Telegram cloud messages and media remain on the servers at least until deleted by all participants. 







\subsubsection{Bots}


In June 2015, Telegram launched a platform for third-party developers to create bots.[58] Bots are Telegram accounts operated by programs. They can respond to messages or mentions, can be invited into groups and can be integrated into other programs. It also accepts online payments with credit cards and Apple Pay.[59] Dutch website Tweakers reported that an invited bot can potentially read all group messages when the bot controller changes the access settings silently at a later point in time. Telegram pointed out that it considered implementing a feature that would announce such a status change within the relevant group.[60] Also there are inline bots, which can be used from any chat screen. In order to activate an inline bot, user needs to type in the message field a bot's username and query. The bot then will offer its content. User can choose from that content and send it within a chat.[61]








\subsubsection{Channels}



In September 2015, Telegram added channels.[62] Channels are a form of one-way messaging where admins are able to post messages but other users are not. Any user is able to create and subscribe to channels. Channels can be created for broadcasting messages to an unlimited number of subscribers.[63] Channels can be publicly available with an alias and a permanent URL so anyone can join. Users who join a channel can see the entire message history. Users can join and leave channels at any time. Depending on a channel's settings, messages may be signed with the channel's name or with the username of the admin who posted them. Non-admin users are unable to see other users who've subscribed to the channel. Furthermore, users can mute a channel, meaning that the user will still receive messages, but won't be notified. Admins can give permission to post comments on the Telegram channel with help of bots. The admin of the channel can obtain general data about the channel. Each message has its own view counter, showing how many users have seen this message, this includes views from forwarded messages. As of May 2019, the creator of a channel can add a discussion group, a separate group where messages in the channel are automatically posted for subscribers to communicate.[64]

In December 2019, Bloomberg moved their messenger-based newsletter service from WhatsApp to Telegram after the former banned bulk and automated messaging.[65][66] The news service is attempting to grow its audience outside the U.S.[66] 










\subsubsection{Stickers}




Stickers are cloud-based, high-definition images intended to provide more expressive emoji. When typing in an emoji, the user is offered to send the respective sticker instead. Stickers come in collections called "sets", and multiple stickers can be offered for one emoji. Telegram comes with one default sticker set,[67] but users can install additional sticker sets provided by third-party contributors. Sticker sets installed from one client become automatically available to all other clients. Sticker images use WebP file format, which is better optimized to be transmitted over internet. 







\subsubsection{Drafts}


Drafts are unfinished messages synced across user devices. One can start typing a message on one device and continue on another. The draft will persist in editing area on any device until it is sent or removed.[68]





\subsubsection{Polls}


Polls are a feature of Telegram that are currently available on Android, iOS and the desktop applications. Polls have the option to be anonymous or visible. A user can enter multiple options into the poll. Quiz mode can also be enabled where a user can select the right answer for their poll and leave it to the group to guess. Quiz bots can also be added to track correct answers and even provide a global leaderboard[85]. 







\section{Viber}



Rakuten Viber is a cross-platform voice over IP (VoIP) and instant messaging (IM) software application operated by Japanese multinational company Rakuten, provided as freeware for the Android, iOS, Microsoft Windows, macOS and Linux platforms. Users are registered and identified through a cellular telephone number, although the service is accessible on desktop platforms without needing mobile connectivity.[8] In addition to instant messaging it allows users to exchange media such as images and video records, and also provides a paid international landline and mobile calling service called Viber Out.[9] As of 2018, there are over a billion registered users on the network.[10][11]

The software was developed in 2010 by Israel-based Viber Media, which was bought by Rakuten in 2014. Since 2017, its corporate name has been Rakuten Viber. It is based in Luxembourg.[12] Viber's offices are located in Amsterdam, Barcelona, Brest, Kiev[13], London, Manila, Minsk, Moscow, Paris, San Francisco, Singapore, Sofia, Tel Aviv, and Tokyo.[14] 










\subsection{Company history}



\subsubsection{Founding}



Viber Media might have been founded in Tel Aviv, Israel, in 2010 by Talmon Marco[15] and Igor Magazinnik, who are friends from the Israel Defense Forces where they were chief information officers.[16] Marco and Magazinnik are also co-founders of the P2P media and file-sharing client iMesh.[17] The company was run from Israel, with much of its development outsourced to Belarus in order to lower labor costs.[16] It was registered in Cyprus. Sani Maroli and Ofer Smocha soon joined the company as well.[18][19][20][21][22][23] Marco commented that Viber allows instant calling and synchronization with contacts because the ID is the user's cell number, unlike Skype which is modeled after a "buddy list" requiring registration and a password.[24]





\subsubsection{Monetisation}



In its first two years of availability, Viber did not generate revenues. It began doing so in 2013, via user payments for Viber Out voice calling and the Viber graphical messaging "sticker store". The company was originally funded by individual investors, described by Marco as "friends and family".[25] They invested \$20 million in the company, which had 120 employees as of May 2013.[26]

On July 24, 2013, Viber's support system was defaced by the Syrian Electronic Army. According to Viber, no sensitive user information was accessed.[27] 





\subsubsection{Acquisition}



On February 13, 2014, Rakuten announced they had acquired Viber Media for \$900 million.[28][29] The sale of Viber earned the Shabtai family (Benny, his brother Gilad, and Gilad's son Ofer) some \$500 million from their 55.2\% stake in the company.[30][31] At that sale price, the founders each realized over 30 times return on their investments.[19]

Djamel Agaoua became Viber Media CEO in February 2017, replacing co-founder Marco who left in 2015.[32]

In July 2017 the corporate name of Viber Media was changed to Rakuten Viber and a new wordmark logo was introduced.[33] Its legal name remains Viber Media, S.à r.l. based in Luxembourg.

Viber has been the official "communication channel" of F.C. Barcelona since Rakuten partnered with the football club in 2017.[34][35] 





\section{Facebook Messenger}



Facebook Messenger (commonly known as Messenger)[8] is an American messaging app and platform developed by Facebook, Inc. Originally developed as Facebook Chat in 2008, the company revamped its messaging service in 2010, and subsequently released standalone iOS and Android apps in August 2011 and standalone Facebook Portal hardware for Messenger-based calling in Q4 2018. Later on, Facebook has launched a dedicated website interface (Messenger.com), and separated the messaging functionality from the main Facebook app, allowing users to use the web interface or download one of the standalone apps. In April 2020, Facebook officially released Messenger for Desktop, which is supported on Windows 10 and macOS and distributed on Microsoft Store and App Store respectively.

Users can send messages and exchange photos, videos, stickers, audio, and files, as well as react to other users' messages and interact with bots. The service also supports voice and video calling. The standalone apps support using multiple accounts, conversations with optional end-to-end encryption, and playing games. 










\subsection{History}



Following tests of a new instant messaging platform on Facebook in March 2008,[9][10] the feature, then-titled "Facebook Chat", was gradually released to users in April 2008.[11][12] Facebook revamped its messaging platform in November 2010,[13] and subsequently acquired group messaging service Beluga in March 2011,[14] which the company used to launch its standalone iOS and Android mobile apps on August 9, 2011.[15][16] Facebook later launched a BlackBerry version in October 2011.[17][18] An app for Windows Phone, though lacking features including voice messaging and chat heads, was released in March 2014.[19][20] In April 2014, Facebook announced that the messaging feature would be removed from the main Facebook app and users will be required to download the separate Messenger app.[21][22] An iPad-optimized version of the iOS app was released in July 2014.[23][24] In April 2015, Facebook launched a website interface for Messenger.[25][26] A Tizen app was released on July 13, 2015.[27] Facebook launched Messenger for Windows 10 in April 2016.[28] In October 2016, Facebook released Facebook Messenger Lite, a stripped-down version of Messenger with a reduced feature set. The app is aimed primarily at old Android phones and regions where high-speed Internet is not widely available. In April 2017, Facebook Messenger Lite was expanded to 132 more countries.[29][30] In May 2017, Facebook revamped the design for Messenger on Android and iOS, bringing a new home screen with tabs and categorization of content and interactive media, red dots indicating new activity, and relocated sections.[31][32]

Facebook announced a Messenger program for Windows 7 in a limited beta test in November 2011.[33][34] The following month, Israeli blog TechIT leaked a download link for the program, with Facebook subsequently confirming and officially releasing the program.[35][36] The program was eventually discontinued in March 2014.[37][38] A Firefox web browser add-on was released in December 2012,[39] but was also discontinued in March 2014.[40]

In December 2017, Facebook announced Messenger Kids, a new app aimed for persons under 13 years of age. The app comes with some differences compared to the standard version. In 2019, Facebook Messenger announced to be the 2nd most downloaded mobile app of the decade, from 2011 to 2019.[41] In December 2019, Facebook Messenger dropped support for users to sign in using only a mobile number, meaning that users must sign in to a Facebook account in order to use the service. [42]

In March 2020, Facebook started to ship its dedicated Messenger for macOS app through the Mac App Store. The app is currently live in regions include France, Australia, Mexico and Poland. [43] 

















\end{document}

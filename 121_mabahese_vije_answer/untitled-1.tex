\documentclass[12pt]{article}


\usepackage{tabularx}
\usepackage[table]{xcolor}
\usepackage{multirow}



\usepackage{amsmath, amssymb}
\usepackage{mathtools}

\usepackage[margin=1.1in,footskip=.25in]{geometry}

\usepackage{listings}
\usepackage{xcolor}
\usepackage{color}

\definecolor{dkgreen}{rgb}{0,0.6,0}
\definecolor{gray}{rgb}{0.5,0.5,0.5}
\definecolor{mauve}{rgb}{0.58,0,0.82}
 
\definecolor{codegreen}{rgb}{0,0.6,0}
\definecolor{codegray}{rgb}{0.5,0.5,0.5}
\definecolor{codepurple}{rgb}{0.58,0,0.82}
\definecolor{backcolour}{rgb}{0.95,0.95,0.92}

\lstdefinestyle{mystyle}{
    backgroundcolor=\color{backcolour},   
    commentstyle=\color{codegreen},
    %keywordstyle=\color{magenta},
    numberstyle=\tiny\color{codegray},
    stringstyle=\color{codepurple},
    basicstyle=\ttfamily\normalsize,
    breakatwhitespace=false,         
    breaklines=true,                 
    captionpos=b,                    
    keepspaces=true,                 
    %numbers=left,                    
    %numbersep=5pt,                  
    showspaces=false,                
    showstringspaces=false,
    showtabs=false,                  
    tabsize=3
}
\lstset{style=mystyle}



\usepackage[most]{tcolorbox}

\tcbset{
    frame code={}
    center title,
    left=10pt,
    right=10pt,
    top=10pt,
    bottom=10pt,
    colback=gray!5,
    colframe=gray,
    width=\dimexpr\textwidth\relax,
    enlarge left by=0mm,
    boxsep=5pt,
    arc=0pt,outer arc=0pt,
}


\usepackage{xepersian}
\settextfont[Scale=1]{Vazir}

\renewcommand{\baselinestretch}{1.3} 


\begin{document}

\noindent
1. برنامه ای بنویسید که نام خودتان را در صفحه ی خروجی چاپ کند .

\begin{latin}
\begin{lstlisting}[language=C++, caption=]
using System;

namespace MyFirstProgram {
	class Program {
		static void Main() {
			System.Console.WriteLine("Sajjad Abdollahi");
		}
	}
}
\end{lstlisting}
\end{latin}



\noindent
2 . برنامه ای بنویسید که 2 عدد صحیح را از ورودی دریافت کند و حاصل جمع آنها را محاسبه کرده و نمایش دهد .


\begin{latin}
\begin{lstlisting}[language=C++, caption=]
using System;

namespace MyFirstProgram {
	class Program {
		static void Main() {
			int a = Convert.ToInt32(Console.ReadLine());
			int b = Convert.ToInt32(Console.ReadLine());
			
			int sum = a + b ;
			
			Console.WriteLine("a + b : " + sum);
		}
	}
}
\end{lstlisting}
\end{latin}




\newpage


\noindent
3 . برنامه ای بنویسید که 2 عدد اعشاری را از ورودی دریافت کرده و حاصل جمع آنها را محاسبه کرده و نمایش دهد .



\begin{latin}
\begin{lstlisting}[language=C++, caption=]
using System;

namespace MyFirstProgram {
	class Program {
		static void Main() {
			double a = Convert.ToDouble(Console.ReadLine());
			double b = Convert.ToDouble(Console.ReadLine());
			
			double sum = a + b;
			
			Console.WriteLine("a + b : " + sum);
		}
	}
}
\end{lstlisting}
\end{latin}



\noindent
4 . برنامه ای بنویسید که 3 عدد دلخواه را از ورودی دریافت کرده و میانگین آنها را محاسبه کرده و نمایش دهد .



\begin{latin}
\begin{lstlisting}[language=C++, caption=,basicstyle=\ttfamily\small]
using System;

namespace MyFirstProgram {
	class Program {
		static void Main() {
			double a = Convert.ToDouble(Console.ReadLine());
			double b = Convert.ToDouble(Console.ReadLine());
			double c = Convert.ToDouble(Console.ReadLine());
			
			double sum = a + b + c;
			double avg = sum / 3 ;
			
			Console.WriteLine("avg : " + avg);
		}
	}
}
\end{lstlisting}
\end{latin}





\newpage

\noindent
5 . برنامه ای بنویسید که نام و نام خانوادگی شما را دریافت کرده و در یک خط به صورت نام-نام خانوادگی چاپ نماید .


\begin{latin}
\begin{lstlisting}[language=C++, caption=]
using System;

namespace MyFirstProgram {
	class Program {
		static void Main() {
			string fname, lname;
			
			fname = Console.ReadLine();
			lname = Console.ReadLine();
			
			Console.WriteLine(fname + "-" + lname);
		}
	}
}
\end{lstlisting}
\end{latin}




\noindent
6 . برنامه ای بنویسید که طول و عرض یک مستطیل را از ورودی دریافت کرده ، مساحت و محیط مستطیل را محاسبه کند .


\begin{latin}
\begin{lstlisting}[language=C++, caption=]
using System;

namespace MyFirstProgram {
	class Program {
		static void Main() {
			double width = Convert.ToDouble(Console.ReadLine());
			double height = Convert.ToDouble(Console.ReadLine());
			
			Console.WriteLine( "Area : " + width * height );
			Console.WriteLine( "P : " + 2 *(width + height) );
		}
	}
}
\end{lstlisting}
\end{latin}





\newpage

\noindent
7 . برنامه ای بنویسید که شعاع یک دایره را دریافت کرده و محیط و مساحت دایره را محاسبه کرده و نمایش دهد .




\begin{latin}
\begin{lstlisting}[language=C++, caption=]
using System;

namespace MyFirstProgram {
	class Program {
		static void Main() {
			const double pi = 3.14;
			double radius = Convert.ToDouble(Console.ReadLine());
			
			Console.WriteLine( "Area : " + pi * radius * radius );
			Console.WriteLine( "P : " + 2 * pi * radius );
		}
	}
}
\end{lstlisting}
\end{latin}




\noindent
8 . برنامه ای بنویسید که قاعده و ارتفاع یک مثلث را دریافت کرده و مساحت مثلث را محاسبه کرده و چاپ کند .




\begin{latin}
\begin{lstlisting}[language=C++, caption=]
using System;

namespace MyFirstProgram {
	class Program {
		static void Main() {
			double h = Convert.ToDouble(Console.ReadLine());
			double b = Convert.ToDouble(Console.ReadLine());
			
			Console.WriteLine( "Area : " + (h * b) / 2 );
		}
	}
}
\end{lstlisting}
\end{latin}






\newpage

\noindent
9 . برنامه ای بنویسید که حقوق ناخالص کارمندی را دریافت کرده و با استفاده از قوانین زیر حقوق خالص کارمند را محاسبه نماید و نمایش دهد .

\begin{itemize}
	\item بیمه $=$ حقوق خالص 
	$\times$ 7 درصد
	\item مالیات $=$ حقوق خالص 
	$\times$ 10 درصد
	\item حقوق خالص $=$ حقوق ناخالص $-$ بیمه $-$ مالیات
\end{itemize}




\begin{latin}
\begin{lstlisting}[language=C++, caption=]
using System;

namespace MyFirstProgram {
	class Program {
		static void Main() {
			double salary = Convert.ToDouble(Console.ReadLine());

			double ensurance , tax , exactSalary ;
			
			ensurance = (salary * 7)/100;
			tax = (salary * 10)/100;
			exactSalary = salary - ensurance - tax;
			
			Console.WriteLine("ensurance : " + ensurance);
			Console.WriteLine("tax : " + tax);
			Console.WriteLine("exactSalary : " + exactSalary);
		}
	}
}
\end{lstlisting}
\end{latin}







\newpage

\noindent
13 . برنامه ای بنویسید که
\lr{ATM}
 یک عدد صحیح دلخواه را به عنوان پول درخواستی از کاربر دریافت کند و سپس آن مبلغ را به پول های 1 و 5 و 10 و 50 هزار تومانی خرد کند .
 
 
 
 \begin{latin}
\begin{lstlisting}[language=C++, caption=]
using System;

namespace MyFirstProgram {
	class Program {
		static void Main() {
			int cash = Convert.ToInt32(Console.ReadLine());
			int fifty , ten , five, one ;
			
			fifty = cash / 50 ;
			cash %= 50 ;
			
			ten = cash / 10;
			cash %= 10;
			
			five = cash / 5;
			cash %= 5;
			
			one = cash;
			
			Console.WriteLine("fifty : " + fifty);
			Console.WriteLine("ten : " + ten);
			Console.WriteLine("five : " + five);
			Console.WriteLine("one : " + one);
		}
	}
}
\end{lstlisting}
\end{latin}
 
 
 
 
 
 
 \newpage
 
 \noindent
14 . برنامه ای بنویسید که بدون استفاده از دستور شرطی 
\lr{if}
،
یک عدد از ورودی دریافت کرده و مقدار قدر مطلق آن را نمایش دهد .



\begin{latin}
\begin{lstlisting}[language=C++, caption=]
using System;

namespace MyFirstProgram {
	class Program {
		static void Main() {
			int n = Convert.ToInt32(Console.ReadLine());
			
			n = Math.Abs(n);

			Console.WriteLine("Math.Abs(n) : " + n);
		}
	}
}
\end{lstlisting}
\end{latin}






\newpage

\noindent
15 . برنامه ای بنویسید که دو عدد دلخواه را از ورودی دریافت کرده و ماکزیمم و مینیمم این دو عدد را بدون استفاده از دستور شرطی
\lr{if}
محاسبه کرده و نمایش دهد .


\begin{align*}
Max(x,y) = \cfrac{|x+y| + |x-y|}{2} \qquad \qquad
Min(x,y) = \cfrac{|x+y| - |x-y|}{2}
\end{align*}




\begin{latin}
\begin{lstlisting}[language=C++, caption=]
using System;

namespace MyFirstProgram {
	class Program {
		static void Main() {
			int x = Convert.ToInt32(Console.ReadLine());
			int y = Convert.ToInt32(Console.ReadLine());
			
			int max = (Math.Abs(x+y) + Math.Abs(x-y))/2 ;
			int min = (Math.Abs(x+y) - Math.Abs(x-y))/2 ;
			

			Console.WriteLine("Max(x,y) : " + max);
			Console.WriteLine("Min(x,y) : " + min);
		}
	}
}
\end{lstlisting}
\end{latin}





\newpage

\noindent
16 . برنامه ای بنویسید که 2 عدد را از ورودی دریافت کرده و در دو متغیر قرار دهد ، سپس بدون استفاده از متغیر سوم ، مقدار این دو متغیر را با یکدیگر عوض کرده و نمایش دهد .





\begin{latin}
\begin{lstlisting}[language=C++, caption=]
using System;

namespace MyFirstProgram {
	class Program {
		static void Main() {
			int a = Convert.ToInt32(Console.ReadLine());
			int b = Convert.ToInt32(Console.ReadLine());
			
			Console.WriteLine("a : " + a);
			Console.WriteLine("b : " + b);
			
			a = a + b;
			b = a - b;
			a = a - b;

			Console.WriteLine("a : " + a);
			Console.WriteLine("b : " + b);
		}
	}
}
\end{lstlisting}
\end{latin}





\newpage

\noindent
17 . برنامه ای بنویسید که عددی را از ورودی دریافت کرده و مشخص کند عدد وارد شده زوج است یا فرد .






\begin{latin}
\begin{lstlisting}[language=C++, caption=]
using System;

namespace MyFirstProgram {
	class Program {
		static void Main() {
			int number = Convert.ToInt32(Console.ReadLine());
	
			if( number % 2 == 0 ) {
				Console.WriteLine("ZOJ");
			} else {
				Console.WriteLine("FARD");
			}
		}
	}
}
\end{lstlisting}
\end{latin}







\newpage

\noindent
18 . برنامه ای بنویسید که 2 عدد را از ورودی دریافت کرده و عدد بزرگتر را نمایش دهد .




\begin{latin}
\begin{lstlisting}[language=C++, caption=]
using System;

namespace MyFirstProgram {
	class Program {
		static void Main() {
			int a = Convert.ToInt32(Console.ReadLine());
			int b = Convert.ToInt32(Console.ReadLine());
	
			if( a > b ) {
				Console.WriteLine(a  + ">" +  b);
			} else if( a < b ) {
				Console.WriteLine(a  + "<" +  b);
			} else if( a == b ) {
				Console.WriteLine(a  + "==" +  b);
			}
		}
	}
}
\end{lstlisting}
\end{latin}






\newpage

\noindent
19 . برنامه ای بنویسید که نمرات 5 درس یک دانش آموز را دریافت کرده و معدل انش آموز را محاسبه نماید، اگر معدل دانش آموز کمتر از 12 بود ، دانش آموز مشروط ، اگر معدل بیشتر از 17 بود به عنوان دانش آموز ممتاز و در غیر این صورت به عنوان دانش آموز متوسط معرفی نماید .






\begin{latin}
\begin{lstlisting}[language=C++, caption=]
using System;

namespace MyFirstProgram {
	class Program {
		static void Main() {
			const int size = 5 ;
			double[] a = new double[size];
			
			double sum = 0 ;
			double avg ;
			
			for(int i=0;i<size;i++) {
				a[i] = Convert.ToInt32(Console.ReadLine());
				sum += a[i];
			}
			
			avg = sum / size;
	
			Console.WriteLine("Average : " + avg);
		}
	}
}
\end{lstlisting}
\end{latin}







\newpage

\noindent
20 . برنامه ای بنویسید که 4 عدد را از ورودی دریافت کند و بزرگترین عدد را نمایش دهد .





\begin{latin}
\begin{lstlisting}[language=C++, caption=]
using System;

namespace MyFirstProgram {
	class Program {
		static void Main() {
			const int size = 4 ;
			double[] a = new double[size];
			
			double max = -1;
			
			for(int i=0;i<size;i++) {
				a[i] = Convert.ToDouble(Console.ReadLine());
				if( i == 0 ) {
					max = a[i];
				} else {
					if(a[i] > max) {
						max = a[i];
					}
				}
			}
	
			Console.WriteLine("Max : " + max);
		}
	}
}
\end{lstlisting}
\end{latin}





\newpage

\noindent
21 . برنامه ای بنویسید که با استفاده از دستور 
\lr{switch}
یک عدد از 0 تا 7 را دریافت کرده و نام روز متناسب با آن را نمایش دهد . ( به طور مثال اگر عدد وارد شده 0 بود ، روز شنبه را نمایش دهد و اگر عدد وارد شده 6 بود ، روز جمعه را نمایش دهد  )





\begin{latin}
\begin{lstlisting}[language=C++, caption=,basicstyle=\ttfamily\small]
using System;
namespace MyFirstProgram {
	class Program {
		static void Main() {
			int day = Convert.ToInt32(Console.ReadLine());
			
			switch(day) {
				case 0:
					Console.WriteLine("Shanbe");
					break;
				case 1:
					Console.WriteLine("1 Shanbe");
					break;
				case 2:
					Console.WriteLine("2 Shanbe");
					break;
				case 3:
					Console.WriteLine("3 Shanbe");
					break;
				case 4:
					Console.WriteLine("4 Shanbe");
					break;
				case 5:
					Console.WriteLine("5 Shanbe");
					break;
				case 6:
					Console.WriteLine("Jome");
					break;
			}
		}
	}
}
\end{lstlisting}
\end{latin}







\newpage

\noindent
22 . برنامه ای بنویسید که 4 عدد را از ورودی دریافت کرده و دومین بزرگترین عدد را نمایش دهد .








\begin{latin}
\begin{lstlisting}[language=C++, caption=]
using System;

namespace MyFirstProgram {
	class Program {
		static void Main() {
			const int size = 4 ;
			double[] a = new double[size];
			
			double max = -1 ;
			double secondMax = -1;
			
			for(int i=0;i<size;i++) {
				a[i] = Convert.ToDouble(Console.ReadLine());
				if( i == 0 ) {
					secondMax = max = a[i];
				} else {
					if(a[i] > max) {
						secondMax = max;
						max = a[i];
					}
				}
			}
	
			Console.WriteLine("secondMax : " + secondMax);
		}
	}
}
\end{lstlisting}
\end{latin}








\newpage

\noindent
23 . برنامه ای بنویسید که حقوق ناخالص کارمندی را دریافت کرده و میزان مالیت را بر اساس قوانین زیر محاسبه کند .

\begin{itemize}
	\item اگر حقوق ناخالص کمتر از 1000 بود معاف از مالیات
	\item اگر حقوق ناخالص کمتر از 2000 بود نرخ مالیات
	 5\%
	\item اگر حقوق ناخالص کمتر از 3000 بود نرخ مالیات
	10\%
	\item اگر حقوق ناخالص بیشتر از 3000 بود نرخ مالیات
	15\%
\end{itemize}





\begin{latin}
\begin{lstlisting}[language=C++, caption=]
using System;

namespace MyFirstProgram {
	class Program {
		static void Main() {
			double salary = Convert.ToDouble(Console.ReadLine());
			double tax = 0 ;
			
			if(salary < 1000) {
				tax = 0 ;
			} else if(salary < 2000) {
				tax = (salary * 5)/200;
			} else if(salary < 3000) { 
				tax = (salary * 10)/200;
			} else if(salary >= 3000) {
				tax = (salary * 15)/200;
			}
			
			Console.WriteLine("tax : " + tax);
		}
	}
}
\end{lstlisting}
\end{latin}







\newpage

\noindent
24 . برنامه ای بنویسید که 3 عدد دلخواه را از ورودی دریافت کرده و مشخص کند که آیا این 3 عدد تشکیل یک مثلث خواهند داد یا خیر .



\begin{align*}
\text{شرط تشکیل مثلث : } \\ 
a+b>c \qquad \&\& \qquad b+c>a \qquad \&\& \qquad a+c>b
\end{align*}






\begin{latin}
\begin{lstlisting}[language=C++, caption=]
using System;

namespace MyFirstProgram {
	class Program {
		static void Main() {
			double a = Convert.ToDouble(Console.ReadLine());
			double b = Convert.ToDouble(Console.ReadLine());
			double c = Convert.ToDouble(Console.ReadLine());

			if( a+b>c && b+c>a && a+c>b ) {
				Console.WriteLine("YES");
			} else {
				Console.WriteLine("NO");
			}
		}
	}
}
\end{lstlisting}
\end{latin}








\newpage

\noindent
25 . برنامه ای بنویسید که 3 عدد دلخواه را به عنوان اضلاع یک مثلث دریافت کند و بررسی کند که این مثلث متساوی الساقین است یا خیر . ( مثلث متساوی الساقین دارای 2 ضلع برابر است )





\begin{latin}
\begin{lstlisting}[language=C++, caption=]
using System;

namespace MyFirstProgram {
	class Program {
		static void Main() {
			double a = Convert.ToDouble(Console.ReadLine());
			double b = Convert.ToDouble(Console.ReadLine());
			double c = Convert.ToDouble(Console.ReadLine());

			if( a == b || a == c || b == c ) {
				Console.WriteLine("YES");
			} else {
				Console.WriteLine("NO");
			}
		}
	}
}
\end{lstlisting}
\end{latin}






\newpage

\noindent
26 . برنامه ای بنویسید که 3 عدد دلخواه را به عنوان اضلاع یک مثلث دریافت کند و بررسی کند که این مثلث متساوی الضلاع است یا خیر . ( مثلث متساوی الاضلاع دارای 3 ضلع برابر است )







\begin{latin}
\begin{lstlisting}[language=C++, caption=]
using System;

namespace MyFirstProgram {
	class Program {
		static void Main() {
			double a = Convert.ToDouble(Console.ReadLine());
			double b = Convert.ToDouble(Console.ReadLine());
			double c = Convert.ToDouble(Console.ReadLine());

			if( a == b && b == c ) {
				Console.WriteLine("YES");
			} else {
				Console.WriteLine("NO");
			}
		}
	}
}
\end{lstlisting}
\end{latin}





\newpage

\noindent
27 . برنامه ای بنویسید که 3 عدد دلخواه را به عنوان اضلاع یک مثلث دریافت کند و بررسی کند که این مثلث قائم الزاویه است یا خیر . 

در مثلث قائم الزاویه یکی از روابط زیر برقرار است :


\begin{align*}
\colorbox{gray!10}{\parbox{90pt}{
$$a^{2} = b^{2} + c^{2}$$
}}
\qquad
\colorbox{gray!10}{\parbox{90pt}{
$$b^{2} = a^{2} + c^{2}$$
}}
\qquad
\colorbox{gray!10}{\parbox{90pt}{
$$c^{2} = a^{2} + b^{2} $$
}}
\end{align*}





\begin{latin}
\begin{lstlisting}[language=C++, caption=]
using System;

namespace MyFirstProgram {
	class Program {
		static void Main() {
			double a = Convert.ToDouble(Console.ReadLine());
			double b = Convert.ToDouble(Console.ReadLine());
			double c = Convert.ToDouble(Console.ReadLine());

			if( Math.Pow(a,2) == Math.Pow(b,2) + Math.Pow(c,2) || 
				Math.Pow(b,2) == Math.Pow(a,2) + Math.Pow(c,2) ||
				Math.Pow(c,2) == Math.Pow(a,2) + Math.Pow(b,2) 
			) {
				Console.WriteLine("YES");
			} else {
				Console.WriteLine("NO");
			}
		}
	}
}
\end{lstlisting}
\end{latin}






\newpage

\noindent
29 . برنامه ای بنویسید که اعداد 1 تا 100 را چاپ کند .



\begin{latin}
\begin{lstlisting}[language=C++, caption=]
using System;

namespace MyFirstProgram {
	class Program {
		static void Main() {
			for(int i=1;i<=100;i++) {
				Console.WriteLine(i);
			}
		}
	}
}
\end{lstlisting}
\end{latin}







\noindent
30 . برنامه ای بنویسید که حاصل جمع اعداد 1 تا 100 را چاپ کند .





\begin{latin}
\begin{lstlisting}[language=C++, caption=]
using System;

namespace MyFirstProgram {
	class Program {
		static void Main() {
			int sum = 0;
			
			for(int i=1;i<=100;i++) {
				sum += i;
			}
			
			Console.WriteLine("Sum : " + sum);
		}
	}
}
\end{lstlisting}
\end{latin}







\newpage

\noindent
31 . برنامه ای بنویسید که اعداد زوج بین 1 تا 100 را چاپ کند .





\begin{latin}
\begin{lstlisting}[language=C++, caption=]
using System;

namespace MyFirstProgram {
	class Program {
		static void Main() {
			for(int i=1;i<=100;i++) {
				if(i % 2 == 0) {
					Console.WriteLine(i);
				}
			}
		}
	}
}
\end{lstlisting}
\end{latin}








\noindent
32 . برنامه ای بنویسید که حاصل جمع اعداد فرد بین 1 تا 100 را چاپ کند .






\begin{latin}
\begin{lstlisting}[language=C++, caption=]
using System;

namespace MyFirstProgram {
	class Program {
		static void Main() {
			int sum = 0;
			
			for(int i=1;i<=100;i++) {
				if(i % 2 != 0) {
					sum += i;
				}
			}

			Console.WriteLine("Sum : " + sum);
		}
	}
}
\end{lstlisting}
\end{latin}






\noindent
33 . برنامه ای بنویسید که یک عدد صحیح را از ورودی دریافت کرده و اعداد کوچکتر از آن را چاپ نماید .







\begin{latin}
\begin{lstlisting}[language=C++, caption=]
using System;

namespace MyFirstProgram {
	class Program {
		static void Main() {
			int n = Convert.ToInt32(Console.ReadLine());
			
			for(int i=n;i>0;i--) {
				Console.WriteLine(i);
			}
		}
	}
}
\end{lstlisting}
\end{latin}







\noindent
34 . برنامه ای بنویسید که یک عدد را از ورودی دریافت کرده و مقسوم علیه های آن را چاپ کند .






\begin{latin}
\begin{lstlisting}[language=C++, caption=]
using System;

namespace MyFirstProgram {
	class Program {
		static void Main() {
			int n = Convert.ToInt32(Console.ReadLine());
			
			for(int i=n;i>0;i--) {
				if(n % i == 0) {
					Console.WriteLine(i);
				}
			}
		}
	}
}
\end{lstlisting}
\end{latin}





\newpage

\noindent
35 . برنامه ای بنویسید که یک عدد را از ورودی دریافت کرده و مجموعه مقسوم علیه های آن را چاپ کند .




\begin{latin}
\begin{lstlisting}[language=C++, caption=]
using System;

namespace MyFirstProgram {
	class Program {
		static void Main() {
			int n = Convert.ToInt32(Console.ReadLine());
			int sum = 0;
			
			for(int i=n;i>0;i--) {
				if(n % i == 0) {
					sum += i;
				}
			}
			
			Console.WriteLine("Sum : " + sum);
		}
	}
}
\end{lstlisting}
\end{latin}





\newpage

\noindent
36 . برنامه ای بنویسید که یک عدد را از ورودی دریافت کرده و تعداد مقسوم علیه های آن را چاپ کند .







\begin{latin}
\begin{lstlisting}[language=C++, caption=]
using System;

namespace MyFirstProgram {
	class Program {
		static void Main() {
			int n = Convert.ToInt32(Console.ReadLine());
			int counter = 0;
			
			for(int i=n;i>0;i--) {
				if(n % i == 0) {
					counter++;
				}
			}
			
			Console.WriteLine("counter : " + counter);
		}
	}
}
\end{lstlisting}
\end{latin}









\newpage


\noindent
37 . برنامه ای بنویسید که یک عدد را از ورودی دریافت کرده و مجموع مقسوم علیه های فرد آن را چاپ کند .





\begin{latin}
\begin{lstlisting}[language=C++, caption=]
using System;

namespace MyFirstProgram {
	class Program {
		static void Main() {
			int n = Convert.ToInt32(Console.ReadLine());
			int sum = 0;
			
			for(int i=n;i>0;i--) {
				if(n % i == 0) {
					if(i % 2 != 0) {
						sum += i;
					}
				}
			}
			
			Console.WriteLine("Sum : " + sum);
		}
	}
}
\end{lstlisting}
\end{latin}








\newpage

\noindent
38 . برنامه ای بنویسید که یک عدد را از ورودی دریافت کرده و تعداد مقسوم علیه های زوج آن را چاپ کند .






\begin{latin}
\begin{lstlisting}[language=C++, caption=]
using System;

namespace MyFirstProgram {
	class Program {
		static void Main() {
			int n = Convert.ToInt32(Console.ReadLine());
			int counter = 0;
			
			for(int i=n;i>0;i--) {
				if(n % i == 0) {
					if(i%2 == 0) {
						counter++;
					}
				}
			}
			
			Console.WriteLine("counter : " + counter);
		}
	}
}
\end{lstlisting}
\end{latin}







\newpage

\noindent
39 . برنامه ای بنویسید که یک عدد را از ورودی دریافت کرده و مشخص کند که عدد وارد شده عدد اول است یا خیر . ( عددی اول است که به غیر از 1 و خودش مقسوم علیه دیگری نداشته باشد ) 







\begin{latin}
\begin{lstlisting}[language=C++, caption=]
using System;

namespace MyFirstProgram {
	class Program {
		static void Main() {
			int n = Convert.ToInt32(Console.ReadLine());
			
			bool aval = true;
			
			for(int i=n/2;i>1;i--) {
				if(n % i == 0) {
					aval = false;
					break;
				}
			}
			
			if(aval == true) {
				Console.WriteLine("AVAL");
			} else if(aval == false) {
				Console.WriteLine("NOT AVAL");
			}
		}
	}
}
\end{lstlisting}
\end{latin}









\newpage

\noindent
40 . برنامه ای بنویسید که یک عدد را از ورودی دریافت کرده و مشخص کند که عدد وارد شده عدد کامل است یا خیر . ( عددی کامل است که مجموع مقسوم علیه های به غیر از خودش با خود عدد برابر باشد )




\begin{latin}
\begin{lstlisting}[language=C++, caption=]
using System;

namespace MyFirstProgram {
	class Program {
		static void Main() {
			int n = Convert.ToInt32(Console.ReadLine());
			
			int sum = 0;
			
			for(int i=n/2;i>=1;i--) {
				if(n % i == 0) {
					sum += i;
				}
			}
			
			if(sum == n) {
				Console.WriteLine("Kamel");
			} else {
				Console.WriteLine("NOT Kamel");
			}
		}
	}
}
\end{lstlisting}
\end{latin}







\newpage

\noindent
41 . برنامه ای بنویسید که با استفاده از حلقه ی تکرار 10 عدد را از ورودی دریافت کند و میانگین آنها را نمایش دهد .





\begin{latin}
\begin{lstlisting}[language=C++, caption=]
using System;

namespace MyFirstProgram {
	class Program {
		static void Main() {
			const int size = 10 ;
			double[] a = new double[size];
			
			double sum = 0 ;
			double avg ;
			
			for(int i=0;i<size;i++) {
				a[i] = Convert.ToDouble(Console.ReadLine());
				sum += a[i];
			}
			
			avg = sum / size;
	
			Console.WriteLine("Average : " + avg);
		}
	}
}
\end{lstlisting}
\end{latin}








\newpage

\noindent
42 . برنامه ای بنویسید که با استفاده از حلقه ی تکرار 10 عدد را از ورودی دریافت کند و بزرگترین آنها را نمایش دهد .






\begin{latin}
\begin{lstlisting}[language=C++, caption=]
using System;

namespace MyFirstProgram {
	class Program {
		static void Main() {
			const int size = 10 ;
			double[] a = new double[size];
			
			double max = -1;
			
			for(int i=0;i<size;i++) {
				a[i] = Convert.ToDouble(Console.ReadLine());
				if( i == 0 ) {
					max = a[i];
				} else {
					if(a[i] > max) {
						max = a[i];
					}
				}
			}
	
			Console.WriteLine("Max : " + max);
		}
	}
}
\end{lstlisting}
\end{latin}







\newpage

\noindent
43 . برنامه ای بنویسید که با استفاده از حلقه ی تکرار 10 عدد را از ورودی دریافت کند و دومین بزرگترین آنها را نمایش دهد .







\begin{latin}
\begin{lstlisting}[language=C++, caption=]
using System;

namespace MyFirstProgram {
	class Program {
		static void Main() {
			const int size = 10 ;
			double[] a = new double[size];
			
			double max = -1; 
			double secondMax = -1;
			
			for(int i=0;i<size;i++) {
				a[i] = Convert.ToDouble(Console.ReadLine());
				if( i == 0 ) {
					secondMax = max = a[i];
				} else {
					if(a[i] > max) {
						secondMax = max;
						max = a[i];
					}
				}
			}
	
			Console.WriteLine("secondMax : " + secondMax);
		}
	}
}
\end{lstlisting}
\end{latin}







\newpage

\noindent
44 . برنامه ای بنویسید که عددی را از ورودی دریافت کرده و تعداد ارقام آن را نمایش دهد . ( برای مثال عدد 123 ، 3 رقم دارد )






\begin{latin}
\begin{lstlisting}[language=C++, caption=]
using System;

namespace MyFirstProgram {
	class Program {
		static void Main() {
			int n = Convert.ToInt32(Console.ReadLine());
			int counter = 0;
			
			while( n != 0 ) {
				counter++;
				n /= 10;
			}
			
			Console.WriteLine("counter : " + counter);
		}
	}
}
\end{lstlisting}
\end{latin}








\newpage

\noindent
45 . برنامه ای بنویسید که یک عدد صحیح را از ورودی دریافت کرده و عدد وارد شده را مقلوب نماید . ( برای مثال مقلوب عدد 123 ، عدد 321 است )






\begin{latin}
\begin{lstlisting}[language=C++, caption=]
using System;

namespace MyFirstProgram {
	class Program {
		static void Main() {
			int n = Convert.ToInt32(Console.ReadLine());
			int digit;
			
			while( n != 0 ) {
				digit = n % 10;
				Console.Write(digit);
				n /= 10;
			}
		}
	}
}
\end{lstlisting}
\end{latin}







\newpage

\noindent
46 . برنامه ای بنویسید که یک عدد صحیح را دریافت کرده و مجموع اعداد زوج آن را نمایش دهد . ( برای مثال مجموع اعداد زوج 249 ، عدد 6 می باشد )




\begin{latin}
\begin{lstlisting}[language=C++, caption=]
using System;

namespace MyFirstProgram {
	class Program {
		static void Main() {
			int n = Convert.ToInt32(Console.ReadLine());
			int digit;
			int sum = 0;
			
			while( n != 0 ) {
				digit = n % 10;
				if(digit % 2 == 0) {
					sum += digit;
				}
				n /= 10;
			}
			
			Console.WriteLine("Sum : " + sum);
		}
	}
}
\end{lstlisting}
\end{latin}






\newpage

\noindent
47 . برنامه ای بنویسید که یم عدد صحیح را دریافت کرده و تعداد اعداد فرد آن را نمایش دهد . ( برای مثال تعداد اعداد فرد عدد 163 ، عدد 2 می باشد )




\begin{latin}
\begin{lstlisting}[language=C++, caption=]
using System;

namespace MyFirstProgram {
	class Program {
		static void Main() {
			int n = Convert.ToInt32(Console.ReadLine());
			int digit;
			int counter = 0;
			
			while( n != 0 ) {
				digit = n % 10;
				if(digit % 2 != 0) {
					counter++;
				}
				n /= 10;
			}
			
			Console.WriteLine("counter : " + counter);
		}
	}
}
\end{lstlisting}
\end{latin}









\newpage

\noindent
49 . برنامه ای بنویسید که 50 جمله ی اول سری فیبوناتچی را چاپ کند .

$$
0 , 1 , 1 , 2 , 3 , 5 , 8 , 13 , 21 , 34 , 55 , \dots
$$


\begin{align*}
F(n) = 
\begin{cases}
0 & n = 0 \\
1 & n = 1 \\
F(n-1) + F(n-2) & n > 1 
\end{cases}
\end{align*}





\begin{latin}
\begin{lstlisting}[language=C++, caption=]
using System;

namespace MyFirstProgram {
	class Program {
		static void Main() {
			long first = 0;
			long second = 1;
			long temp ;
			
			Console.WriteLine(first);
			Console.WriteLine(second);
			

			for(int i=1;i<=50;i++) {
				temp = second;
				second += first ;
				first = temp;
				
				Console.WriteLine(second);
			}
		}
	}
}
\end{lstlisting}
\end{latin}







\newpage

\noindent
51 . برنامه ای بنویسید که یک جدول ضرب 10 در 10 را چاپ نماید .






\begin{latin}
\begin{lstlisting}[language=C++, caption=]
using System;

namespace MyFirstProgram {
	class Program {
		static void Main() {
			for(int i=1;i<=10;i++) {
				for(int j=1;j<=10;j++) {
					Console.Write("{0,-5}",i * j);
				}
				Console.WriteLine();
			}
		}
	}
}
\end{lstlisting}
\end{latin}














\newpage


\noindent
59 . برنامه ای بنویسید که یک عدد را از ورودی دریافت کرده و فاکتوریل آن را محاسبه نماید .

\begin{tcolorbox}
$$
n ! = 1 \times 2 \times 3 \times \dots \times (n-1) \times (n-2) \times n
$$
\end{tcolorbox}





\begin{latin}
\begin{lstlisting}[language=C++, caption=]
using System;

namespace MyFirstProgram {
	class Program {
		static void Main() {
			int n = Convert.ToInt32(Console.ReadLine());
			int fact = 1;
			
			for(int i=1;i<=n;i++) {
				fact *= i;
			}
			
			Console.WriteLine("factorial(" + n + ") : " + fact );
		}
	}
}
\end{lstlisting}
\end{latin}





\end{document}
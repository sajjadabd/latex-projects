\documentclass{article}

\usepackage{amsmath, amssymb}
\usepackage[margin=1.1in,footskip=.25in]{geometry}

\usepackage{mathtools}

\usepackage{tabularx}
\usepackage[table]{xcolor}
\usepackage{multirow}


\usepackage{fancyhdr}

\fancypagestyle{myplain}
{
  \fancyhf{}
  \renewcommand\headrulewidth{0pt}
  \renewcommand\footrulewidth{0pt}
  \fancyfoot[C]{\thepage}
}
\fancyhf{}
\fancyhf{}
\fancyhead[CO]{\nouppercase\leftmark}
\fancyhead[CE]{\hdrtitle}
\fancyhead[LE,RO]{\thepage}

\pagestyle{fancy}

\renewcommand\sectionmark[1]{\markboth{#1}{}}%don't move this



\usepackage{hyperref}
\hypersetup{
    colorlinks=true, %set true if you want colored links
    linktoc=all,     %set to all if you want both sections and subsections linked
    linkcolor=black,  %choose some color if you want links to stand out
}

\usepackage{xepersian}
\settextfont[Scale=1]{Vazir}

\renewcommand{\baselinestretch}{1.3} 


\begin{document}

\section{انواع وابستگی ها}

\begin{itemize}
	\item وابستگی تابعی
	\item وابستگی تابعی کامل
	\item وابستگی با واسطه
	\item وابستگی تابعی چند مقداری
	\item وابستگی پیوندی
\end{itemize}


\begin{latin}
	\begin{LTR}
		\begin{itemize}
			\item Functional Dependency
			\item Full-Functional Dependency
			\item Transitive Dependency
			\item Multi-Valued Dependency
			\item Join Dependency
		\end{itemize}
	\end{LTR}
\end{latin}




\section{وابستگی تابعی}

رابطه ی 
$$R(A,B,\dots)$$
را در نظر بگیرید، می گوییم B با A وابستگی تابعی دارد و نشان می دهیم، 
$$A \to B$$
اگر و فقط اگر در هر مقدار ممکن از متغیر رابطه R ، به هر مقدار A فقط یک مقدار B متناظر باشد .



\begin{latin}
\begin{center}
  \bgroup
  \def\arraystretch{1.5}%
  \begin{tabular}{  c | c | c }
    A & B & C \\
    \hline
    3 & 2 & 1 \\ 
    8 & 7 & 2 \\ 
    9 & 7 & 3 \\ 
    3 & 2 & 4 \\
  \end{tabular}
  \egroup
\end{center}
\end{latin}


\newpage

\subsection{مثال}

A با B وابستگی تابعی دارد .

$$
B \to A
$$

\begin{latin}
\begin{center}
  \bgroup
  \def\arraystretch{1.5}%
  \begin{tabular}{  c | c }
    A & B  \\
    \hline
    1 & 4  \\ 
    1 & 5  \\ 
    3 & 7  \\ 
  \end{tabular}
  \egroup
\end{center}
\end{latin}



\subsection{نکته}

اگر A کلید اصلی رابطه ی 
$R(A,B,\dots)$
 باشد ، در این صورت هر صفت خاصه ی دیگر با A دارای وابستگی تابعی است .
\begin{align*}
A &\to B \\
A &\to C \\
&\vdots
\end{align*}



\subsection{مثال}

تعیین وابستگی های تابعی در رابطه ی 
$R(A,B,C)$ :


\begin{latin}
\begin{center}
  \bgroup
  \def\arraystretch{1.5}%
  \begin{tabular}{  c | c | c }
    A & B & C \\
    \hline
    1 & 5 & 3 \\ 
    1 & 7 & 8 \\ 
    2 & 7 & 9 \\ 
    2 & 5 & 3 \\
  \end{tabular}
  \egroup
\end{center}
\end{latin}





\begin{align*}
(A,B) &\to C \\
C &\to B
\end{align*}




\newpage

\section{وابستگی تابعی کامل}

اگر X و Y دو زیر مجموعه از مجموعه ی رابطه ی R باشند ، می گوییم Y با X وابستگی تابعی کامل دارد و نشان می دهیم :
$$
X \Rightarrow Y
$$

اگر و فقط اگر Y با X وابستگی تابعی داسته باشد ولی با هیچ زیر مجموعه از X وابستگی تابعی نداشته باشد .


* بدیهی است اگر X صفت ساده باشد ، وابستگی کامل خواهد بود . 


\subsection{مثال}

در مثال زیر وابستگی تابعی $ C $ به $ (A, B) $ کامل نمی باشد .

\begin{latin}
\begin{center}
  \bgroup
  \def\arraystretch{1.5}%
  \begin{tabular}{  c | c | c }
    A & B & C \\
    \hline
    \hline
    \hline
  \end{tabular}
  \egroup
\end{center}
\end{latin}

\begin{align*}
(A,B) &\to C \\
A &\to C
\end{align*}




\section{وابستگی باواسطه}

رابطه ی $R(A,B,C)$  مفروض است، اگر داشته باشیم :

\begin{align*}
A &\to B \\
B &\to C
\end{align*}

می گوییم C با A وابستگی با واسطه دارد .

برای از بین بردن این وابستگی رابطه را به صورت زیر تجزیه می کنیم :

\begin{align*}
&R_{1}(A, B) \\
&R_{2}(B, C)
\end{align*}



\newpage

\section{قواعد استنتاج آرمسترانگ}


با فرض رابطه ی $R(A,B,C,D)$ قواعد زیر برقرارند :

\subsection{انعکاسی}

\begin{latin}
	\begin{center}
		if $B \subseteq A$ then $A \to B$
	\end{center}
\end{latin}


\subsection{تعدی ( تراگذاری )}

\[
\begin{rcases*}
A \to B \\
B \to C
\end{rcases*} \Rightarrow A \to C
\]


\subsection{افزایش}

\begin{latin}
	\begin{center}
		if $A \to B$ then $AC \to BC$
	\end{center}
\end{latin}

\subsection{تجزیه}

\[
 A \to BC \Rightarrow
\begin{cases*}
A \to B \\
A \to C
\end{cases*} 
\]

\subsection{اجتماع}

\[
\begin{rcases*}
A \to B \\
A \to C
\end{rcases*} \Rightarrow A \to BC
\]


\subsection{ترکیب}

\[
\begin{rcases*}
A \to B \\
C \to D
\end{rcases*} \Rightarrow AC \to BD
\]

\subsection{شبه تعدی}


\[
\begin{rcases*}
A \to B \\
CB \to D
\end{rcases*} \Rightarrow AC \to D
\]

\[
\begin{rcases*}
AB \to C \\
A \to B
\end{rcases*} \Rightarrow A \to C
\]



\section{تعیین مجموعه حداقل وابستگی ها}


\begin{align*}
R = \{ &S\#, city , status \} \\
F = \{ &S\# \to city ,  \\
&city \to status , \\
&S\# \to status  \} 
\end{align*}


جواب : وابستگی سوم از 2 وابستگی اول قابل استنتاج است و می توان آن را ذکر نکرد .

\begin{align*}
F_{optimum} &= \{ S\# \to city ,  city \to status \} 
\end{align*}


\section{تعیین مجموعه حداقل وابستگی ها}


\begin{align*}
R = \{ &u , v , w , y , z \} \\
F = \{ &u \to xy , \\
&x \to y , \\
&xy \to zv  \} 
\end{align*}



حل : 

\begin{align*}
&u \to xy \Rightarrow \begin{cases*}  u \to x \\ u \to y \end{cases*}  \\
&\begin{rcases*} u \to xy \\  xy \to zv \end{rcases*} \Rightarrow u \to zv \Rightarrow \begin{cases*} 
u \to z \\ u \to v \end{cases*} \\
&\begin{rcases*} xy \to zv \\ x \to y \end{rcases*} \Rightarrow x \to zv \Rightarrow \begin{cases*} x \to z \\ x \to v \end{cases*} \\
\end{align*}

بنابراین مجموعه ی وابستگی ها برابر است با :

\begin{align*}
F = \{ &u \to x , \\ 
&u \to y , \\
&u \to z , \\
&u \to v , \\
&x \to z , \\
&x \to v , \\
&x \to y \} 
\end{align*}


که 
$u \to z$
و
$u \to v$
و
$u \to y$
اضافی هستند چون می توان آنها را به دست آورد . پس F کمینه برابر است با :

\begin{align*}
F_{optimum} = \{ &u \to x , \\
&x \to z , \\
&x \to v , \\
&x \to y \} 
\end{align*}


\section{کلید کاندید}
کلید کاندید صفتی است که از طریق آن به همه ی صفت های دیگر می توان رسید .


\subsection{تعیین کلید کاندید }


\begin{align*}
R = \{ &S , T , U , V , W \}  \\
F = \{ &S \to T , 
V \to SW , 
T \to U \}
\end{align*}

حل :


\begin{align*}
&V \to SW \Rightarrow  \begin{cases*}  V \to S \\ V \to W \end{cases*} \\ 
& \begin{rcases*} V \to S \\ S \to T \end{rcases*} \Rightarrow V \to T \\
&\begin{rcases*} V \to T \\ T \to U \end{rcases*} \Rightarrow V \to U \\
\end{align*}


همه ی صفت ها با $V$  در وابستگی تابعی هستند بنابراین $V$  کلید کاندید است .


\subsection{تعیین کلید کاندید }


\begin{align*}
R = \{ &A , B , C , D , E , F \}  \\
F = \{ &A \to BE , \\
&C \to F , \\
&B \to C , \\
&B \to E , \\
&DB \to E \}
\end{align*}

حل :

\begin{align*}
&A \to BE \Rightarrow  \begin{cases*}  A \to B \\ A \to E  \end{cases*}  \\
&\begin{rcases*}B \to C \\ B \to F \end{rcases*} \Rightarrow B \to F \\
&\begin{rcases*}A \to B \\ B \to F \end{rcases*} \Rightarrow A \to F \\
&\begin{rcases*}A \to B \\ B \to C \end{rcases*} \Rightarrow A \to C \\
\end{align*}


همه ی صفت ها با A در وابستگی تابعی هستند بنابراین A کلید کاندید است .



\newpage

\subsection{تعیین کلید کاندید }

\begin{align*}
R = \{ &A , B , C , D , E , F , G \}  \\
F = \{ &ABD \to EG , \\
&C \to DG , \\
&E \to FG , \\
&AB \to C , \\
&G \to F \}
\end{align*}


حل :
\begin{align*}
&C \to DG \Rightarrow  \begin{cases*} C \to D \\ C \to G  \end{cases*} \\
&E \to FG \Rightarrow  \begin{cases*} E \to F \\ E \to G  \end{cases*} \\
& \begin{rcases*}C \to G \\ G \to F  \end{rcases*} \Rightarrow C \to F \\
& \begin{rcases*}AB \to C \\ C \to D  \end{rcases*} \Rightarrow AB \to D \\
& \begin{rcases*}ABD \to EG \\ AB \to D  \end{rcases*} \Rightarrow AB \to EG \\
&AB \to EG \Rightarrow  \begin{cases*} AB \to E \\ AB \to G  \end{cases*} \\
& \begin{rcases*}AB \to E \\ E \to F  \end{rcases*} \Rightarrow AB \to F \\
&AB \to A \\
&AB \to B \\
\end{align*}

همه ی صفت ها با AB در وابستگی تابعی هستند بنابراین AB کلید کاندید است .




\subsection{تعیین کلید کاندید }

\begin{align*}
R = \{ &A , B , C , D , E , F , G \}  \\
F = \{ &AF \to BE , \\
&FC \to DE , \\
&F \to CD , \\
&D \to E , \\
&C \to A \}
\end{align*}

حل :

\begin{align*}
&AF \to BE \Rightarrow \begin{cases*} AF \to B \\ AF \to E \end{cases*} \\
&FC \to DE \Rightarrow \begin{cases*} FC \to D \\ FC \to E \end{cases*} \\
&F \to CD \Rightarrow \begin{cases*} F \to C \\ F \to D \end{cases*} \\
&D \to E \\
&C \to A 
\end{align*}

\begin{align*}
&\begin{rcases*}FC \to E \\ F \to C \end{rcases*} \Rightarrow F \to E \\
&\begin{rcases*}F \to C \\ C \to A \end{rcases*} \Rightarrow F \to A \\
&\begin{rcases*}AF \to B , F \to A  \end{rcases*}\Rightarrow F \to B \\
\end{align*}



F
همه ی صفت ها به غیر از G را نتیجه می دهد ، بنابراین 
$ (F , G) $
کلید کاندید است .








\end{document}
\documentclass[12pt]{article}


\usepackage{tabularx}
\usepackage[table]{xcolor}
\usepackage{multirow}



\usepackage{amsmath, amssymb}
\usepackage{mathtools}

\usepackage[margin=1in,footskip=.25in]{geometry}

\usepackage{listings}
\usepackage{xcolor}
\usepackage{color}

\definecolor{dkgreen}{rgb}{0,0.6,0}
\definecolor{gray}{rgb}{0.5,0.5,0.5}
\definecolor{mauve}{rgb}{0.58,0,0.82}
 
\definecolor{codegreen}{rgb}{0,0.6,0}
\definecolor{codegray}{rgb}{0.5,0.5,0.5}
\definecolor{codepurple}{rgb}{0.58,0,0.82}
\definecolor{backcolour}{rgb}{0.95,0.95,0.92}

\lstdefinestyle{mystyle}{
    backgroundcolor=\color{backcolour},  
    commentstyle=\color{codegreen},
    %keywordstyle=\color{magenta},
    numberstyle=\tiny\color{codegray},
    stringstyle=\color{codepurple},
    basicstyle=\ttfamily\normalsize,
    breakatwhitespace=false,         
    breaklines=true,                 
    captionpos=b,                    
    keepspaces=true,                                            
    showspaces=false,                
    showstringspaces=false,
    showtabs=false,                  
    tabsize=3
}
\lstset{style=mystyle}



\usepackage[most]{tcolorbox}

\tcbset{
    frame code={}
    center title,
    left=10pt,
    right=10pt,
    top=10pt,
    bottom=10pt,
    colback=gray!5,
    colframe=gray,
    width=\dimexpr\textwidth\relax,
    enlarge left by=0mm,
    boxsep=5pt,
    arc=0pt,outer arc=0pt,
}


\usepackage{xepersian}
\settextfont[Scale=1]{Vazir}

\renewcommand{\baselinestretch}{1.3} 


\begin{document}

\noindent
1. برنامه ای بنویسید که نام خودتان را در صفحه ی خروجی چاپ کند .


\begin{latin}
\begin{lstlisting}[language=C++, caption=]
#include<iostream>
using namespace std;

int main() {
	cout << "Sajjad Adollahi";
	return 0;
}
\end{lstlisting}
\end{latin}




\noindent
2 . برنامه ای بنویسید که 2 عدد صحیح را از ورودی دریافت کند و حاصل جمع آنها را محاسبه کرده و نمایش دهد .



\begin{latin}
\begin{lstlisting}[language=C++, caption=]
#include<iostream>
using namespace std;

int main() {
	float number1,number2;
	cin >> number1;
	cin >> number2;
	
	float sum = number1 + number2;
	cout << sum ;
	
	return 0;
}
\end{lstlisting}
\end{latin}



\newpage


\noindent
3 . برنامه ای بنویسید که 2 عدد اعشاری را از ورودی دریافت کرده و حاصل جمع آنها را محاسبه کرده و نمایش دهد .




\begin{latin}
\begin{lstlisting}[language=C++, caption=]
#include<iostream>
using namespace std;

int main() {
	float number1,number2;
	cin >> number1;
	cin >> number2;
	
	float sum = number1 + number2;
	cout << sum ;
	
	return 0;
}
\end{lstlisting}
\end{latin}



\noindent
4 . برنامه ای بنویسید که 3 عدد دلخواه را از ورودی دریافت کرده و میانگین آنها را محاسبه کرده و نمایش دهد .



\begin{latin}
\begin{lstlisting}[language=C++, caption=]
#include<iostream>
using namespace std;

int main() {
	float a,b,c;
	cin >> a;
	cin >> b;
	cin >> c;
	
	float avg = (a+b+c)/3;
	cout << avg ;
	
	return 0;
}
\end{lstlisting}
\end{latin}



\newpage

\noindent
5 . برنامه ای بنویسید که نام و نام خانوادگی شما را دریافت کرده و در یک خط به صورت نام-نام خانوادگی چاپ نماید .



\begin{latin}
\begin{lstlisting}[language=C++, caption=]
#include<iostream>
using namespace std;

int main() {
	char fname[20];
	char lname[20];
	
	cin >> fname ;
	cin >> lname ;
	
	cout << fname << "-" << lname ;
	
	return 0;
}
\end{lstlisting}
\end{latin}



\noindent
6 . برنامه ای بنویسید که طول و عرض یک مستطیل را از ورودی دریافت کرده ، مساحت و محیط مستطیل را محاسبه کند .


\begin{latin}
\begin{lstlisting}[language=C++, caption=]
#include<iostream>
using namespace std;

int main() {
	float width ,height;
	cin >> width;
	cin >> height;
	
	cout << "Area : " ;
	cout << width * height << endl;
		
	cout << "Perimeter : ";
	cout << 2 * (width + height);
	
	return 0;
}
\end{lstlisting}
\end{latin}



\noindent
7 . برنامه ای بنویسید که شعاع یک دایره را دریافت کرده و محیط و مساحت دایره را محاسبه کرده و نمایش دهد .


\begin{latin}
\begin{lstlisting}[language=C++, caption=]
#include<iostream>
using namespace std;

int main() {
	const float pi = 3.14;
	float radius;
	cin >> radius;
	
	cout << "Area : ";
	cout << pi * radius * radius << endl;
	
	cout << "Perimeter : ";
	cout << 2 * pi * radius;
	
	return 0;
}
\end{lstlisting}
\end{latin}



\noindent
8 . برنامه ای بنویسید که قاعده و ارتفاع یک مثلث را دریافت کرده و مساحت مثلث را محاسبه کرده و چاپ کند .


\begin{latin}
\begin{lstlisting}[language=C++, caption=]
#include<iostream>
using namespace std;

int main() {
	float b;
	float h;
	cin >> b >> h;
	
	cout << "Area : ";
	cout << ( b * h ) / 2;
	
	return 0;
}
\end{lstlisting}
\end{latin}



\noindent
9 . برنامه ای بنویسید که حقوق ناخالص کارمندی را دریافت کرده و با استفاده از قوانین زیر حقوق خالص کارمند را محاسبه نماید و نمایش دهد .

\begin{itemize}
	\item بیمه $=$ حقوق خالص 
	$\times$ 7 درصد
	\item مالیات $=$ حقوق خالص 
	$\times$ 10 درصد
	\item حقوق خالص $=$ حقوق ناخالص $-$ بیمه $-$ مالیات
\end{itemize}





\begin{latin}
\begin{lstlisting}[language=C++, caption=]
#include<iostream>
using namespace std;

int main() {
	float salary ;
	cin >> salary ;
	
	float ensurance = ( salary * 7 ) / 100;
	float tax = ( salary * 10 ) / 100;
	float exactSalary = salary - ensurance - tax;
	
	cout << "ensurance : " << ensurance << endl;
	cout << "tax : " << tax << endl;
	cout << "exactSalary : " << exactSalary;
	
	return 0;
}
\end{lstlisting}
\end{latin}



\newpage


\noindent
10 . برنامه ای بنویسید که سن شما را به روز دریافت کرده و با سال ، ماه ، هفته و روز نمایش دهد .  




\begin{latin}
\begin{lstlisting}[language=C++, caption=]
#include<iostream>
using namespace std;

int main() {
	int age ;
	cin >> age ;
	
	int year , month , week , day;
	
	year = age / 365 ;
	age = age % 365 ;
	
	month = age / 12 ;
	age = age % 12 ;
	
	week = age / 4 ;
	age = age % 4 ;
	
	day = age ;
	
	cout << "year : " << year << endl ;
	cout << "month : " << month << endl ;
	cout << "week : " << week << endl ;
	cout << "day : " << day << endl ;
	
	return 0;
}
\end{lstlisting}
\end{latin}




\newpage

\noindent
13 . برنامه ای بنویسید که
\lr{ATM}
 یک عدد صحیح دلخواه را به عنوان پول درخواستی از کاربر دریافت کند و سپس آن مبلغ را به پول های 1 و 5 و 10 و 50 هزار تومانی خرد کند .
 
 
 
 
 \begin{latin}
\begin{lstlisting}[language=C++, caption=]
#include<iostream>
using namespace std;

int main() {
	int cash ;
	cin >> cash ;
	
	int fifty, ten , five , one ;
	
	fifty = cash / 50 ;
	cash %= 50 ;
	
	ten = cash / 10 ;
	cash %= 10 ;
	
	five = cash / 5 ;
	cash %= 5 ;
	
	one = cash ;
	
	cout << "fifty : " << fifty  << endl;
	cout << "ten : " << ten << endl;
	cout << "five : " << five << endl;
	cout << "one : " << one  << endl;
	
	return 0;
}
\end{lstlisting}
\end{latin}
 
 
 
 
\newpage
 
\noindent
14 . برنامه ای بنویسید که بدون استفاده از دستور شرطی 
\lr{if}
،
یک عدد از ورودی دریافت کرده و مقدار قدر مطلق آن را نمایش دهد .



\begin{latin}
\begin{lstlisting}[language=C++, caption=]
#include<iostream>
#include<cmath>
using namespace std;

int main() {
	int number ; 
	cin >> number;
	
	cout << abs(number);
	
	return 0;
}
\end{lstlisting}
\end{latin}



\noindent
15 . برنامه ای بنویسید که دو عدد دلخواه را از ورودی دریافت کرده و ماکزیمم و مینیمم این دو عدد را بدون استفاده از دستور شرطی
\lr{if}
محاسبه کرده و نمایش دهد .


\begin{align*}
Max(x,y) = \cfrac{|x+y| + |x-y|}{2} \qquad \qquad
Min(x,y) = \cfrac{|x+y| - |x-y|}{2}
\end{align*}



\begin{latin}
\begin{lstlisting}[language=C++, caption=]
#include<iostream>
#include<cmath>
using namespace std;

int main() {
	int x , y ;
	
	cout << "Max : " << ( abs(x+y) + abs(x-y) ) / 2  << endl ;
	
	cout << "Min : " << ( abs(x+y) - abs(x-y) ) / 2  << endl ;
	
	return 0;
}
\end{lstlisting}
\end{latin}





\noindent
16 . برنامه ای بنویسید که 2 عدد را از ورودی دریافت کرده و در دو متغیر قرار دهد ، سپس بدون استفاده از متغیر سوم ، مقدار این دو متغیر را با یکدیگر عوض کرده و نمایش دهد .





\begin{latin}
\begin{lstlisting}[language=C++, caption=]
#include<iostream>
using namespace std;

int main() {
	int a, b;
    cin >> a >> b;
    
    a = a + b;
    b = a - b;
    a = a - b;
    
    cout << a << endl;
    cout << b ;
    
	return 0;
}
\end{lstlisting}
\end{latin}





\noindent
17 . برنامه ای بنویسید که عددی را از ورودی دریافت کرده و مشخص کند عدد وارد شده زوج است یا فرد .






\begin{latin}
\begin{lstlisting}[language=C++, caption=]
#include<iostream>
using namespace std;

int main() {
	int number ; 
	cin >> number ;
	if ( number % 2 == 0 ) {
		cout << "ZOJ" ;
	} else {
		cout << "FARD" ;
	}
	return 0;
}
\end{lstlisting}
\end{latin}









\noindent
18 . برنامه ای بنویسید که 2 عدد را از ورودی دریافت کرده و عدد بزرگتر را نمایش دهد .




\begin{latin}
\begin{lstlisting}[language=C++, caption=]
#include<iostream>
using namespace std;

int main() {
	int a , b ;
	cin >> a >>  b;
    
	if ( a > b ) {
		cout << a << ">" << b ;
	} else if (a < b) {
		cout << a << "<" << b ;
	} else if (a == b) {
		cout << a << "==" << b ;
	}
	
	return 0;
}
\end{lstlisting}
\end{latin}




\newpage

\noindent
19 . برنامه ای بنویسید که نمرات 5 درس یک دانش آموز را دریافت کرده و معدل انش آموز را محاسبه نماید، اگر معدل دانش آموز کمتر از 12 بود ، دانش آموز مشروط ، اگر معدل بیشتر از 17 بود به عنوان دانش آموز ممتاز و در غیر این صورت به عنوان دانش آموز متوسط معرفی نماید .







\begin{latin}
\begin{lstlisting}[language=C++, caption=]
#include<iostream>
using namespace std;

int main() {
	const int size = 5;
	int a[size];
	int sum = 0;
	
	for(int i=0;i<size;i++) {
		cin >> a[i];
		sum += a[i];
	}
	
	float avg = sum / size;
	
	cout << "Average : " << avg << endl ;
	
	if( avg < 12 ) {
		cout << "Mashrot";
	} else if( avg > 17 ) {
		cout << "Momtaz";
	} else {
		cout << "Motevaset";
	}
	
	return 0;
}
\end{lstlisting}
\end{latin}








\newpage

\noindent
20 . برنامه ای بنویسید که 4 عدد را از ورودی دریافت کند و بزرگترین عدد را نمایش دهد .





\begin{latin}
\begin{lstlisting}[language=C++, caption=]
#include<iostream>
using namespace std;

int main() {
	const int size = 5;
	int a[size];
	int max ;
	
	for(int i=0;i<size;i++) {
		cin >> a[i];
		if( i == 0 ) {
			max = a[i];
		} else {
			if(a[i] > max) {
				max = a[i];
			}
		}
	}
	
	cout << "Max : " << max ;
	
	return 0;
}
\end{lstlisting}
\end{latin}






\newpage

\noindent
21 . برنامه ای بنویسید که با استفاده از دستور 
\lr{switch}
یک عدد از 0 تا 7 را دریافت کرده و نام روز متناسب با آن را نمایش دهد . ( به طور مثال اگر عدد وارد شده 0 بود ، روز شنبه را نمایش دهد و اگر عدد وارد شده 6 بود ، روز جمعه را نمایش دهد  )




\begin{latin}
\begin{lstlisting}[language=C++, caption=,basicstyle=\ttfamily\small]
#include<iostream>
using namespace std;
int main() {
	int day ;
	cin >> day ;
	switch(day) {
		case 0:
			cout << "Shanbe" << endl;
			break;
		case 1: 
			cout << "1 Shanbe" << endl;
			break;
		case 2:
			cout << "2 Shanbe" << endl;
			break;
		case 3:
			cout << "3 Shanbe" << endl;
			break;
		case 4:
			cout << "4 Shanbe" << endl;
			break;
		case 5:
			cout << "5 Shanbe" << endl;
			break;
		case 6:
			cout << "Jome" << endl;
			break;
		case 7:
			cout << "Shanbe" << endl;
			break;
	}
	return 0;
}
\end{lstlisting}
\end{latin}








\noindent
22 . برنامه ای بنویسید که 4 عدد را از ورودی دریافت کرده و دومین بزرگترین عدد را نمایش دهد .







\begin{latin}
\begin{lstlisting}[language=C++, caption=]
#include<iostream>
using namespace std;

int main() {
	const int size = 4;
	int a[size];
	int max ;
	int secondMax ; 
	
	for(int i=0;i<size;i++) {
		cin >> a[i];
		if( i == 0 ) {
			secondMax = max = a[i];
		} else {
			if(a[i] > max) {
				secondMax = max;
				max = a[i];
			}
		}
	}
	
	cout << "Second Max : " << secondMax ;
	
	return 0;
}
\end{lstlisting}
\end{latin}








\newpage

\noindent
23 . برنامه ای بنویسید که حقوق ناخالص کارمندی را دریافت کرده و میزان مالیت را بر اساس قوانین زیر محاسبه کند .

\begin{itemize}
	\item اگر حقوق ناخالص کمتر از 1000 بود معاف از مالیات
	\item اگر حقوق ناخالص کمتر از 2000 بود نرخ مالیات
	 5\%
	\item اگر حقوق ناخالص کمتر از 3000 بود نرخ مالیات
	10\%
	\item اگر حقوق ناخالص بیشتر از 3000 بود نرخ مالیات
	15\%
\end{itemize}






\begin{latin}
\begin{lstlisting}[language=C++, caption=]
#include<iostream>
using namespace std;

int main() {
	int salary ;
	int tax ;
	cin >> salary;
	
	if(salary < 1000) {
		tax = 0;
	} else if(salary < 2000) {
		tax = (salary * 5) / 100;
	} else if(salary < 3000) {
		tax = (salary * 10) / 100;
	} else if(salary >= 3000) {
		tax = (salary * 15) / 100;
	}
	
	cout << "Tax : " << tax;
	
	return 0;
}
\end{lstlisting}
\end{latin}






\newpage

\noindent
24 . برنامه ای بنویسید که 3 عدد دلخواه را از ورودی دریافت کرده و مشخص کند که آیا این 3 عدد تشکیل یک مثلث خواهند داد یا خیر .



\begin{align*}
\text{شرط تشکیل مثلث : } \\ 
a+b>c \qquad \&\& \qquad b+c>a \qquad \&\& \qquad a+c>b
\end{align*}




\begin{latin}
\begin{lstlisting}[language=C++, caption=]
#include<iostream>
using namespace std;

int main() {
	float a,b,c;
	cin >> a >> b >> c;
	
	if( a+b>c && b+c>a && a+c>b ) {
		cout << "yes";
	} else {
		cout << "no";
	}
	
	return 0;
}
\end{lstlisting}
\end{latin}




\newpage

\noindent
25 . برنامه ای بنویسید که 3 عدد دلخواه را به عنوان اضلاع یک مثلث دریافت کند و بررسی کند که این مثلث متساوی الساقین است یا خیر . ( مثلث متساوی الساقین دارای 2 ضلع برابر است )



\begin{latin}
\begin{lstlisting}[language=C++, caption=]
#include<iostream>
using namespace std;

int main() {
	float a,b,c;
	cin >> a >> b >> c;
	
	if( a == b || a == c || b == c ) {
		cout << "yes";
	} else {
		cout << "no";
	}
	
	return 0;
}
\end{lstlisting}
\end{latin}




\newpage

\noindent
26 . برنامه ای بنویسید که 3 عدد دلخواه را به عنوان اضلاع یک مثلث دریافت کند و بررسی کند که این مثلث متساوی الضلاع است یا خیر . ( مثلث متساوی الاضلاع دارای 3 ضلع برابر است )







\begin{latin}
\begin{lstlisting}[language=C++, caption=]
#include<iostream>
using namespace std;

int main() {
	float a,b,c;
	cin >> a >> b >> c;
	
	if( a == b && b == c ) {
		cout << "yes";
	} else {
		cout << "no";
	}
	
	return 0;
}
\end{lstlisting}
\end{latin}




\newpage

\noindent
27 . برنامه ای بنویسید که 3 عدد دلخواه را به عنوان اضلاع یک مثلث دریافت کند و بررسی کند که این مثلث قائم الزاویه است یا خیر . 

در مثلث قائم الزاویه یکی از روابط زیر برقرار است :


\begin{align*}
\colorbox{gray!10}{\parbox{90pt}{
$$a^{2} = b^{2} + c^{2}$$
}}
\qquad
\colorbox{gray!10}{\parbox{90pt}{
$$b^{2} = a^{2} + c^{2}$$
}}
\qquad
\colorbox{gray!10}{\parbox{90pt}{
$$c^{2} = a^{2} + b^{2} $$
}}
\end{align*}






\begin{latin}
\begin{lstlisting}[language=C++, caption=]
#include<iostream>
#include<cmath>
using namespace std;

int main() {
	float a,b,c;
	cin >> a >> b >> c;
	
	if( pow(a,2) == pow(b,2) + pow(c,2) || 
		pow(b,2) == pow(a,2) + pow(c,2) || 
		pow(c,2) == pow(a,2) + pow(b,2) 
	) {
		cout << "yes";
	} else {
		cout << "no";
	}
	
	return 0;
}
\end{lstlisting}
\end{latin}






\newpage

\noindent
29 . برنامه ای بنویسید که اعداد 1 تا 100 را چاپ کند .



\begin{latin}
\begin{lstlisting}[language=C++, caption=]
#include<iostream>
using namespace std;

int main() {
	for(int i=1;i<=100;i++) {
		cout << i << endl;
	}
	
	return 0;
}
\end{lstlisting}
\end{latin}




\noindent
30 . برنامه ای بنویسید که حاصل جمع اعداد 1 تا 100 را چاپ کند .





\begin{latin}
\begin{lstlisting}[language=C++, caption=]
#include<iostream>
using namespace std;

int main() {
	int sum = 0;
	
	for(int i=1;i<=100;i++) {
		sum += i;
	}
	
	cout << "Sum : " << sum;
	
	return 0;
}
\end{lstlisting}
\end{latin}




\newpage


\noindent
31 . برنامه ای بنویسید که اعداد زوج بین 1 تا 100 را چاپ کند .





\begin{latin}
\begin{lstlisting}[language=C++, caption=]
#include<iostream>
using namespace std;

int main() {
	for(int i=1;i<=100;i++) {
		if( i % 2 == 0 ) {
			cout << i << endl;
		}
	}

	return 0;
}
\end{lstlisting}
\end{latin}








\noindent
32 . برنامه ای بنویسید که حاصل جمع اعداد فرد بین 1 تا 100 را چاپ کند .





\begin{latin}
\begin{lstlisting}[language=C++, caption=]
#include<iostream>
using namespace std;

int main() {
	int sum = 0;
	
	for(int i=1;i<=100;i++) {
		if( i % 2 != 0 ) {
			sum += i;
		}
	}
	
	cout << "Sum : " << sum;
	
	return 0;
}
\end{lstlisting}
\end{latin}






\newpage

\noindent
33 . برنامه ای بنویسید که یک عدد صحیح را از ورودی دریافت کرده و اعداد کوچکتر از آن را چاپ نماید .





\begin{latin}
\begin{lstlisting}[language=C++, caption=]
#include<iostream>
using namespace std;

int main() {
	int n ;
	cin >> n;
	
	for(int i=n-1;i>0;i--) {
		cout << i << endl;
	}

	return 0;
}
\end{lstlisting}
\end{latin}









\noindent
34 . برنامه ای بنویسید که یک عدد را از ورودی دریافت کرده و مقسوم علیه های آن را چاپ کند .






\begin{latin}
\begin{lstlisting}[language=C++, caption=]
#include<iostream>
using namespace std;

int main() {
	int n ;
	cin >> n;
	
	for(int i=n/2;i>0;i--) {
		if( n % i == 0 ) {
			cout << i << endl;
		}
	}

	return 0;
}
\end{lstlisting}
\end{latin}






\newpage

\noindent
35 . برنامه ای بنویسید که یک عدد را از ورودی دریافت کرده و مجموعه مقسوم علیه های آن را چاپ کند .




\begin{latin}
\begin{lstlisting}[language=C++, caption=]
#include<iostream>
using namespace std;

int main() {
	int n ;
	cin >> n;
	
	int sum = 0;
	
	for(int i=n/2;i>0;i--) {
		if( n % i == 0 ) {
			sum += i;
		}
	}

	cout << "sum : " << sum;
	
	return 0;
}
\end{lstlisting}
\end{latin}








\newpage

\noindent
36 . برنامه ای بنویسید که یک عدد را از ورودی دریافت کرده و تعداد مقسوم علیه های آن را چاپ کند .







\begin{latin}
\begin{lstlisting}[language=C++, caption=]
#include<iostream>
using namespace std;

int main() {
	int n ;
	cin >> n;
	
	int counter = 0;
	
	for(int i=n/2;i>0;i--) {
		if( n % i == 0 ) {
			counter++;
		}
	}

	cout << "counter : " << counter;
	
	return 0;
}
\end{lstlisting}
\end{latin}








\newpage


\noindent
37 . برنامه ای بنویسید که یک عدد را از ورودی دریافت کرده و مجموع مقسوم علیه های فرد آن را چاپ کند .






\begin{latin}
\begin{lstlisting}[language=C++, caption=]
#include<iostream>
using namespace std;

int main() {
	int n ;
	cin >> n;
	
	int sum = 0;
	
	for(int i=n/2;i>0;i--) {
		if( n % i == 0 ) {
			if( i%2 != 0 ) {
				sum += i ;
			}
		}
	}

	cout << "sum : " << sum;
	
	return 0;
}
\end{lstlisting}
\end{latin}





\newpage


\noindent
38 . برنامه ای بنویسید که یک عدد را از ورودی دریافت کرده و تعداد مقسوم علیه های زوج آن را چاپ کند .







\begin{latin}
\begin{lstlisting}[language=C++, caption=]
#include<iostream>
using namespace std;

int main() {
	int n ;
	cin >> n;
	
	int counter = 0;
	
	for(int i=n/2;i>0;i--) {
		if( n % i == 0 ) {
			if( i%2 == 0 ) {
				counter++;
			}
		}
	}

	cout << "counter : " << counter;
	
	return 0;
}
\end{lstlisting}
\end{latin}








\newpage

\noindent
39 . برنامه ای بنویسید که یک عدد را از ورودی دریافت کرده و مشخص کند که عدد وارد شده عدد اول است یا خیر . ( عددی اول است که به غیر از 1 و خودش مقسوم علیه دیگری نداشته باشد ) 






\begin{latin}
\begin{lstlisting}[language=C++, caption=]
#include<iostream>
using namespace std;

int main() {
	int n ;
	cin >> n;
	
	bool aval = true;
	
	for(int i=2;i*i<=n;i++) {
		if( n % i == 0 ) {
			aval = false;
		}
	}

	if(aval == true) {
		cout << "AVAL";
	} else {
		cout << "NOT AVAL";
	}
	
	return 0;
}
\end{lstlisting}
\end{latin}










\newpage

\noindent
40 . برنامه ای بنویسید که یک عدد را از ورودی دریافت کرده و مشخص کند که عدد وارد شده عدد کامل است یا خیر . ( عددی کامل است که مجموع مقسوم علیه های به غیر از خودش با خود عدد برابر باشد )





\begin{latin}
\begin{lstlisting}[language=C++, caption=]
#include<iostream>
using namespace std;

int main() {
	int n ;
	cin >> n;
	
	int sum = 0;
	
	for(int i=n/2;i>=1;i--) {
		if( n % i == 0 ) {
			sum += i;
		}
	}

	if(sum == n) {
		cout << "Kamel";
	} else {
		cout << "NOT Kamel";
	}
	
	return 0;
}
\end{lstlisting}
\end{latin}







\newpage

\noindent
41 . برنامه ای بنویسید که با استفاده از حلقه ی تکرار 10 عدد را از ورودی دریافت کند و میانگین آنها را نمایش دهد .






\begin{latin}
\begin{lstlisting}[language=C++, caption=]
#include<iostream>
using namespace std;

int main() {
	const int size = 10;
	int a[size];
	int sum = 0;
	
	for(int i=0;i<size;i++) {
		cin >> a[i];
		sum += a[i];
	}
	
	float avg = sum / size;

	cout << "Average : " << avg << endl;
	
	return 0;
}
\end{lstlisting}
\end{latin}







\newpage

\noindent
42 . برنامه ای بنویسید که با استفاده از حلقه ی تکرار 10 عدد را از ورودی دریافت کند و بزرگترین آنها را نمایش دهد .







\begin{latin}
\begin{lstlisting}[language=C++, caption=]
#include<iostream>
using namespace std;

int main() {
	const int size = 10;
	int a[size];
	int max ;
	
	for(int i=0;i<size;i++) {
		cin >> a[i];
		if( i == 0 ) {
			max = a[i];
		} else {
			if(a[i] > max) {
				max = a[i];
			}
		}
	}
	
	cout << "Max : " << max ;
	
	return 0;
}
\end{lstlisting}
\end{latin}






\newpage

\noindent
43 . برنامه ای بنویسید که با استفاده از حلقه ی تکرار 10 عدد را از ورودی دریافت کند و دومین بزرگترین آنها را نمایش دهد .




\begin{latin}
\begin{lstlisting}[language=C++, caption=]
#include<iostream>
using namespace std;

int main() {
	const int size = 10;
	int a[size];
	int max ;
	int secondMax ; 
	
	for(int i=0;i<size;i++) {
		cin >> a[i];
		if( i == 0 ) {
			secondMax = max = a[i];
		} else {
			if(a[i] > max) {
				secondMax = max;
				max = a[i];
			}
		}
	}
	
	cout << "Second Max : " << secondMax ;
	
	return 0;
}
\end{lstlisting}
\end{latin}







\newpage

\noindent
44 . برنامه ای بنویسید که عددی را از ورودی دریافت کرده و تعداد ارقام آن را نمایش دهد . ( برای مثال عدد 123 ، 3 رقم دارد )








\begin{latin}
\begin{lstlisting}[language=C++, caption=]
#include<iostream>
using namespace std;

int main() {
	int number ;
	cin >> number ;
	
	int counter = 0;
	
	while( number != 0 ) {
		counter++;
		number /= 10;
	}
	
	cout << "counter : " << counter ;
	
	return 0;
}
\end{lstlisting}
\end{latin}







\newpage

\noindent
45 . برنامه ای بنویسید که یک عدد صحیح را از ورودی دریافت کرده و عدد وارد شده را مقلوب نماید . ( برای مثال مقلوب عدد 123 ، عدد 321 است )






\begin{latin}
\begin{lstlisting}[language=C++, caption=]
#include<iostream>
using namespace std;

int main() {
	int number ;
	cin >> number ;
	
	int counter;
	int digit;
	
	while( number != 0 ) {
		digit = number % 10;
		number /= 10;
		cout << digit ;
	}
	
	return 0;
}
\end{lstlisting}
\end{latin}







\newpage

\noindent
46 . برنامه ای بنویسید که یک عدد صحیح را دریافت کرده و مجموع اعداد زوج آن را نمایش دهد . ( برای مثال مجموع اعداد زوج 249 ، عدد 6 می باشد )



\begin{latin}
\begin{lstlisting}[language=C++, caption=]
#include<iostream>
using namespace std;

int main() {
	int number ;
	cin >> number ;
	
	int counter;
	int digit;
	int sum = 0;
	
	while( number != 0 ) {
		digit = number % 10;
		number /= 10;
		if (digit % 2 == 0) {
			sum += digit;
		}
	}
	
	cout << "sum : " << sum ;
	
	return 0;
}
\end{lstlisting}
\end{latin}






\newpage

\noindent
47 . برنامه ای بنویسید که یم عدد صحیح را دریافت کرده و تعداد اعداد فرد آن را نمایش دهد . ( برای مثال تعداد اعداد فرد عدد 163 ، عدد 2 می باشد )






\begin{latin}
\begin{lstlisting}[language=C++, caption=]
#include<iostream>
using namespace std;

int main() {
	int number ;
	cin >> number ;
	
	int digit;
	int counter = 0;
	
	while( number != 0 ) {
		digit = number % 10;
		number /= 10;
		if (digit % 2 != 0) {
			counter++;
		}
	}
	
	cout << "counter : " << counter ;
	
	return 0;
}
\end{lstlisting}
\end{latin}







\newpage

\noindent
49 . برنامه ای بنویسید که 50 جمله ی اول سری فیبوناتچی را چاپ کند .

$$
0 , 1 , 1 , 2 , 3 , 5 , 8 , 13 , 21 , 34 , 55 , \dots
$$


\begin{align*}
F(n) = 
\begin{cases}
0 & n = 0 \\
1 & n = 1 \\
F(n-1) + F(n-2) & n > 1 
\end{cases}
\end{align*}






\begin{latin}
\begin{lstlisting}[language=C++, caption=]
#include<iostream>
using namespace std;

int main() {	
	long long first = 0;
	long long second = 1;
	long long temp = second;
	
	cout << first << endl;
	cout << second << endl;
	
	for(int i=0;i<50;i++) {
		
		temp = second;
		second += first ;
		first = temp;
		
		cout << second << endl ;
	}
	
	return 0;
}
\end{lstlisting}
\end{latin}






\newpage

\noindent
51 . برنامه ای بنویسید که یک جدول ضرب 10 در 10 را چاپ نماید .







\begin{latin}
\begin{lstlisting}[language=C++, caption=]
#include<iostream>
#include <iomanip>
using namespace std;

int main() {
	for(int i=1;i<=10;i++) {
		for(int j=1;j<=10;j++) {
			cout << left << setw(5) << i * j ;
		}
		cout << endl;
	}
	
	return 0;
}
\end{lstlisting}
\end{latin}







\newpage

\noindent
54 . برنامه ای بنویسید که یک عدد دلخواه را از ورودی دریافت کرده و معادل مبنای 2 آن را نمایش دهد .






\begin{latin}
\begin{lstlisting}[language=C++, caption=]
#include<iostream>
using namespace std;

int main() {
	int base = 2;
	
	int n ;
	cin >> n;
	
	int digit;
	int binary = 0 ;
	int counter = 0;
	int ten = 1;
	
	while( n != 0 ) {
		digit = n % base;
		binary += digit * ten;
		n /= base;
		
		ten *= 10;
		counter++;
	}
	
	cout << binary << endl;
	
	return 0;
}
\end{lstlisting}
\end{latin}







\newpage

\noindent
55 . برنامه ای بنویسید که یک عدد دلخواه را از ورودی دریافت کرده و معادل مبنای 8 آن را نمایش دهد .






\begin{latin}
\begin{lstlisting}[language=C++, caption=]
#include<iostream>
using namespace std;

int main() {
	int base = 8;
	
	int n ;
	cin >> n;
	
	int digit;
	int octal = 0;
	int counter = 1;
	int ten = 1;
	
	while( n != 0 ) {
		digit = n % base ;
		octal += digit * ten;
		n /= base;
		
		ten *= 10;
		counter++;
	}
	
	cout << octal;
	
	return 0;
}
\end{lstlisting}
\end{latin}







\newpage

\noindent
59 . برنامه ای بنویسید که یک عدد را از ورودی دریافت کرده و فاکتوریل آن را محاسبه نماید .

\begin{tcolorbox}
$$
n ! = 1 \times 2 \times 3 \times \dots \times (n-1) \times (n-2) \times n
$$
\end{tcolorbox}







\begin{latin}
\begin{lstlisting}[language=C++, caption=]
#include<iostream>
using namespace std;

int main() {
	int n ;
	cin >> n;
	
	int fact = 1;
	
	for(int i=1;i<=n;i++) {
		fact *= i;
	}
	
	cout << "factorial(" << n << ") : " << fact << endl;  
	
	return 0;
}
\end{lstlisting}
\end{latin}








\end{document}
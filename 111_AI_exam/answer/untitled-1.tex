\documentclass[12pt]{article}

\usepackage{tabularx}
\usepackage[table]{xcolor}
\usepackage{multirow}


\usepackage{amsmath, amssymb}
\usepackage{mathtools}

\usepackage[margin=1in,footskip=.25in]{geometry}

\usepackage{listings}

\usepackage[most]{tcolorbox}

\tcbset{
    frame code={}
    center title,
    left=10pt,
    right=10pt,
    top=10pt,
    bottom=10pt,
    colback=gray!5,
    colframe=gray,
    width=\dimexpr\textwidth\relax,
    enlarge left by=0mm,
    boxsep=5pt,
    arc=0pt,outer arc=0pt,
}

\usepackage{xepersian}
\settextfont[Scale=1]{Vazir}
\renewcommand{\baselinestretch}{1.3} 

\begin{document}

\pagenumbering{gobble} % remove page numbering

\noindent
هوش مصنوعی چیست ؟

\begin{tcolorbox}
به توانایی فکر کردن و یادگرفتن یک برنامه ی کامپیوتری یا ماشین  هوش مصنوعی می گویند 
\end{tcolorbox}


\vspace{30pt}

\noindent
انواع دانش را نام ببرید ؟

\begin{tcolorbox}
\textbf{  دانش صریح : }


دانش صریح یا آشکار، دانشی است که به‌وضوح تعریف یا فرموله می‌شود و از طریق فناوری‌های اطلاعاتی نیز به اشتراک گذاشته می‌شود. دانش صریح می‌تواند در ذخایر کتابخانه‌ها، آرشیوها، پایگاه‌های اطلاعاتی ذخیره‌شده و بر اساس یک مبنای مشخص ارزیابی شود.

\textbf{دانش ضمنی : }

دانش ضمنی از مدل‌های ذهنی، باورها و اعتقادات هر فرد تشکیل می‌شود که آن‌چنان در ذهن وی جای گرفته‌اند که بدیهی تلقی می‌شوند. دانش ضمنی، ریشه در درون افراد دارد و بیان آن در قالب کلمات دشوار است
\end{tcolorbox}


\vspace{30pt}

\noindent
زبانهای مورد استفاده در فرموله کردن و نمایش دانش را نام ببرید ؟
\begin{latin}
\begin{tcolorbox}
There are mainly four ways of knowledge representation which are given as follows: 
\begin{itemize}
	\item Logical Representation
	\item Semantic Network Representation
	\item Frame Representation
	\item Production Rules
\end{itemize}
\end{tcolorbox}
\end{latin}

\vspace{30pt}


\noindent
هرس
\textbf{$\alpha$}
و
\textbf{$\beta$}
را شرح دهید ؟

\begin{tcolorbox}
هرس آلفا بتا 
\lr{(Alpha Beta Pruning)}
 الگوریتمی است که کارایی الگوریتم درخت
  \lr{Minimax}
  (درخت کمینه بیشینه یا درخت بازی) را بهبود می بخشد؛ با استفاده از هرس آلفا بتا، بخش هایی از درخت کمینه بیشینه که پیمایششان بی تأثیر است پیمایش نمی شوند و به این ترتیب پیمایش درخت کمینه بیشینه تا یک عمق مشخص در زمانی کمتر صورت می گیرد.
\end{tcolorbox}

\vspace{30pt}


\noindent
انواع روابط در شبکه ی معنی را نام ببرید ؟


\begin{tcolorbox}
شبکه معنایی 
\lr{(semantic network)}
 یا شبکه قاب 
\lr{(frame network)}
  یک نوع پایگاه دانش است که «روابط معنایی بین مفاهیم» را در یک شبکه نمایش می دهد.
\end{tcolorbox}


\begin{tcolorbox}
بعضی از معمول ترین رابطه‌های معنایی تعریف شده از این قرار است: 
\begin{itemize}
	\item جزءواژگی 
	\lr{(meronymy)} : 
	\lr{A}
	یک جزء واژه از
	\lr{B}
	است اگر 
	\lr{A}
	بخشی از
	\lr{B}
	باشد .
	\item کل‌واژه 
	\lr{(holonymy)} : 
	\lr{B}
	یک کل واژه از
	\lr{A}
	است اگر
	\lr{B}
	شامل 
	\lr{A}
	باشد .
	\item تابعیت
	\lr{(troponymy)} : 
	\lr{A}
	از 
	\lr{B}
	تبعیت دارد یعنی
	\lr{A}
	نوعی از 
	\lr{B}
	است .
	\item  مافوق بودن
	\lr{(hypernymy)}
	\item مترادف بودن 
	\lr{(synonymy)} :
	\lr{A}
	به همان چیزی اشاره می کند که
	\lr{B}
	اشاره می کند .
	\item متضاد
	\lr{(antonymy)} :
	\lr{A}
	به چیزی اشاره می کند که متضاد با چیزی است که
	\lr{B}
	به آن اشاره می کند .
\end{itemize}
    
\end{tcolorbox}


\vspace{30pt}



\noindent
مزایا و معایب شبکه ی معنی را بیان کنید ؟


\vspace{30pt}



\noindent
روشهای استنتاج از قوانین را نام ببرید ؟


\vspace{30pt}


\noindent
خصوصیات قابها را بیان کنید ؟

\begin{tcolorbox}
قاب 
\lr{(Frame)}
یا فریم نوعی ساختمان داده در هوش مصنوعی است، که از آن برای تقسیم دانش به زیرساختارها از طریق نمایش «وضعیت تفکر قالبی 
\lr{(stereotype)}
» استفاده می‌شود.
\end{tcolorbox}



\begin{tcolorbox}
مزیت اول: وجود استثنا

مزیت دوم: ساخت ساده‌تر شبکه معنایی

مزیت سوم: اسنتاج ساده

مزیت چهارم: انعطاف‌پذیری از طریق رویه
\end{tcolorbox}



\begin{tcolorbox}
ساختار قاب

یک فریم شامل اطلاعاتی دربارهٔ «چگونگی استفاده از قاب»، «در آینده چه مورد انتظار است»، «اگر این انتظارها برآورده نشود باید چه کرد» هستند.

بخشی از اطلاعات در قاب معمولاً تغییر نمی‌کنند در حالیکه اطلاعات دیگر که در «پایانه» 
\lr{(terminal)}
 ذخیره شده‌اند، معمولاً تغییر می‌کنند. ترمینال‌ها را می‌توان به صورت متغیر درنظر گرفت.
\end{tcolorbox}


\vspace{30pt}



\noindent
با توجه به پایگاه دانش زیر به سوالات داده شده پاسخ دهید ؟





\begin{latin}
\begin{align*}
&add(0,x,x) \\
&add(1,4,5) \\
&add(1,3,4) \\
&sub(x,y,z) \leftarrow ge(x,y) \land add(y,z,x) \\
&ge(4,1) \\
&ge(5,1) \\
\end{align*}
\end{latin}


\noindent
سوالات :


\begin{latin}
\begin{align*}
add(1,4,5) \\
add(x,y,5) \\
add(0,x,x) \\
sub(x,y,z) \\
\end{align*}
\end{latin}




\end{document}
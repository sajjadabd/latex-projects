\documentclass[12pt]{article}

\usepackage{tabularx}
\usepackage[table]{xcolor}
\usepackage{multirow}


\newcolumntype{C}{ >{\centering\arraybackslash} m{4cm} }
\newcolumntype{E}{ >{\centering\arraybackslash} m{11cm} }
\newcolumntype{I}{ >{\centering\arraybackslash} m{12cm} }
\newcolumntype{D}{ >{\centering\arraybackslash} m{3cm} }
\newcolumntype{F}{ >{\centering\arraybackslash} m{1cm} }
\newcolumntype{G}{ >{\centering\arraybackslash} m{2cm} }
\newcolumntype{H}{ >{\centering\arraybackslash} m{8cm} }
\newcolumntype{K}{ >{\centering\arraybackslash} m{7cm} }
\newcolumntype{Y}{ >{\centering\arraybackslash} m{10cm} }

\usepackage{amsmath, amssymb}
\usepackage{mathtools}

\usepackage{listings}
\lstdefinestyle{mystyle}{
    basicstyle=\ttfamily\small,
    breakatwhitespace=false,         
    breaklines=true,                 
    captionpos=b,                    
    keepspaces=true,                               
    showspaces=false,                
    showstringspaces=false,
    showtabs=false,                  
    tabsize=3
}






\lstset{style=mystyle}


\usepackage[margin=1in,footskip=.25in]{geometry}

\usepackage{xepersian}
\settextfont[Scale=1]{Vazir}

\renewcommand{\baselinestretch}{1.3} 

\begin{document}

\noindent
به نظر شما هدف از ارائه ی درس طراحی الگوریتم ها در مقطع کارشناسی فناوری اطلاعات چیست ؟


\vspace{30pt}

\noindent
در تولید یک محصول نرم افزاری محاسبه ی پیچیدگی زمانی و فضایی یکی از مهمترین متریک های محاسبه شده توسط مهندسین نرم افزار است .


\vspace{30pt}


\noindent
الف ) به نظر شما چرا پیچیدگی الگوریتم و نرم افزار را محاسبه می کنند ؟

\vspace{30pt}


\noindent
ب ) منظور از 
$f(n) = O(g(n))$
و
$f(n) = \Omega(g(n))$
و
$f(n) = \theta(g(n))$
چیست ؟


\vspace{30pt}



\noindent
ج ) کدامیک از موارد زیر می تواند درست باشد .

\begin{align*}
\begin{rcases}
f(n) = O(g(n)) \\
f(n) = \theta(g(n))
\end{rcases} \Rightarrow
f(n) = \Omega(g(n))
\end{align*}


\begin{align*}
\begin{rcases}
f(n) = \Omega(g(n)) \\
f(n) = \theta(g(n))
\end{rcases} \Rightarrow
f(n) = \Omega(g(n))
\end{align*}




\begin{align*}
\begin{rcases}
f(n) = \theta(h(n)) \\
g(n) = \Omega(f(n))
\end{rcases} \Rightarrow
g(n) = \Omega(h(n))
\end{align*}




\begin{align*}
\begin{rcases}
f(n) = \Omega(g(n)) \\
f(n) = O(g(n))
\end{rcases} \Rightarrow
f(n) = \theta(g(n))
\end{align*}







\begin{align*}
\begin{rcases}
f(n) = \theta(g(n)) \\
g(n) = \Omega(f(n))
\end{rcases} \Rightarrow
f(n) = \theta(g(n))
\end{align*}



\newpage



\noindent
تفاوت میان الگوریتم های از نوع 
\textbf{تقسیم و غلبه}
 و
\textbf{حریصانه}
 چیست ؟


\vspace{30pt}





\noindent
اگر الگوریتمی روی سیستمی با اندازه ورودی 10 به مدت 8 میلی ثانیه اجرا شود ، همین الگوریتم با اندازه ورودی 100 روی سیستمی دیگر با قدرت پردازشی و سرعت اجرایی نصف سیستم موجود در چه زمانی اجرا خواهد شد ( پیچیدگی این الگوریتم از مرتبه ی 
$n\log{(n)}$
می باشد )


\vspace{30pt}



\noindent
الگوریتم جستجوی دودویی را بیان کنید و مشخص کنید از کدام دسته از الگوریتم هاست و پیچیدگی آن چیست ؟


\vspace{30pt}



\noindent
فلوچارتی را رسم کنید که عددی را از ورودی دریافت کند و مشخص کند که زوج است یا خیر ؟




\vspace{30pt}



\noindent
پیچیدگی موارد زیر را مشخص کنید ؟



\begin{latin}
\begin{center}
  \bgroup
  \def\arraystretch{1.5}%
  \begin{tabular}{ E  D  }
	\begin{lstlisting}[language=C++, caption=]
	z = 0;
	for(i=1;i<=n;i++) {
		for(j=1;j<=n;j+=2) {
			for(k=1;k<=n^2;n*=3) {
				z++;
			}
		}
	}
	\end{lstlisting}
     &  
	\begin{lstlisting}[language=C++, caption=]
	.
	.
	.
	.
	.
	.
	.
	.
	\end{lstlisting}
  \end{tabular}
  \egroup
\end{center}
\end{latin}






\begin{latin}
\begin{center}
  \bgroup
  \def\arraystretch{1.5}%
  \begin{tabular}{ I  D  }
	\begin{lstlisting}[language=C++, caption=]
	F(n) {
		if( n <= 1 ) {
			return 1;
		} else {
			return F(n/2) + F(n/3) + F(n-1) + F(n-2);
		}
	}
	\end{lstlisting}
     &  
	\begin{lstlisting}[language=C++, caption=]
	.
	.
	.
	.
	.
	.
	.
	\end{lstlisting}
  \end{tabular}
  \egroup
\end{center}
\end{latin}






\newpage


\noindent
فرض کنید تابع 
$g(x,y)$
به این صورت تعریف شده باشد که در مکان
$x$
به تعداد
$y$
علامت 
$*$
را چاپ کند . در این صورت :


\noindent
خروجی الگوریتم
\lr{\lstinline{F_Print(0,8,3)}}
چیست ؟

\noindent
پیچیدگی زمانی این الگوریتم را محاسبه کنید ؟


\begin{latin}
\begin{center}
  \bgroup
  \def\arraystretch{1.5}%
  \begin{tabular}{ E  D  }
	\begin{lstlisting}[language=C++, caption=]
	F_Print(int a,int b,int c) {
		int m = (a+b)/2 ;
		if(c > 0) {
			g(m,c);
			F_Print(a,m,c-1);
			F_Print(m,b,c-1);
		}
	}
	\end{lstlisting}
     &  
	\begin{lstlisting}[language=C++, caption=]
	.
	.
	.
	.
	.
	.
	.
	.
	\end{lstlisting}
  \end{tabular}
  \egroup
\end{center}
\end{latin}







\end{document}
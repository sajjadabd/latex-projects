\documentclass[12pt]{article}

\usepackage[siunitx, RPvoltages]{circuitikz}
\usepackage{tikz}
\usetikzlibrary{arrows}
\usetikzlibrary{shapes}

\begin{document}

\tikz \draw (0,0) to[R=$R_1$] (2,0);


\begin{circuitikz}[american]
 \draw[thin, <-, >=triangle 45] (1,2) node{$i_1$}  ++(-40:0.5) arc (-40:170:0.5);
 \draw (0,0) to[isource, l=$I_0$] (0,3) --(2,3)
  to[R=$R_1$] (2,0) -- (0,0);
 \draw (2,3) -- (4,3) to[R=$R_2$]
 (4,0) -- (2,0);
\end{circuitikz}



\begin{circuitikz}[american]
 \draw (0,0) to[isource, l=$I_0$] (0,3)
 to[short, -*, i=$I_0$] (2,3)
 to[R=$R_1$, i=$i_1$] (2,0) -- (0,0);
 \draw (2,3) -- (4,3)
 to[R=$R_2$, i=$i_2$]
 (4,0) to[short, -*] (2,0);
\end{circuitikz}




\begin{circuitikz}[american]
 \draw (0,0) to[isource, l=$I_0$] (0,3)
 to[short, -*, i=$I_0$] (2,3)
 to[R=$R_1$, i>_=$i_1$] (2,0) -- (0,0);
 \draw (2,3) -- (4,3)
 to[R=$R_2$, i>_=$i_2$]
 (4,0) to[short, -*] (2,0);
\end{circuitikz}






\begin{circuitikz}[american, voltage shift
=0.5]
 \draw (0,0) to[isource, l=$I_0$, v=$V_0$]
(0,3)
 to[short, -*, i=$I_0$] (2,3)
 to[R=$R_1$, i>_=$i_1$] (2,0) -- (0,0);
 \draw (2,3) -- (4,3)
 to[R=$R_2$, i>_=$i_2$]
 (4,0) to[short, -*] (2,0);
\end{circuitikz}






\begin{circuitikz}[american, voltage shift
=0.5]
 \draw (0,0) to[isource, l=$I_0$, v=$V_0$]
(0,3)
 to[short, -*, f=$I_0$] (2,3)
 to[R=$R_1$, f>_=$i_1$] (2,0) -- (0,0);
 \draw (2,3) -- (4,3)
 to[R=$R_2$, f>_=$i_2$]
 (4,0) to[short, -*] (2,0);
 \draw[red, thick] (1.5,2.5) rectangle
(4.5,3.5)
 node[pos=0.5, above]{KCL};
\end{circuitikz}








\begin{circuitikz}[bigR/.style={R, bipoles/length=3cm}]
 \draw (0,3) to [bigR, o-o] ++(4,0);
 \draw (0,1.5) to [bigR, o-o] ++(4,0)
 to[R, o-o] ++(2,0); % will fail here
 \draw (0,0) to [R, o-o] ++(4,0);
\end{circuitikz}



\begin{circuitikz}[bigR/.style={R, resistors/scale=1.8}]
 \draw (0,3) to [bigR, o-o] ++(4,0);
 \draw (0,1.5) to [bigR, o-o] ++(4,0)
 to[R, o-o] ++(2,0); % ok now
 \draw (0,0) to [R, o-o] ++(4,0);
\end{circuitikz}




{
\ctikzset{bipoles/thickness=1}
\tikz \draw (0,0) to[C=1<\farad>] (2,0); \par
\ctikzset{bipoles/thickness=4}
\tikz \draw (0,0) to[C=1<\farad>] (2,0);
}





{
\tikz \draw (0,0) to[R=1<\ohm>] (2,0); \par
\ctikzset{bipoles/resistor/height=.6}
\tikz \draw (0,0) to[R=1<\ohm>] (2,0);
}










\begin{circuitikz}[
 longpot/.style = {pR, resistors/scale=0.75,
 resistors/width=1.6, resistors/zigs=6}]
 \draw (0,1.5) to[R, l=$R$] ++(4,0);
 \draw (0,0) to[longpot, l=$P$] ++(4,0);
 \ctikzset{resistors/scale=1.5}
 \draw (0,-1.5) to[R, l=$R$] ++(4,0);
\end{circuitikz}












\begin{circuitikz}[american]
 \draw (0,0) to[cute choke] ++(3,0);
 \draw (0,-1) to[cute choke, twolineschoke] ++(3,0);

 \ctikzset{bipoles/cutechoke/cthick=2, twolineschoke}

 \draw (0,-2) to[cute choke] ++(3,0);
 \draw (0,-3) to[cute choke, onelinechoke] ++(3,0);
\end{circuitikz}











\begin{circuitikz}[american]
 \draw (0,1) to[sV=$V$] ++(3,0);
 \draw (0,0) to[sI=$I$] ++(3,0);
\end{circuitikz}






\begin{circuitikz}
 \draw (0,0) to[R, l=$R_1$] (2,0);
\end{circuitikz}






\begin{circuitikz}
 \draw (0,0) to[R=$R_1$] (2,0);
\end{circuitikz}



\begin{circuitikz}
 \draw (0,0) to[R, i=$i_1$] (2,0);
\end{circuitikz}





\begin{circuitikz}\draw
 (0,0) to[resistor=1<\kilo\ohm>] (2,0);
\end{circuitikz}





\begin{circuitikz}
 \draw (0,0) to[R, l=$R_1$,a=1<\kilo\ohm>] (2,0);
\end{circuitikz}




\begin{circuitikz}
 \draw (0,0) to[R, l_=$R_1$,a^=1<\kilo\ohm>] (2,0);
\end{circuitikz}




\begin{circuitikz}
 \draw (0,0) to[R, i^>=$i_1$] (2,0);
\end{circuitikz}


\begin{circuitikz}
 \draw (0,0) to[R, i_>=$i_1$] (2,0);
\end{circuitikz}



\begin{circuitikz}
 \draw (0,0) to[R, i^<=$i_1$] (2,0);
\end{circuitikz}



\begin{circuitikz}
 \draw (0,0) to[R, i_<=$i_1$] (2,0);
\end{circuitikz}



\begin{circuitikz}
 \draw (0,0) to[R, i>^=$i_1$] (2,0);
\end{circuitikz}

\begin{circuitikz}
 \draw (0,0) to[R, i>_=$i_1$] (2,0);
\end{circuitikz}

\begin{circuitikz}
 \draw (0,0) to[R, i<^=$i_1$] (2,0);
\end{circuitikz}

\begin{circuitikz}
 \draw (0,0) to[R, i<_=$i_1$] (2,0);
\end{circuitikz}




\begin{circuitikz}[american]
 \draw (0,0) to[V=10V, i_=$i_1$] (2,0);
\end{circuitikz}




\begin{circuitikz}[american]
 \draw (0,0) to[V=10V,invert, i_=$i_1$] (2,0);
\end{circuitikz}




\begin{circuitikz}
 \draw (0,0) to[R, f<=$i_1$] (3,0);
\end{circuitikz}

\begin{circuitikz}
 \draw (0,0) to[R, f>=$i_1$] (3,0);
\end{circuitikz}


\begin{circuitikz}
 \draw (0,0) to[R, f^<=$i_1$] (3,0);
\end{circuitikz}





\begin{circuitikz}
 \draw (0,0) to[R, f_<=$i_1$] (3,0);
\end{circuitikz}

\begin{circuitikz}
 \draw (0,0) to[R, f_>=$i_1$] (3,0);
\end{circuitikz}



\begin{circuitikz}
 \draw (0,0) to[battery,l_=1V, v=$u_1$, i=$i_1$] (2,0);
\end{circuitikz}





\begin{circuitikz}[american voltages]
 \draw (0,0) to[R, v^>=$v_1$] (2,0);
\end{circuitikz}


\begin{circuitikz}[american voltages]
 \draw (0,0) to[R, v_>=$v_1$] (2,0);
\end{circuitikz}



\begin{circuitikz}[american voltages]
 \draw (0,0) to[R, v^<=$v_1$] (2,0);
\end{circuitikz}


\begin{circuitikz}[american voltages]
 \draw (0,0) to[R, v_<=$v_1$] (2,0);
\end{circuitikz}





\begin{circuitikz}[american]
 \draw (0,0) to[I=1A, v_=$u_1$] (2,0);
\end{circuitikz}



\begin{circuitikz}[american]
 \draw (0,0) to[I<=1A, v_=$i_1$] (2,0);
\end{circuitikz}


\begin{circuitikz}[american voltages, voltage shift=0.5]
 \draw (0,0) to[battery,l_=1V, v=$u_1$, i=$i_1$] (2,0);
\end{circuitikz}


\begin{circuitikz}
 \draw (0,0) to[sI=$a_1$] (2,0);
\end{circuitikz}


\begin{circuitikz}
 \draw (0,0) to[csI=$k\cdot a_1$] (2,0);
\end{circuitikz}


\begin{circuitikz}[american currents]
 \draw (0,0) to[I=$a_1$] (2,0);
\end{circuitikz}


\begin{circuitikz}[american currents]
 \draw (0,0) to[I, i=$a_1$] (2,0);
\end{circuitikz}



\begin{circuitikz}[american currents]
 \draw (0,0) to[cI=$k\cdot a_1$] (2,0);
\end{circuitikz}


\begin{circuitikz}[american currents]
 \draw (0,0) to[sI=$a_1$] (2,0);
\end{circuitikz}



\begin{circuitikz}[american currents]
 \draw (0,0) to[csI=$k\cdot a_1$] (2,0);
\end{circuitikz}


\begin{circuitikz}
 \draw (0,0) to[sV=$a_1$] (2,0);
\end{circuitikz}


\begin{circuitikz}
 \draw (0,0) to[csV=$k\cdot a_1$] (2,0);
\end{circuitikz}


\begin{circuitikz}[american voltages]
 \draw (0,0) to[V=$a_1$] (2,0);
\end{circuitikz}


\begin{circuitikz}[american voltages]
 \draw (0,0) to[V, v=$a_1$] (2,0);
\end{circuitikz}



\begin{circuitikz}[american voltages]
 \draw (0,0) to[cV=$k v_e$] (2,0);
\end{circuitikz}



\begin{circuitikz}[american voltages]
 \draw (0,0) to[sV=$a_1$] (2,0);
\end{circuitikz}


\begin{circuitikz}[american voltages]
 \draw (0,0) to[csV=$k v_e$] (2,0);
\end{circuitikz}



\begin{circuitikz}
 \draw (0,0) to[R, l=1<\kilo\ohm>] (2,0);
\end{circuitikz}


\begin{circuitikz}
 \draw (0,0) to[R, l=$\SI{1}{\kilo\ohm}$] (2,0);
\end{circuitikz}



\begin{circuitikz}
 \draw (0,0) to[R, i=1<\milli\ampere>] (2,0);
\end{circuitikz}



\begin{circuitikz}
 \draw (0,0) to[R, i=$\SI{1}{\milli\ampere}$] (2,0);
\end{circuitikz}





\begin{circuitikz}
 \draw (0,0) to[R, *-o] ++(2,0) to[R, -d] ++(2,0)
 to[R, bipole nodes={diamondpole}{odiamondpole, fill=red}] ++(2,0);
 \draw (0,-1) to[R, *-o] ++(2,0) ;
 \draw (2,-1) to[R, -d] ++(2,0) to[R, bipole nodes={none}{squarepole}] ++(2,0);
\end{circuitikz}






\begin{circuitikz}
 \ctikzset{-s/.style = {bipole nodes={none}{osquarepole, fill=red}}}
 \draw (0,0) to[R, -s] ++(2,0);
\end{circuitikz}



\begin{circuitikz}
 \draw (0,0) to[R, o-o] (2,0);
\end{circuitikz}



\begin{circuitikz}
 \draw (0,0) to[R, -o] (2,0);
\end{circuitikz}



\begin{circuitikz}
 \draw (0,0) to[R, o-] (2,0);
\end{circuitikz}



\begin{circuitikz}
 \draw (0,0) to[R, *-*] (2,0);
\end{circuitikz}



\begin{circuitikz}
 \draw (0,0) to[R, -*] (2,0);
\end{circuitikz}



\begin{circuitikz}
 \draw (0,0) to[R, *-] (2,0);
\end{circuitikz}



\begin{circuitikz}
 \draw (0,0) to[R, d-d] (2,0);
\end{circuitikz}


\begin{circuitikz}
 \draw (0,0) to[R, -d] (2,0);
\end{circuitikz}



\begin{circuitikz}
 \draw (0,0) to[R, d-] (2,0);
\end{circuitikz}



\begin{circuitikz}
 \draw (0,0) to[R, o-*] (2,0);
\end{circuitikz}


\begin{circuitikz}
 \draw (0,0) to[R, *-o] (2,0);
\end{circuitikz}



\begin{circuitikz}
 \draw (0,0) to[R, o-d] (2,0);
\end{circuitikz}



\begin{circuitikz}
 \draw (0,0) to[R, d-o] (2,0);
\end{circuitikz}



\begin{circuitikz}
 \draw (0,0) to[R, *-d] (2,0);
\end{circuitikz}




\begin{circuitikz}
 \draw (0,0) to[R, d-*] (2,0);
\end{circuitikz}




\begin{circuitikz}
 \draw (0,0) to[R=1<\kilo\ohm>,
 i>_=1<\milli\ampere>, o-*] (3,0);
\end{circuitikz}




\begin{circuitikz} \draw
 (0,0) to[V=1<\volt>] (0,2)
 { to[R=1<\ohm>, color=red] (2,2) }
 to[C=1<\farad>] (2,0) -- (0,0);
\end{circuitikz}




\begin{circuitikz}[american, scale = 1.5][americanvoltages]
  \draw (0,0)
  to[V=$V_{in}$] (0,2) % The voltage source
  to[R, v^<=$R_1$] (2,2) % The resistor
  to[C, v^<=$C_1$] (2,1) % Capacitor One
  to[C, v^<=$C_2$] (2,0) %Capacitor Two
  to[L, v^<=$L_1$] (0,0); %Inductor One

  \draw[thin, <-, >=triangle 45] (1,1)node{$i_1$}  ++(-60:0.5) arc (-60:170:0.5);
\end{circuitikz}



\begin{circuitikz}[american, scale = 1.5][american voltages]
\draw (0,0)
  to[V=$V_{in}$] (0,2) % The voltage source
  to[R, v^<=$R_1$] (2,2) % The resistor
  to[C, v^<=$C_1$] (2,1) % Capacitor One
  to[C, v^<=$C_2$] (2,0) %Capacitor Two
  to[L, v^<=$L_1$] (0,0); %Inductor One

\draw [magenta,latex-] (1.0,0.5) node[above,color=magenta]{$i_a$} arc (-90:165:5mm); 

\end{circuitikz}


\begin{circuitikz}
    \draw (0,0) to [battery1=\SI{9}{V}] (3,0)
                to [pC,v=$v_C$] (6,0);
\end{circuitikz}


\begin{circuitikz}[american voltages, voltage shift=1]
    \draw (0,0) to [battery1=\SI{9}{V}] (3,0);
\end{circuitikz}


\begin{circuitikz}
    \draw (0,0) to [battery1,v>=\SI{9}{V}] (3,0)
                to [pC,v=$v_C$] (6,0) 
                to[cV, v=\SI{9}{V}] (9,0)
                ;
\end{circuitikz}


\begin{circuitikz}[american voltages, voltage shift=1]
    \draw (0,0) to [battery1,v>=\SI{9}{V}] (3,0);
\end{circuitikz}







\begin{circuitikz}[american]
	\draw (0,0) -- (0,3);
	\draw (0,3) to[R=$R_1$] (7,3);
	\draw (7,3) to[short, f>=$i_1$] (7,0);
	\draw (7,0) to[R=$R_2$,  f>=$i_2$] (2,0);
	\draw (2,0) to[battery1,v=$\varepsilon_{1}$] (0,0);
	\draw (7,0) to[short, f>=$i_3$] (7,-3);
	\draw (7,-3) to[R=$R_3$] (2,-3);
	\draw (0,-3) to[battery1,v=$\varepsilon_{2}$] (2,-3);
	\draw (0,-3) -- (0,0);
	\draw[thin, <-, >=triangle 45] (3,1.5) node{$s_1$}  ++(-40:1) arc (-40:170:1);
	\draw[thin, <-, >=triangle 45] (3,-2) node{$s_2$}  ++(-40:1) arc (-40:170:1);
\end{circuitikz}




\begin{circuitikz}[american]
	\draw (0,0) -- (0,3);
	\draw (0,3) to[R=$R_1$] (7,3);
	\draw (7,3) to[short, f>=$i_1$] (7,0);
	\draw (7,0) to[R=$R_2$,  f>=$i_2$] (2,0);
	\draw (2,0) to[battery1,v=$\varepsilon_{1}$] (0,0);
	\draw (7,0) to[short, f>=$i_3$] (7,-3);
	\draw (7,-3) to[R=$R_3$] (2,-3);
	\draw (0,-3) to[battery1,v=$\varepsilon_{2}$] (2,-3);
	\draw (0,-3) -- (0,0);
	\draw[->stealth] (1,1) to [bend left=90]  node[below, yshift=-.5cm,node font=\large]{$s_1$} (6,1);
\end{circuitikz}



\end{document}
\documentclass{article}


%\usepackage[T1]{fontenc}
%\usepackage{lipsum}
%\usepackage{tocloft}
\usepackage{fancyhdr}

\fancypagestyle{myplain}
{
  \fancyhf{}
  \renewcommand\headrulewidth{0pt}
  \renewcommand\footrulewidth{0pt}
  \fancyfoot[C]{\thepage}
}
\fancyhf{}
\fancyhf{}
\fancyhead[CO]{\nouppercase\leftmark}
\fancyhead[CE]{\hdrtitle}
\fancyhead[LE,RO]{\thepage}

\pagestyle{fancy}

\renewcommand\sectionmark[1]{\markboth{#1}{}}%don't move this

\setcounter{section}{0}


\usepackage[margin=1.2in,footskip=.25in]{geometry}
\usepackage{hyperref}
\hypersetup{
    colorlinks=true, %set true if you want colored links
    linktoc=all,     %set to all if you want both sections and subsections linked
    linkcolor=black,  %choose some color if you want links to stand out
}

\renewcommand{\baselinestretch}{1.3} 
\usepackage{xepersian}
\settextfont[Scale=1]{Vazir}
\begin{document}

\tableofcontents

\newpage

\section{فصل اول - محصول}

\subsection{نرم افزار چیست؟}

\begin{itemize}
	\item از دیدگاه مهندس نرم افزار محصولی است شامل برنامه، مستندات و داده ها
	\item از دیدگاه کاربر اطلاعاتی است که به نحوی به درد می خورد
\end{itemize}



\subsection{تفاوت نرم افزار با سخت افزار}
نرم افزار دارای ویژگی هایی است که تفاوت چشمگیری با ویژگی های سخت افزار دارد .

\begin{itemize}
	\item نرم افزار، مهندسی و بسط داده می شود و چیزی نیست که به معنای کلاسیک ساخته شود
	\begin{itemize}
		\item چه در نرم افزار و چه در سخت افزار کیفیت بالا از طریق طراحی خوب به دست می آید .
	\end{itemize}
	\item نرم افزار فرسوده نمی شود، بلکه فاسد می شود .
	\begin{itemize}
		\item نرم افزار نسبت به ناملایمات محیطی که باعث فرسایش نرم افزار می شود، نفوذ پذیر نیست .
	\end{itemize}
	\item گرچه صنعت در حال حرکت به سوی مونتاژ قطعات است، اکثر نرم افزارها همچنان به صورت سفارشی ساخته می شوند .
\end{itemize}


\subsection{قطعیت اطلاعاتی به چه معناست؟}
به معنای قابلیت پیش بینی ترتیب و زمان بندی اطلاعات است .



\subsection{انواع گروه بندی نرم افزار}

\begin{itemize}
	\item نرم افزارهای سیستمی
	\item نرم افزارهای زمان حقیقی
	\item نرم افزارهای تجاری
	\item نرم افزار های مهندسی و علمی
	\item نرم افزار های تعبیه شده
	\item نرم افزار های کامپیوترهای شخصی
	\item نرم افزارهای مبتنی بر وب
	\item نرم افزارهای هوش مصنوعی
\end{itemize}

\newpage

\subsection{نرم افزارهای سیستمی}
مجموعه ای از برنامه هاست که برای سرویس دهی به برنامه های دیگر نوشته شده اند .

 مثل : 
\begin{itemize}
	\item کامپایلر ها
	\item ویراستار ها
	\item برنامه های مدیریت فایل
\end{itemize}


\subsubsection{مشخصه های نرم افزار های سیستمی}

\begin{itemize}
	\item بر هم کنش سنگین با سخت افزار کامپیوتر
	\item استفاده سنگین توسط چند کاربر
	\item لزوم زمان بندی برای انجام کارها
	\item مدیریت فرآیند و اشتراک منابع
	\item ساختمان داده های پیچیده
	\item واسطهای خارجی چندگانه
\end{itemize}



\subsection{نرم افزارهای زمان حقیقی}

نرم افزاری که رویداد های جهان واقعی را همانطوری که رخ می دهند، نظارت ، تحلیل و کنترل می کنند .

\subsubsection{عناصر نرم افزار زمان حقیقی}

\begin{itemize}
	\item قطعه جمع آوری کننده داده ها
	\begin{itemize}
		\item اطلاعات را از محیط خارجی جمع آوری و قالب بندی می کند .
	\end{itemize}
	\item قطعه تحلیل کننده
	\begin{itemize}
		\item اطلاعات را بنا به نیاز کاربردی انتقال می دهد .
	\end{itemize}
	\item قطعه کنترل / خروجی
	\begin{itemize}
		\item به محیط خارجی پاسخ می دهد
	\end{itemize}
	\item قطعه نظارت
	\begin{itemize}
		\item همه ی قطعات دیگر را هماهنگ می کند .
	\end{itemize}
\end{itemize}



\subsection{نرم افزار های تجاری}
این نوع برنامه های کاربردی، داده های موجود را دوباره به شیوه ای سازماندهی می کند که عملیات تجاری و تصمیم گیری مدیریتی تسهیل شوند .


\subsection{نرم افزارهای تعبیه شده}
برای کنترل محصولات و سیستم های مربوط به بازارهای صنعتی و مصرفی به کار می رود .

مثل : 

\begin{itemize}
	\item صفحه کلید برای ماکروویو
	\item عملیات دیجیتال در خودرو
	\begin{itemize}
		\item کنترل سوخت
		\item صفحه نمایش داشبورد
		\item سیستم ترمز
	\end{itemize}
\end{itemize}



\subsection{بحران نرم افزار}

مشکلات مرتبط با نرم افزار را بحران می نامند .

مجموعه مشکلات موجود بر سر راه بسط نرم افزار های کامپیوتری شامل :

\begin{itemize}
	\item مشکلات مرتبط با چگونگی بسط نرم افزار
	\item چگونگی پشتیبانی حجم رو به رشدی از نرم افزار های موجود
	\item چگونگی همگام ماندن با تقاضای فزاینده برای نرم افزار های جدید .
\end{itemize}



\subsection{اسطوره های نرم افزاری در انواع سطوح}

\begin{itemize}
	\item اسطوره های مدیریتی
	\item اسطوره های مشتریان
	\item اسطوره های سازندگان
\end{itemize}

\newpage

\subsection{اسطوره های مدیریتی}

\subsubsection{اسطوره}
ما کتابی داریم که شامل استاندارد ها و روال های لازم برای ساخت نرم افزار هاست و آنچه را که افراد من باید بدانند در اختیارشان قرار می دهد .


\subsubsection{واقعیت}
همچین کتابی وجود ندارد، حتی اگر وجود داشته باشد، بروز و کامل نیست .





\subsubsection{اسطوره}
افراد من ابزارهای نرم افزارسازی حرفه ای را دارند و جدیدترین کامپیوتر ها را برای آنها خواهیم خرید .


\subsubsection{واقعیت}
برای ساخت نرم افزار با کیفیت، فقط داشتن سخت افزار خوب کافی نیست بلکه داشتن نرم افزار های کمکی نیز اهمیت بالایی دارند


\subsubsection{اسطوره}
اگر از برنامه عقب بیفتیم، می توانیم بر تعداد برنامه نویسان بیفزاییم و عقب افتادگی را جبران کنیم .


\subsubsection{واقعیت}
با افزودن افراد جدید به تیم ساخت نرم افزار بر میزان تاخیر ساخت آن افزوده می شود زیرا با از راه رسیدن افراد جدید، افراد قدیمی باید زمانی را صرف آموزش آنها کنند، در نتیجه زمانی که باید صرف کار روی نرم افزار شود، هدر می رود .



\subsection{اسطوره های مشتریان}

\subsubsection{اسطوره}
نیازهای پروژه پیوسته در حال تغییر است و این تغییرات را به راحتی می توان در نرم افزار جای داد زیرا نرم افزار انعطاف پذیر است .


\subsubsection{واقعیت}
تاثیر تغییر به زمان اعمال تغییر بستگی دارد، 

درخواستهای اولیه به تغییر را به راحتی می توان پاسخ داد .

هنگامی که تغییرات در اثنای طراحی نرم افزار درخواست می شوند ، هزینه ها بالاتر می رود .

تغییر ، در صورتی که نرم افزار به مرحله استفاده رسید، هزینه ها به مراتب بالاتر می رود .


\subsection{اسطوره های سازندگان}


\subsubsection{اسطوره}
هنگامی که برنامه را نوشتیم و برنامه کار کرد دیگر کار تمام است .


\subsubsection{واقعیت}
تجربه نشان داده است که بین 60 تا 80 درصد کوششهای روی نرم افزارها ، پس از نخستین بار تحویل آنها به مشتری صورت می پذیرد .

\subsubsection{اسطوره}
تا هنگامی که برنامه را اجرا نکردم، راهی برای ارزیابی کیفیت آن ندارم .


\subsubsection{واقعیت}
مرور تکنیکی رسمی از زمان آغاز پروژه قابل اجراست .



\subsubsection{اسطوره}
تنها چیز قابل تحویل برای یک پروژه موفق، برنامه ای است که کار کند .


\subsubsection{واقعیت}
نرم افزار شامل عناصر فراوان است، برنامه ای که کار می کند فقط بخشی از پیکربندی نرم افزار است .


\section{فصل دوم - فرآیند}


\subsection{فرآیند چیست؟}
یک نقشه ی راه که در ایجاد نتیجه ای با کیفیت و به موقع شما را یاری می کند .

\subsection{دلیل اهمیت فرآیند}
زیرا باعث ثبات، کنترل و سازماندهی فعالیتی می شود .


\subsection{حاصل فرآیند چیست؟}
حاصل کار فرآیند، برنامه ها ، مستندات و داده ها می باشد


\subsection{معیارهای ارزیابی فرآیند}

\begin{itemize}
	\item کیفیت
	\item به موقع بودن
	\item کارایی دراز مدت محصول
\end{itemize}
بهترین ملاک ها برای بازدهی فرآیند است .


\subsection{آیا فرآیند مترادف با مهندسی نرم افزار است؟}
پاسخ این است : بلی و خیر

فرآیند نرم افزار روش مهندسی را مشخص می کند .

ولی مهندسی نرم افزار شامل فناوری هایی می شود که فرآیند را تشکیل می دهند .



\subsection{تعریف مهندسی نرم افزار بر اساس فریتز باور؟}
مهندسی نرم افزار عبارت است از وضع اصول مهندسی به جا و مناسب و استفاده از آنها برای به دست آوردن یک نرم افزار مقرون به صرفه که قابل اطمینان بوده و به طرز کارآمدی عمل کند .



\subsection{لایه های مهندسی نرم افزار}

\begin{itemize}
	\item ابزارها
	\item روشها
	\item فرآیندها
	\item تاکید بر کیفیت
\end{itemize}


\subsection{زمینه های فرآیند کلیدی یا KPA را تعریف کنید ؟}
زمینه های فرآیند کلیدی، پایه ای برای کنترل مدیریتی پروژه های نرم افزاری تشکیل داده و بستری برای : 
\begin{itemize}
	\item اعمال روشهای فنی
	\item تولید محصولات کاری
	\item تعیین مراحل
	\item حصول اطمینان از کیفیت
	\item مدیریت مناسب تغییرات
\end{itemize}
ایجاد می کنند .



\subsection{روشهای مهندسی نرم افزار}
روشهای مهندسی نرم افزار، شیوه های فنی برای ساخت نرم افزار متکی بر یک مجموعه اصول بنیادی است که بر تمام زمینه های فناوری حاکم است .

روشها شامل آرایه وسیعی از وظایف است :

\begin{itemize}
	\item تحلیل خواسته ها
	\item طراحی
	\item ساخت برنامه ها
	\item آزمایش
	\item پشتیبانی
\end{itemize}




\subsection{مراحل مهندسی}

\begin{itemize}
	\item تحلیل
	\item طراحی
	\item ساخت
	\item بررسی
	\item مدیریت نهاد های فنی و اجتماعی
\end{itemize}




\subsection{فاز های مهندسی نرم افزار}

\begin{itemize}
	\item فاز تعریف
	\item فاز توسعه
	\item فاز پشتیبانی
\end{itemize}



\subsection{ فاز تعریف}

فاز تعریف بر چیستی تاکید دارد . در فاز تعریف، مهندس نرم افزار تعیین می کند :

\begin{itemize}
	\item  چه اطلاعاتی باید پردازش شود .
	\item کدام عمل و کارایی مطلوب است .
	\item چه رفتارهای سیستمی قابل انتظار است .
	\item چه رابطه هایی را می توان برقرار کرد .
	\item چه محدودیتهایی در طراحی وجود دارد
	\item در نهایت خواسته های کلیدی سیستم و نرم افزار تشخیص داده می شود .
\end{itemize}


\subsection{3 کار عمده در فاز تعریف}

\begin{itemize}
	\item مهندسی اطلاعات یا سیستم
	\item طرح ریزی پروژه نرم افزار
	\item تحلیل خواسته ها
\end{itemize}



\subsection{ فاز توسعه}

فاز توسعه بر چگونگی تاکید دارد .

در فاز توسعه مهندس نرم افزار می کوشد تعیین کند داده ها چه ساختاری داشته باشند .


\subsection{3 وظیفه ی فنی در فاز توسعه}

\begin{itemize}
	\item طراحی نرم افزار
	\item تولید دستور ها
	\item آزمایش نرم افزار
\end{itemize}


\subsection{ فاز پشتیبانی}

فاز پشتیبانی بر تغییرات تاکید دارد .

این تغییرات  ممکن است شامل تصحیحاتی در جهت تکامل نرم افزار باشند یا تغییراتی که ناشی از تغییر خواسته های مشتریان است . 

\subsection{4 نوع تغییر در فاز پشتیبانی}

\begin{itemize}
	\item تصحیح
	\begin{itemize}
		\item نگهداری تصحیحی، نرم افزار را در جهت تصحیح معایب تغییر می دهد .
	\end{itemize}
	\item تطابق
	\begin{itemize}
		\item نگهداری تطابقی، منجر به انجام اصلاحاتی در نرم افزار می شود که پاسخگوی تغییرات محیط خارجی شود
	\end{itemize}
	\item بهبود
	\begin{itemize}
		\item نگهداری بهبودی، نرم افزار را به مافوق خواسته های اولیه ی خود سوق می دهد .
	\end{itemize}
	\item جلوگیری
	\begin{itemize}
		\item نگهداری پیشگیرانه، تغییراتی در برنامه کامپیوتری اعمال می کند که بتوان برنامه را راحت تر تصحیح کرد، تطابق داد و بهبود بخشید .
	\end{itemize}
\end{itemize}



\subsection{اعمال چتری بر کل فاز های مهندسی نرم افزار}

اعمال چتری در سرتاسر فرآیند نرم افزار اجرا می شوند .

\begin{itemize}
	\item کنترل و دنبال کردن پروژه نرم افزاری
	\item مرور فنی رسمی
	\item تضمین کیفیت نرم افزار
	\item مدیریت پیکربندی نرم افزار
	\item تولید و تهیه مستندات
	\item مدیریت قابل استفاده مجدد
	\item سنجش
	\item مدیریت ریسک
\end{itemize}


\subsection{پنج سطح بلوغ فرآیند}


\begin{itemize}
	\item سطح 1 : آغاز
	\item سطح 2 : تکرار پذیر
	\item سطح 3 : تعریف شده
	\item سطح 4 : مدیریت شده
	\item سطح 5 : بهینه سازی
\end{itemize}







\subsection{KPA چیست}

KPA ها آن دسته از وظایف مهندسی نرم افزار را توصیف می کند که باید برای رسیدن به حد مطلوبی از عملکرد در یک سطح معین به انجام برسد .


\subsection{ویژگی های KPA}

\begin{itemize}
	\item اهداف
	\item وظایف
	\item تواناییها
	\item روشهای نظارت بر پیاده سازی
	\item روشهای بازبینی پیاده سازی
\end{itemize}


\subsection{ تعریف مدل فرآیند نرم افزار}

برای حل مسائل در یک مجموعه صنعتی، یک مهندس نرم افزار باید یک راهبرد توسعه تعیین کند که در برگیرنده لایه های فرآیند ، روشها و ابزارها و فازهای مهندسی نرم افزار ( تعریف، توسعه، پشتیبانی ) باشد، این راهبرد را مدل فرآیند یا الگوی مهندسی نرم افزار می نامند .

یک مدل فرآیند برای مهندسی نرم افزار بر اساس :

\begin{itemize}
	\item ماهیت پروژه و نوع کاربرد
	\item روشها و ابزارهای مورد استفاده
	\item کنترلها و قطعات قابل تحویل مورد نیاز
\end{itemize}

انتخاب می شود .

\subsection{مدلهای فرآیند نرم افزار}

\begin{itemize}
	\item مدل ترتیبی خطی
	\item مدل ساخت نمونه اولیه
	\item مدل RAD
\end{itemize}



\subsection{  مدل ترتیبی خطی}

این مدل که مدل آبشار یا چرخه ی حیات کلاسیک خوانده می شود، یک روش سیستماتیک و ترتیبی برای بسط نرم افزار پیشنهاد می کند که در سطح سیستمی آغاز می شود و به 

\begin{itemize}
	\item تحلیل
	\item طراحی
	\item کدنویسی
	\item آزمایش
	\item پشتیبانی
\end{itemize}

پیشروی می کند .


\newpage

\subsection{فعالیت های مدل ترتیبی خطی}

\begin{itemize}
	\item مهندسی سیستم / اطلاعات و مدلسازی
		\begin{itemize}
			\item تحلیل و مهندسی سیستم شامل جمع آوری خواسته ها در سطح سیستم و طراحی سطح بالاست .
			\item مهندسی اطلاعات شامل جمع آوری خواسته ها در سطح تجارت راهبردی و در سطح زمینه ی کاری است .
		\end{itemize}
	\item تحلیل خواسته های نرم افزار
	\item طراحی
	
		تمرکز طراحی بر روی چهار صفات از برنامه تاکید دارد :
		\begin{itemize}
		\item ساختمان داده ها
		\item معماری نرم افزار
		\item نمایش واسطها
		\item جزئیات رویه ای
		\end{itemize}
	\item تولید کد
	\item آزمایش
	\item پشتیبانی
\end{itemize}


\subsection{مشکلاتی که به هنگام اجرای مدل ترتیبی خطی به وجود می آید ؟}

\begin{itemize}
	\item با پیش رفتن تیم پروژه، ممکن است تغییرات باعث ایجاد سردرگمی شود .
	\item مدل ترتیبی خطی به بیان واضح نیاز دارد و غالباً برای مشتری دشوار است که همه ی نیازهای خود را به وضوح بیان کند .
	\item مشتری باید حوصله داشته باشد زیرا تا آخرین روز پروژه هیچ نسخه ای از برنامه در دسترس او قرار نخواهد گرفت .
\end{itemize}



\subsection{مدل ساخت نمونه اولیه}

غالباً مشتری یک مجموعه اهداف برای نرم افزار تعیین می کند، ولی جزئیات ورودی ها و یا خواسته های خروجی را مشخص نمی کند یا سازنده از کارایی الگوریتم و یا تطابق با سیستم عامل مطمئن نباشد، در این شرایط الگوی ساخت نمونه اولیه ممکن است روش خوبی باشد .

الگوی ساخت نمونه اولیه با جمع آوری خواسته ها آغاز می شود و پس از ملاقات مشتری و سازنده و تعیین اهداف کلی نرم افزار با یک طراحی سریع صورت میگیرد .


در طراحی سریع، هدف اصلی ارائه ی آن دسته از ویژگی های نرم افزار است که به چشم مشتری می آیند ( مثل روشهای ورود اطلاعات و فرمت های خروجی )

طراحی سریع منجر به ساخت یک نمونه اولیه می شود و نمونه اولیه به عنوان راهکاری برای تشخیص خواسته های نرم افزار عمل می کند .


\subsection{مشکلات ساخت نمونه اولیه}

\begin{itemize}
	\item معمولاً مشتری تقاضا می کند که با چند ترمیم جزئی این نمونه اولیه به محصول نهایی تبدیل شود بدون اینکه کیفیت کلی نرم افزار و قابلیت نگهداری مد نظر باشد و اکثر اوقات مدیریت هم قبول می کند
	\item سازندگان ممکن است برای ساختن هرچه سریعتر نمونه اولیه، در پیاده سازی آن کوتاهی کنند و دست به انتخاب های نامناسب در پیاده سازی نرم افزار بزنند .
\end{itemize}






\subsection{فازهای مدل RAD}

\begin{itemize}
	\item مدلسازی تجاری
	\item مدلسازی داده ای
	\item مدلسازی فرآیند
	\item تولید برنامه کاربردی
	\item آزمایش و توان عملیاتی
\end{itemize}


\subsection{نتایج منفی مدل RAD}


\begin{itemize}
	\item برای ایجاد تعداد مناسبی از تیم های RAD به منابع انسانی کافی نیاز دارد
	\item نیاز به سازندگان و مشتریانی دارد که برای کامل کردن سیستم به یک چارچوب زمانی فشرده تر، معتقد هستند
	\item اگر سیستمی را نتوان به طور مناسب تقسیم بندی کرد، ساخت مولفه های لازم مشکل آفرین خواهد شد
	\item وقتی احتمال بروز خطرات فنی بالا باشد، مدل مناسبی نیست .
\end{itemize}


\subsection{انواع مدلهای تکاملی فرآیند نرم افزار}

\begin{itemize}
	\item مدل گام به گام
	\item مدل مارپیچی
	\item مدل مارپیچی winwin
	\item مدل بسط همزمان
	\item بسط مبتنی بر مولفه ها
	\item مدل روشهای رسمی
	\item تکنیکهای نسل چهارم
\end{itemize}



\subsection{مقایسه ی مدل ترتیبی خطی و مدل های تکاملی}

مدل ترتیبی خطی برای توسعه ی آسان طراحی شده است و در این مدل فرض می شود که یک سیستم کامل پس از طی شدن یک ترتیب خطی کامل شده است .

مدل ساخت نمونه اولیه برای کمک به مشتری در شناخت خواسته ها طراحی می شود و این مدل اساساً برای تحویل یک سیستم آماده و کامل طراحی نشده است .

مدل های تکاملی ، تکراری هستند . این مدل ها به شیوه ای طراحی می شوند که مهندس نرم افزار را قادر سازند تا نسخه هایی از نرم افزار را توسعه دهند که هر یک از قبلی کاملتر باشند .



\subsection{مدل گام به گام}

مدل گام به گام عناصر مدل ترتیبی خطی را با فلسفه ی تکراری مدل ساخت نمونه اولیه تلفیق می کند .

هر ترتیبی خطی یک قطعه قابل تحویلی از نرم افزار ( گام ) را ارائه می دهد .


\subsection{مدل مارپیچی}
مدل مارپیچی ، ماهیت تکراری مدل ساخت نمونه اولیه را با جنبه های کنترلی و سیستماتیک مدل ترتیبی خطی تلفیق می کند .

این مدل پتانسیل لازم برای بسط سریع نسخه های تکاملی نرم افزار را دارا می باشد . 

با استفاده از مدل مارپیچی، نرم افزار به صورت یک سری نگارشهای تکاملی توسعه می یابد .

طی نخستین تکرار ها، نگارش تکاملی ممکن است یک مدل کاغذی یا یک نمونه اولیه باشد، طی تکرارهای بعدی، هر بار نسخه ی کاملتری از سیستم، تولید می شود .

مدل مارپیچی به چند فعالیت چارچوبی تقسیم می شود که نواحی کاری نامیده می شوند .

یک مدل مارپیچی حاوی 6 ناحیه ی کاری است .

\subsection{6 ناحیه ی کاری مدل مارپیچی}

\begin{itemize}
	\item ارتباط با مشتری
	\item طرح ریزی
	\item تحلیل ریسک
	\item مهندسی
	\item ساخت و ارائه
	\item ارزیابی مشتری
\end{itemize}

\newpage

\subsection{مدل مارپیچی winwin}

در این مدل، مشتری و سازنده وارد یک فرآیند گفتگو و بحث می شوند و در این فرآیند ممکن است از مشتری خواسته شود تا میان عملکرد، کارایی و ویژگی های دیگر سیستم یا محصول در مقابل هزینه و زمان بازیابی موازنه برقرار کند تا  یک نتیجه برد برد حاصل شود .

یعنی مشتری با دستیابی به محصول یا سیستمی که واجد اکثر نیازهای اوست برنده می شود و سازنده با کار کردن در چارچوب مهلت و بودجه ای واقعی و قابل حصول برنده می شود .


\subsection{سه مرحله فرآیند در مدل winwin را نام ببرید ؟ 
( نقاط لنگر گاه )
}


\begin{itemize}
	\item نخستین نقطه ی لنگرگاهی اهداف چرخه ی حیات نام دارد
	\item دومین نقطه ی لنگرگاهی معماری چرخه ی حیات نام دارد
	\item سومین نقطه ی لنگرگاهی قابلیت عملیاتی اولیه است .
\end{itemize}


\subsection{مدل بسط همزمان را توضیح دهید؟}
مدل فرآیند همزمانی را می توان به عنوان یک سری فعالیت هایی فنی اصلی، وظایف و حالتهای مرتبط با آن نمایش داد .



\subsection{بسط مبتنی بر مولفه ها}
فناوری شی گرا چارچوبی فنی برای مدل فرآیند مبتنی بر مولفه جهت مهندسی نرم افزار فراهم می کند .

مدل بسط مبتنی بر مولفه ها بسیاری از ویژگی های مدل مارپیچی را در خود دارد .

مدل بسط مبتنی بر مولفه ها برنامه های کاربردی را با استفاده از مولفه های نرم افزاری بسته بندی شده می سازد .


\subsection{مدل روشهای رسمی}
مدل روشهای رسمی شامل مجموعه ای از فعالیتها است که به مشخص کردن ریاضی و رسمی نرم افزار  کامپیوتری منجر می شود .



\subsection{علم مهندسی نرم افزار را تعریف کنید ؟}
مهندسی نرم افزار علمی است که فرآیند، روشها و ابزار ها را با هم تلفیق می سازد تا نرم افزار کامپیوتر را پدید آورد . 




\subsection{اعمال مباحثاتی در مدل مارپیچی winwin}

\begin{itemize}
	\item شناسایی واگذارنده کلیدی سیستم یا زیر سیستم
	\item تعیین شرایط برد واگذارنده
	\item بحث و گفتگو درباره شرایط برد واگذارنده
\end{itemize}



\section{فصل سوم - مفاهیم مدیریت پروژه}

\subsection{در مدیریت پروژه های نرم افزاری بر چند نکته تاکید می شود ؟}

چهار نکته : 

\begin{itemize}
	\item افراد
	\item محصول
	\item فرآیند
	\item پروژه
\end{itemize}




\subsection{مدل بلوغ مدیریت افراد، چه زمینه های کلیدی را برای نرم افزار نویسان تعیین می کند؟}

\begin{itemize}
	\item یافتن افراد جدید
	\item انتخاب
	\item مدیریت اجرا
	\item آموزش
	\item جبران
	\item توسعه حرفه ای
	\item سازماندهی و طراحی کار
	\item توسعه تیمی و فرهنگی
\end{itemize}



\subsection{بازیگران فرآیند نرم افزار}


\begin{itemize}
	\item مدیران ارشد
	\item مدیران پروژه
	\item سازندگان
	\item مشتریان
	\item کاربران نهایی
\end{itemize}



\subsection{مدل MOI برای رهبری}



\begin{itemize}
	\item انگیزش
	\item سازماندهی
	\item ایده ها و نوآوری ها
\end{itemize}




\subsection{چهار نکته ی کلیدی در مدیریت پروژه}


\begin{itemize}
	\item حل مسئله
	\item هویت مدیریتی
	\item نیل به اهداف
	\item تاثیر و ساخت تیم
\end{itemize}


\subsection{ساختار بهترین تیم به چه عواملی بستگی دارد؟}


\begin{itemize}
	\item به شیوه ی مدیریت سازمان
	\item به تعداد افراد تشکیل دهنده تیم
	\item سطوح مهارتی آنان
	\item میزان کلی دشواری مسئله
\end{itemize}





\subsection{سازماندهی کلی برای یک تیم بر اساس نظریه ی مانتی}


\begin{itemize}
	\item تمرکز زدایی دموکراتیک ( DD )
	\begin{itemize}
		\item تیم مهندسی نرم افزار فاقد رهبری دائمی است . تصمیم گیری درباره ی مشکلات و روش کار از طریق اجماع گروه صورت می پذیرد .
	\end{itemize}
	\item تمرکز زدایی کنترل شده ( CD )
	\begin{itemize}
		\item تیم مهندسی نرم افزار دارای رهبری مشخص است که وظایف معینی را هماهنگ می کند ، حل مشکلات از طریق فعالیت های گروهی است . 
	\end{itemize}
	\item تمرکز کنترل شده ( CC )
	\begin{itemize}
		\item هماهنگ سازی داخلی تیم و حل مشکلات سطح بالا بر عهده مدیر تیم است .
	\end{itemize}
\end{itemize}


\newpage

\subsection{بر اساس نظریه ی مانتی چه عواملی را باید هنگام طرح ریزی ساختار تیم های مهندسی نرم افزار در نظر گرفت؟}


\begin{itemize}
	\item دشواری مسئله ای که قرار است حل شود
	\item اندازه برنامه های حاصل بر حسب تعداد خطوط کد یا تعداد نقاط تابع
	\item زمان کنار هم ماندن اعضای تیم
	\item میزان قابلیت پیمانه ای مسئله ( تقسیم  پذیری مسئله به اجزای کوچکتر )
	\item کیفیت و قابلیت اطمینان مورد نیاز برای سیستمی که قرار است ساخته شود
	\item قطعیت تاریخ تحویل
	\item میزان ارتباطات مورد نیاز برای پروژه
\end{itemize}





\subsection{بر اساس نظریه ی کنستانتین، چهار الگوی سازمانی برای تیمهای مهندسی نرم افزار را توضیح دهید؟}


\begin{itemize}
	\item الگوی بسته
	\begin{itemize}
		\item هنگام ساخت نرم افزارهایی خوب عمل می کند که نیاز به کارهای روتین دارد ولی برای نوآوری و کارهای جدید احتمال موفقیت کمتر است .
	\end{itemize}
	\item الگوی تصادفی
	\begin{itemize}
		\item به ظرفیت تک تک افراد وابسته است . در صورت نیاز به کشفیات فنی یا نوآوری، تیمی که از الگوی تصادفی پیروی می کند بهترین نتیجه را می دهد ولی در صورتی که کارکردی منظم مورد نیاز باشد، چنین تیمی به دردسر می افتد .
	\end{itemize}
	\item الگوی باز
	\begin{itemize}
		\item در الگوی باز هدف بر آن است که در عین دستیابی به نتایج الگوی بسته، به نوآوری های الگوی تصادفی نیز برسیم . در این الگو کارها از طریق همکاری و ارتباطات سنگین و تصمیم گیری های مبتنی بر اجماع انجام می پذیرد . الگوی باز برای حل مسائل پیچیده مناسب است ولی ممکن است بازدهی تیم های دیگر را نداشته باشد .
	\end{itemize}
	\item الگوی همزمان
	\begin{itemize}
		\item اعضای تیم دسته بندی می شوند و هرکدام روی بخشی از مسئله یا پروژه کار می کنند . ارتباطات بین دسته ها بسیار کم می باشد .
	\end{itemize}
\end{itemize}




\subsection{برای دستیابی به تیمی با کارایی بالا}


\begin{itemize}
	\item اعضای تیم باید به یکدیگر اطمینان داشته باشند
	\item توزیع مهارت ها مناسب حل مسئله باشد
	\item اگر قرار است یکپارچگی حفظ شود، افراد خودکامه باید از تیم بروند 
\end{itemize}




\subsection{تیم ژل شده چه نوع تیمی است؟}

\begin{itemize}
	\item تیم ژل شده گروهی از افراد است، که چنان به هم پیوند خورده اند که کلیت آنها بزرگتر از حاصل جمع تک تک آنهاست 
	\item با ژل شدن یک تیم احتمال موفقیت بالا می رود ، نیازی به مدیریت آنها به شیوه های سنتی نیست، و قطعاً نیازی به انگیزش ندارند، آنها تکانه ی لازم را دارند .
\end{itemize}





\subsection{پنج عامل مسمومیت تیم از دیدگاه جکمن}


\begin{itemize}
	\item فضای کاری آشفته که  در آن انرژی اعضای تیم هدر می رود، و توجه آنان را از اهداف کاری دور می کند
	\item ناراحتی زیاد ناشی از عوامل فنی، کاری و پرسنلی
	\item رویه های قطعه قطعه شده و با هماهنگی ضعیف، یا مدل فرآیندی که خوب مشخص نشده باشد
	\item تعریف ناواضح نقشها که منجر به فقدان مسئولیت پذیری می شود
	\item قرار گرفتن پیوسته در معرض شکست که منجر به از دست رفتن اعتماد به نفس می شود
\end{itemize}



\subsection{یک عامل مسمومیت تیم دیگر}
علاوه بر 5 سمی که جکمن توصیف می کند، تیم نرم افزاری غالباً با اختلاف تمایلات انسانی در اعضا نیز مشکل دارد، 

ذکر این نکته اهمیت دارد که شناخت تفاوت های انسانی نخستین گام در راه ایجاد تیمهایی است که ژل می شوند .




\newpage


\subsection{مجموعه ای از تکنیک های هماهنگ سازی پروژه از دیدگاه کراول و استریتر}



\begin{itemize}
	\item روشهای رسمی غیر شخصی
	\begin{itemize}
		\item شامل  : 
		\begin{itemize}
			\item مستندات مهندسی نرم افزار
			\item یادداشتهای فنی
			\item نقاط عطف پروژه
			\item ابزارهای کنترل پروژه
			\item  برنامه های زمانبندی
			\item گزارش خطاها
		\end{itemize}
		  می شود 
	\end{itemize}
	\item روشهای رسمی بین اشخاص
	\begin{itemize}
		\item شامل جلسات مرور وضعیت و طراحی و بازرسی کد می شود .
	\end{itemize}
	\item روشهای غیر رسمی بین اشخاص
	\begin{itemize}
		\item شامل جلسات گروهی برای پخش اطلاعات و حل مسئله و جمع آوری خواسته های پرسنل
	\end{itemize}
	\item ارتباط الکترونیکی
	\begin{itemize}
		\item شامل پست الکترونیکی، سیستم های ویدیو کنفرانس و . . .
	\end{itemize}
	\item شبکه ی بین اشخاص
	\begin{itemize}
		\item بحث های غیر رسمی با اعضای تیم و آنان که در خارج از پروژه هستند ولی ممکن است دارای تجربه یا دیدگاهی باشند که می تواند به اعضای تیم کمک کند .
	\end{itemize}
\end{itemize}




\subsection{دامنه کاربرد نرم افزار}
نخستین فعالیت مدیریت نرم افزار، تعیین دامنه کاربرد نرم افزار است .

دامنه کاربرد نرم افزار با پاسخ دادن به چه پرسش هایی مشخص می شود .

\begin{itemize}
	\item محیط
	\item اهداف اطلاعات
	\item عملکرد و کارایی
\end{itemize}





\subsection{مدیر پروژه با در نظر گرفتن کدام موارد، مدل فرآیند نرم افزار را انتخاب می کند ؟}


\begin{itemize}
	\item مشتریانی که محصول را درخواست کرده اند و کسانی که کار را انجام می دهند 
	\item ویژگی های خود محصول
	\item محیط پروژه که تیم نرم افزاری در آن کار می کند
\end{itemize}





\subsection{در یک پروژه کوچک، برای فعالیت ارتباط با مشتری نیاز به چه وظایف کاری می باشد ؟}



\begin{itemize}
	\item تهیه ی لیستی از مسائل مربوط به روشن کردن امور
	\item ملاقات با مشتری برای پرداختن به مسائل مربوط به روشن کردن امور
	\item تهیه بیان حوزه
	\item مرور بیان حوزه با همه دست اندرکاران
	\item اصلاح بیان حوزه در صورت نیاز
\end{itemize}



\subsection{در یک پروژه پیچیده و بزرگ، برای فعالیت ارتباط با مشتری نیاز به چه وظایف کاری می باشد ؟}

\begin{itemize}
	\item بازبینی درخواست مشتری
	\item طرح ریزی و زمانبندی یک ملاقات رسم با مشتری
	\item پژوهش برای تعیین راهکار پیشنهادی و روشهای موجود
	\item آماده سازی یک سند کاری و برنامه برای ملاقات رسمی
	\item اجرای ملاقات
	\item تهیه ی مشخصات جزئی که ویژگی های رفتاری، عملیاتی و داده ای نرم افزار را منعکس کند
	\item بازبینی هر مشخصات جزئی برای تصحیح، سازگاری و عدم ابهام
	\item مونتاژ مشخصات جزئی در مستندات تعیین حوزه
	\item بازبینی مستندات تعیین حوزه
	\item اصلاح مستندات تعیین حوزه بنا به نیاز
\end{itemize}


\newpage

\subsection{علائم در خطر بودن پروژه از منظر جان ریل}


\begin{itemize}
	\item افراد نرم افزاری نیازهای مشتری را درک نمیکنند
	\item حوزه محصول خوب تعیین نشده است
	\item تغییرات خوب مدیریت نمی شود
	\item فناوری انتخاب شده تغییر می کند
	\item نیازهای تجاری تغییر می کند، یا خوب تعریف نمی شوند
	\item مهلت ها واقع بینانه نیست
	\item کاربران مقاومت می کنند
	\item حمایت مالی از بین می رود یا هرگز به طور مناسب به دست نیامده است
	\item تیم پروژه فاقد افرادی با مهارتهای لازم است
	\item مدیران و سازندگان از بهترین کوشش ها و دروس فراگرفته شده پرهیز می کنند
\end{itemize}



\subsection{قاعده 90-90}
90\% اول یک سیستم، 90\% زمان و کار را به خود اختصاص می دهد 


\subsection{روش 5 بخشی جان ریل مبتنی بر عقل سلیم برای پروژه های نرم افزاری}

\begin{itemize}
	\item در آغاز کار درست گام بردارید
	\item تکانه را حفظ کنید
	\item پیشرفت کار را دنبال کنید
	\item هوشمندانه تصمیم گیری کنید
	\item تحلیلی نوگرایانه ارائه دهید
\end{itemize}


\newpage


\subsection{اصل $W5HH$}

روش پیشنهادی بوهم، که اهداف پروژه، نقاط عطف و زمانبندی ها، مسئولیت ها، روش فنی و مدیریتی و منابع مورد نیاز را در بر می گیرد که پس از یک سری پرسش ها منجر به تعریف ویژگی های کلیدی پروژه و طرح پروژه می شود .

\begin{itemize}
	\item این سیستم چرا توسعه می یابد؟
	\item چه چیزی و در چه زمانی انجام خواهد شد؟
	\item چه کسی مسئول یک عمل است؟
	\item از لحاظ سازمانی چه جایگاهی دارند؟
	\item کار از نظر مدیریتی و فنی چگونه انجام خواهد شد؟
	\item از هر منبع چه میزان لازم است؟
\end{itemize}


اصل $W5HH$ برای همه ی پروژه های نرم افزاری، بدون توجه به اندازه و پیچیدگی آنها، قابل استفاده است .


\subsection{مجموعه پرسش ها برای اینکه سازمانی تعیین کند که آیا پروژه ای اعمال بحرانی را پیاده سازی کرده است ؟}


\begin{itemize}
	\item مدیریت ریسک رسمی
	\begin{itemize}
		\item ریسک های اصلی که پروژه را تهدید می کنند، کدام است ؟
	\end{itemize}
	\item برآورد زمانبندی و هزینه ی تجربی
	\begin{itemize}
		\item برآورد هزینه و زمان ساخت نرم افزار چقدر است ؟
	\end{itemize}
	\item مدیریت پروژه مبتنی بر معیارها
	\begin{itemize}
		\item معیارهایی که مشکلات تکاملی را نشان می دهد چیست ؟
	\end{itemize}
	\item دنبال کردن ارزش به دست آمده
	\begin{itemize}
		\item آیا ارزش های به دست آمده را گزارش می کنیم ؟
	\end{itemize}
	\item دنبال کردن نقایص در برابر اهداف کیفیتی
	\begin{itemize}
		\item آیا تعداد نقایص در هر بازرسی را پیگیری و گزارش می کنید ؟
	\end{itemize}
	\item مدیریت برنامه آگاهی از افراد
	\begin{itemize}
		\item بازدهی کارمندان در توسعه ی نرم افزار چقدر است ؟
	\end{itemize}
\end{itemize}



\subsection{چهار مورد که تاثیری اساسی بر مدیریت پروژه های نرم افزاری دارند؟}

\begin{itemize}
	\item افراد
	\item محصول
	\item فرآیند
	\item پروژه
\end{itemize}



\section{فصل چهارم - معیارهای پروژه و فرآیند نرم افزار}


\subsection{اندازه گیری در سرتاسر پروژه چه نتایجی دارد؟}


\begin{itemize}
	\item کمک به برآورد
	\item کنترل کیفیت
	\item ارزیابی بهره وری
	\item کنترل پروژه
\end{itemize}



\subsection{چهار دلیل برای اندازه گیری فرآیندهای نرم افزاری، محصولات و منابع را نام ببرید؟}

\begin{itemize}
	\item مشخص کردن ویژگی ها
	\item ارزیابی کردن
	\item پیش بینی کردن
	\item بهبود بخشیدن
\end{itemize}


\newpage


\subsection{موازین، معیارها و شاخصها}

میزان عبارت از کمیتی است که نشانگر حد، مقدار، ابعاد، ظرفیت یا اندازه صفتی از محصول یا فرآیند است .

معیار چنین تعریف می شود :

میزان کمّی از حدی که یک سیستم، مولفه یا فرآیند می تواند دارای یک صفت مفروض باشد .

شاخص، معیار یا ترکیبی از معیارهاست که درکی از فرآیند نرم افزار، پروژه نرم افزاری یا خود محصول به دست می دهد .



\subsection{معیار}


معیار ها را باید طوری جمع آوری کرد که شاخص های فرآیند و محصول را بتوان کشف کرد، 

شاخصهای فرآیند به سازمان مهندسی نرم افزار این امکان را می دهند که از بازدهی فرآیند موجود دیدی درست به دست آورند ،

مدیران و سازندگان را قادر می سازند تا بتواند ارزیابی کنند چه چیزهایی عملی هستند و چه چیزهایی نیستند .

معیارهای فرآیند از همه ی پروژه ها و در مدت زمان طولانی جمع آوری می شوند . هدف از استفاده از آنها فراهم آوردن شاخصهایی است که منجر به بهبود فرآیند نرم افزار در بلند مدت می شود .

شاخصهای پروژه مدیر پروژه را قادر می سازد تا :

\begin{itemize}
	\item به وضعیت پروژه در حال پیشرفت دست پیدا کند
	\item خطرات بالقوه را دنبال کند
	\item نواحی مشکل آفرین را پیش از بحرانی شدن آنها کشف کند
	\item جریان کار یا وظایف را تنظیم کند
	\item توانایی تیم پروژه در کنترل کیفیت محصولات نرم افزار را ارزیابی کند
\end{itemize}



\subsection{بهبود بخشیدن به فرآیند}

تنها راه موجه برای بهبود بخشیدن به هر فرآیندی، اندازه گیری صفات مشخصی از آن فرآیند است ، توسعه یک مجموعه معیارهای با معنی بر اساس این صفات ، و سپس استفاده از این معیارها برای فراهم آوردن شاخصهایی است، که منجر به راهبردی برای بهبود کار می شوند .


\newpage

\subsection{بازدهی یک فرآیند نرم افزار را توضیح دهید؟}
بازدهی یک فرآیند نرم افزار را به طور غیر مستقیم اندازه گیری می کنیم .

یعنی یک مجموعه معیارها را بر اساس نتایجی به دست می آوریم که از فرآیند قابل حصول هستند . شامل :

\begin{itemize}
	\item موازین خطاهای کشف شده، پیش از ارائه نرم افزار
	\item نقایص گزارش شده توسط کاربران نهایی
	\item محصولات کاری تحویل شده
	\item انرژی انسانی صرف شده
	\item زمان صرف شده
	\item مطابقت با برنامه زمان بندی
\end{itemize}



\subsection{معیارهای فرآیند توسط چه ویژگی هایی اندازه گیری می شوند ؟}

\begin{itemize}
	\item انرژی و زمان صرف شده در اجرای فعالیت های چتری
	\item فعالیت های کلی مهندسی نرم افزار
\end{itemize}



\subsection{چند مورد از معیار های خصوصی را نام ببرید؟}

\begin{itemize}
	\item تعداد نقایص ( توسط فرد )
	\item تعداد نقایص ( توسط پیمانه )
	\item خطاهای یافت شده در اثنای توسعه
\end{itemize}

\subsection{معیارهای عمومی}
معیارهای عمومی اطلاعاتی را جذب می کنند که در آغاز برای افراد جنبه خصوصی داشتند .


\subsection{بهبود آماری فرآیند نرم افزاری یا SSPI را توضیح دهید؟}
هرچه سازمانی در جمع آوری و استفاده از معیارهای فرآیند احساس راحتی بیشتری داشته باشد ، به دست آوردن شاخصهای ساده روشی دقیق تر را نتیجه می دهد که به آن SSPI می گویند .

SSPI از تحلیل شکست نرم افزار برای جمع آوری داده های مربوط به همه ی خطاها و نقایص مشاهده شده در اثنای  توسعه و به کارگیری سیستم یا محصول استفاده می کند .



\subsection{مراحل تحلیل شکست را توضیح دهید ؟}


\begin{itemize}
	\item همه ی خطاها و نقایص ،بر حسب منشا، دسته بندی می شوند ، مثل : 
	\begin{itemize}
		\item نقص در مشخصات
		\item  نقص در منطق
		\item عدم تطابق با استاندارد ها
	\end{itemize}
	\item هزینه تصحیح هر خطا و نقص ثبت می شود
	\item تعداد خطاها و نقایص در هر دسته شمارش و به ترتیب نزولی مرتب می شوند
	\item هزینه کلی خطا و نقایص در هر گروه محاسبه می شود
	\item داده های حاصل مورد تحلیل قرار می گیرند تا دسته هایی که منجر به بالاترین هزینه ها در سازمان می شوند، کشف شوند .
	\item تدابیری برای اصلاح فرآیند با هدف حذف دسته ای از خطاها و نقایص که بیشترین هزینه را دارند اندیشیده می شود .
\end{itemize}




\subsection{نمودار ستون فقرات ماهی گریدی را توضیح دهید؟}


نمودار ستون فقرات ماهی ( خط مرکزی ) نشانگر عامل کیفیت مورد نظر است ، هر یک از دنده ها که به ستون فقرات متصل است نشانگر علل بالقوه برای مشکل کیفیتی است ( از قبیل کمبود خواسته ها، مشخصه ای مبهم ، خواسته های نادرست ، تغییر خواسته ها ) سپس ستون فقرات و دنده ها به هر یک از دنده های اصلی نمودار، افزوده می شوند تا علت آن گسترش داده شود .

با تحلیل یک نمودار استخوان ماهی، می توان شاخصهایی به دست آورد که سازمان نرم افزاری را قادر می سازد  که در جهت کاهش دادن فرکانس خطاها و نقایص فرآیند را اصلاح کند .







\subsection{هر پروژه باید چه مواردی را اندازه گیری کند؟}


\begin{itemize}
	\item ورودی ها
	\item خروجی ها
	\item نتایج
\end{itemize}



\subsection{اندازه گیری های جهان فیزیکی را به چند گروه می توان تقسیم کرد ؟}


\begin{itemize}
	\item موازین مستقیم
	\item موازین غیر مستقیم
\end{itemize}



موازین مستقیم فرآیند مهندسی نرم افزار، شامل هزینه و انرژی به کار رفته است .

موازین مستقیم محصول، شامل تعداد خطوط کد تولید شده ( LOC ) ، سرعت اجرا ، اندازه حافظه و نقایص گزارش شده در یک مدت زمان تعیین شده است .

موازین غیر مستقیم محصول عبارتند از : عملکرد ، کیفیت ، پیچیدگی ، بازدهی ، قابلیت نگهداری 



\subsection{تقسیم بندی دامنه معیارهای نرم افزار}

\begin{itemize}
	\item معیار های فرآیند
	\item معیار های پروژه
	\item معیار های محصول
\end{itemize}



\subsection{معیار های جنبه ی خصوصی، معیارهای جنبه ی عمومی}
معیارهایی از محصول که برای افراد جنبه ی خصوصی دارند غالباً ترکیب می شوند و معیارهای پروژه را توسعه می دهند که برای تیم نرم افزاری جنبه ی عمومی دارند .



\subsection{معیارهای اندازه گرا}

معیارهای اندازه گرا از نرمال سازی موازین کیفیتی و یا بهره وری و با در نظر گرفتن اندازه نرم افزار به دست می آیند .



\subsection{تعداد خطوط کد}
برای توسعه ی معیارهایی که با معیارهای مشابه حاصل از پروژه های دیگر قابل امتزاج باشد، تعداد خطوط کد را به عنوان مقدار نرمال سازی انتخاب می کنیم .


\subsection{مجموعه ای از معیارهای اندازه گیری که برای هر پروژه می توان استفاده کرد؟}

\begin{itemize}
	\item تعداد خطا به ازای هر هزار خط کد
	\item نقایص به ازای هر هزار خط کد
	\item هزینه ی مصرفی برای هر خط
	\item تعداد صفحات مستند به ازای هر هزار خط کد
	\item تعداد خطاها به ازای هر نفر در ماه
	\item تعداد خط به ازای هر نفر در ماه
	\item هزینه ی مصرفی به ازای هر صفحه از مستندات
\end{itemize}






\subsection{معیارهای عملکردگرا}
معیارهای عملکردگرا از میزان عملکرد ارائه شده توسط برنامه کاربردی، به عنوان مقداری برای نرمال سازی استفاده می کنند .

\subsection{نقطه عملکرد $( Function\:\:Point )$ چیست ؟ }
نقاط عملکرد، با استفاده از یک رابطه ی تجربی مبتنی بر موازین قابل شمارش ( مستقیم ) از دامنه ی اطلاعات نرم افزار و ارزیابی پیچیدگی نرم افزار به دست می آیند .

هنگامی که نقاط عملکرد محاسبه شدند، از آنها به شیوه ای مشابه LOC برای نرمال سازی موازین مربوط به بهره برداری نرم افزار، کیفیت و صفات دیگر استفاده می شود .

\subsection{مقادیر دامنه اطلاعات به چه صورت تعیین می شوند؟}

\begin{itemize}
	\item تعداد ورودی های کاربر
	\item تعداد خروجی های کاربر
	\item تعداد درخواست های کاربر
	\item تعداد فایلها
	\item تعداد واسط های خارجی
\end{itemize}



\subsection{هدف اصلی مهندسی نرم افزار}
هدف اصلی مهندسی نرم افزار، تولید سیستم، برنامه کاربردی یا محصولی با کیفیت بالاست .




\subsection{مجموعه عوامل کیفیتی مک کال و کارائو برای توسعه ی کیفیت نرم افزار، نرم افزار را از چند دیدگاه ارزیابی می کند ؟}

\begin{itemize}
	\item کارکرد محصول ( استفاده از آن )
	\item بازبینی محصول ( تغییر دادن آن )
	\item گذار محصول ( اصلاح آن برای کار کردن در محیطی متفاوت )
\end{itemize}



\subsection{شاخص های کیفیت نرم افزار را نام ببرید ؟}

\begin{itemize}
	\item درستی
	\item قابلیت نگهداری
	\item یکپارچگی
	\item قابلیت استفاده
\end{itemize}


\subsection{درستی}
یک برنامه باید درست کار کند وگرنه برای کاربران خود ارزشی ندارد ، متداولترین میزان برای درستی ، تعداد نقایص به ازای هر KLOC می باشد .



\subsection{قابلیت نگهداری}
قابلیت نگهداری عبارت است از سهولت تصحیح یک برنامه در صورت مواجهه با خطا، تطابق با تغییرات محیط یا بهبود بخشیدن به برنامه در صورت تغییر یافتن خواسته های مشتری .

راهی برای اندازه گیری مستقیم قابلیت نگهداری وجود ندارد،

بنابراین باید از موازین غیر مستقیم استفاده کرد 

یک معیار مبتنی بر زمان ، میانگین زمان لازم برای تغییر ( MTTC ) است، یعنی زمان لازم برای :
\begin{itemize}
	\item تحلیل تقاضای تغییر
	\item طراحی یک اصلاح مناسب
	\item پیاده سازی تغییر
	\item آزمون آن
	\item و توزیع تغییر در میان همه ی کاربران
\end{itemize}



\subsection{یکپارچگی}
یکپارچگی، میزان توانایی سیستم برای تحمل حملاتی ( چه تصادفی و چه عمدی ) است که متوجه امنیت آن می شود .

این حملات روی هر سه مولفه نرم افزار صورت می پذیرد :

\begin{itemize}
	\item برنامه ها
	\item داده ها
	\item مستندات
\end{itemize}


\subsection{قابلیت استفاده}
اگر قابلیت استفاده از برنامه ای آسان نباشد، سرنوشت آن شکست خواهد بود .

قابلیت استفاده کوششی برای تعیین کمّی سهولت استفاده بر حسب چهار خصوصیت قابل اندازه گیری است :

\begin{itemize}
	\item مهارت فیزیکی و هوشی لازم برای فراگیری سیستم
	\item زمان لازم برای بازدهی خوب در به کار گیری سیستم
	\item افزایش خالص در بهره وری نسبت به روشی که جایگزین می شود
	\item ارزیابی موضوعی از احساس کاربران نسبت به سیستم ( که غالباً از طریق پرسشنامه به دست می آید )
\end{itemize}



\section{فصل پنجم - برنامه ریزی برای پروژه ی نرم افزاری}


\subsection{برنامه ریزی پروژه چیست؟}

مدیریت پروژه نرم افزاری با مجموعه ای از فعالیت ها آغاز می شود که جمعاً برنامه ریزی برای پروژه نامیده می شود .


\subsection{برنامه ریزی پروژه شامل چه مواردی می شود ؟}

شامل برآوردی برای تعیین میزان پول، کار ، منابع و زمان لازم برای ساخت یک محصول یا سیستم نرم افزاری مشخص .


\subsection{هنگام انتخاب مدیر پروژه کدام ویژگی اهمیت بیشتری دارد ؟}
\begin{itemize}
	\item بداند چه چیزی خراب می شود، قبل از آنکه این اتفاق رخ دهد
	\item جرات برآورد آینده را داشته باشد
\end{itemize}



\subsection{برآورد منابع، هزینه و زمانبندی برای کارهای مهندسی نرم افزار به چه ویژگی هایی نیاز دارد؟}

\begin{itemize}
	\item تجربه
	\item دستیابی به اطلاعات تاریخی خوب
	\item جرات انجام پیش بینی های کمّی در هنگام وجود اطلاعات صرفاً کیفی
\end{itemize}


\subsection{نکته - عدم قطعیت در برآورد}
برآورد ذاتاً با ریسک همراه است و همین ریسک به عدم قطعیت منجر می شود 


\subsection{تاثیر پیچیدگی بر عدم قطعیت}
پیچیدگی پروژه اثری شدید بر عدم قطعیت ذاتی برنامه ریزی دارد .

\subsection{نسبی بودن پیچیدگی}
پیچیدگی یک میزان نسبی است که از آشنایی با کارهای قبلی تاثیر می پذیرد .


\subsection{تاثیر اندازه پروژه بر برآورد}
اندازه پروژه عامل مهم دیگری است که  می تواند ، صحت و موثر بودن برآورد ها را تحت تاثیر قرار دهد .

\subsection{درجه ی عدم قطعیت ساختاری}
درجه ی عدم قطعیت ساختاری نیز بر ریسک برآورد تاثیر می گذارید . 


\subsection{منظور از ساختار در کلمه ی 'درجه ی عدم قطعیت ساختاری'}

\begin{itemize}
	\item درجه ای از یکنواختی خواسته ها
	\item سهولت تقسیم کردن وظایف
	\item ماهیت سلسله مراتبی اطلاعاتی
\end{itemize}




\subsection{امکان سنجی یا امکان پذیری نرم افزار چند بعد دارد ؟}

\begin{itemize}
	\item فناوری
	\item مالی
	\item زمان
	\item منابع
\end{itemize}



\subsubsection{فناوری}


\begin{itemize}
	\item آیا پروژه از نظر فنی امکان پذیر است ؟
	\item آیا می تواند پیشرفت داشته باشد ؟
	\item آیا نقایص را می توان تا آن حد کاهش داد، که با نیازها تطبیق پیدا کند ؟
\end{itemize}



\subsubsection{مالی}


\begin{itemize}
	\item آیا از نظر مالی امکان پذیر است ؟
	\item آیا می توان پروژه را با هزینه ی مناسب  برای مشتری به پایان رساند ؟
\end{itemize}


\subsubsection{زمان}


\begin{itemize}
	\item آیا زمان تحویل پروژه به بازار مناسب هست ؟
\end{itemize}


\subsubsection{منابع}


\begin{itemize}
	\item آیا سازمان دارای منابع انسانی، سخت افزاری ، نرم افزاری و . . .  لازم برای موفقیت هست ؟
\end{itemize}



\subsection{منابع}

دومین هدف برنامه ریزی برای نرم افزار، برآورد منابع لازم برای توسعه ی نرم افزار است .


\subsection{انواع منابع توسعه}

\begin{itemize}
	\item محیط توسعه ( ابزار های سخت افزاری و نرم افزاری )
	\item مولفه های نرم افزاری قابل استفاده مجدد
	\item افراد
\end{itemize}



\subsection{ویژگی های مشخص کننده منابع را نام ببرید؟}

\begin{itemize}
	\item توصیف منبع
	\item بیانی از قابلیت دسترسی
	\item موعد زمانی نیاز به آن منبع
	\item مدت زمان به کارگیری آن
\end{itemize}




\subsection{منابع نرم افزاری قابل استفاده مجدد}

یعنی ایجاد و استفاده مجدد از مولفه های سازنده


مولفه های سازنده :

\begin{itemize}
	\item برای رجوع آسان در یک کاتالوگ لیست می شوند
	\item برای کاربرد آسان استاندارد می شوند
	\item برای یکپارچگی اعتبار سنجی می شوند
\end{itemize}



\subsection{چهار دسته از مولفه های نرم افزاری قابل استفاده مجدد}


\begin{itemize}
	\item مولفه های آماده
	\item مولفه های تجربه کامل
	\item مولفه های تجربه ناقص
	\item مولفه های جدید
\end{itemize}




\subsection{مولفه های آماده}
نرم افزارهای موجود که از یک شخص ثالث قابل دسترسی هستند یا برای یک پروژه قبلی توسعه یافته اند .


\subsection{مولفه های تجربه کامل}

مولفه های توسعه داده شده برای پروژه های گذشته که مشابه نرم افزار فعلی هستند، اعضای تیم نرم افزار فعلی در حیطه ی کاربرد ارائه شده توسط این مولفه ها تجربه کامل دارند و اصلاحات مورد نیاز برای این مولفه ها، ریسک نسبتاً پایینی دارد .



\subsection{مولفه های تجربه ناقص}

مولفه هایی که برای پروژه های گذشته توسعه داده شده اند و با نرم افزار پروژه فعلی در ارتباط اند ولی نیاز به اصلاحات اساسی دارند . اعضای تیم نرم افزار فعلی در حیطه ی کاربرد این مولفه ها تجربه محدودی دارند ، از این رو اصلاحات مورد نیاز برای مولفه های تجربه ناقص، دارای درجه ی بالایی از ریسک هستند .


\subsection{مولفه های جدید}
مولفه های نرم افزاری که تیم نرم افزاری باید مشخصاً برای نیازهای پروژه فعلی بسازند .



\subsection{دستورالعمل هایی برای استفاده از مولفه های قابل استفاده مجدد}


\begin{itemize}
	\item اگر مولفه های آماده نیازهای پروژه را برآورده می کنند آنها را به دست آورید ، هزینه ی به دست آوردن و سر هم کردن مولفه های آماده همواره کمتر از هزینه ی توسعه ی نرم افزار هم ارز است . به علاوه درجه ی ریسک هم پایین است .
	\item اگر مولفه های تجربه کامل در دسترس اند، ریسک ناشی از اصلاح و سرهم بندی عموماً قابل قبول است ، در برنامه ریزی پروژه باید استفاده از این مولفه ها منعکس شود .
	\item اگر مولفه های تجربه ناقص در دسترس باشند، استفاده از آنها برای پروژه فعلی باید مورد تحلیل قرار گیرد ، هزینه ی اصلاح مولفه های تجربه ناقص، گاه می تواند بیش از هزینه ی لازم برای توسعه ی یک نرم افزار جدید باشد .
\end{itemize}



\newpage

\section{نمونه سوالات}

\begin{enumerate}
	\item بحران نرم افزار و عوامل موثر بر آن را تشریح کنید ؟
	\item نرم افزار های بلادرنگ و توکار را تشریح کنید ؟
	\item مدل ساخت نمونه و مدل افزایشی را شرح داده و فازهای آن را با رسم شکل بیان نمایید ؟
	\item بر اساس نظریه ی مانتی چهار الگوی سازمانی برای تیم های مهندسی نرم افزار را تشریح کنید ؟
	\item اصل $W5HH$ را تشریح کنید ؟
	\item امکان سنجی برای یک پروژه نرم افزاری چیست ؟ ابعاد آن کدام است، تشریح نمایید ؟
	\item موازین کیفیت نرم افزار را نام برده و یکی را به دلخواه تشریح کنید ؟
	\item انواع منابع نرم افزاری قابل استفاده مجدد را نام برده و تشریح نمایید ؟
	\item نرم افزار های سیستمی و زمان حقیقی را تشریح نمایید ؟
	\item مدل ساخت ترتیبی خطی و مدل افزایشی را شرح داده و فازهای آن را با رسم شکل بیان کنید ؟
	\item منظور از قابلیت نگهداری و یکپارچگی نرم افزار چیست ؟ 
\end{enumerate}


\end{document}